% tlpguide.tex
% v1.0, released 24 Mar 2021
% Copyright 2021 Cambridge University Press

\documentclass{tlp}

\usepackage{my_prelude}
\usepackage{infer}
\usepackage{assume}
\usepackage{dynamic_run}
\usepackage{static_check}
\DeclareUnicodeCharacter{03BB}{$\lambda$}

\bibliographystyle{tlplike}

\begin{document}

\lefttitle{Davide Fissore and Enrico Tassi}

\jnlPage{1}{8}
\jnlDoiYr{2021}
\doival{10.1017/xxxxx}

\title[Determinacy Checking for HO Prolog with Cut]{
  Determinacy Checking for Elpi}

\begin{authgrp}
\author{\gn{Davide Fissore}}
\affiliation{Université Côte d'Azur, Inria\thanks{This work has been supported by the French government, through the France 2030
investment plan managed by the Agence Nationale de la Recherche, as part of the
``UCA DS4H'' project, reference ANR-17-EURE-0004.}}
\author{\gn{Enrico Tassi}}
\affiliation{Université Côte d'Azur, Inria}
\end{authgrp}

\history{\sub{xx xx xxxx;} \rev{xx xx xxxx;} \acc{xx xx xxxx}}

\maketitle

\begin{abstract}
We study the determinacy checker for a higher-order logic programming language
with cut and matching.
We study the determinacy checker for a higher-order logic programming language
with cut and matching.
We study the determinacy checker for a higher-order logic programming language
with cut and matching.
We study the determinacy checker for a higher-order logic programming language
with cut and matching.

\end{abstract}

\begin{keywords}
Functionality, Determinacy Analysis, Higher Order, Logic Programming, Cut
\end{keywords}



\section{Introduction}

We are interested in the static analsysis of Elpi programs, in particular
in checking their determinacy. Elpi is a higher oreder logic programming
language, a dialect of $\lambda$Prolog~\cite{dale} well suited to manipulate
incomplete syntax trees with binders~\cite{lpar,journal}.
Elpi finds applications as an extension
language for The Rocq\footnote{formerly known as Coq} Interactive prover, where
Elpi has been used for program and proof synthesis~\cite{derive1,derive2,hb} and more recently
as the target language for type class resolution~\cite{coqws,ppdp}.
Type class resolution is typically used to implement overloading~\cite{haskell,ms}
where the solution to a query provides the meaning of the overloaded symbol.
In this context it is of paramount importance that this solution is unique,
i.e. non ambiguous. A static check for this property is of particular
interest in the context of the platform of Rocq libraries: code developed
by different teams is combined together, reused in order to lower the cost
of mechanization.

As of typday Elpi comes with a quite standard type checker~\cite{pf} but
features no mode nor determinacy analisys. The literature provides
many works on the subject~\cite{robadescrittadopo} but none of these
works can be applied to Elpi due to its higher order nature, inherited from
$\lambda$Prolog, and its nonstandard notion of input.

Our contributions is a  determinacy checking algorithm that
\begin{itemize}
\item covers logic programs with cut
\item covers higher order logic programming constructs such as first class predicates and clauses
\item can be applied to pre-existing code bases, i.e. tracks miscalled functions rather than aborting
\end{itemize}
Last but not least we provide an operational semantics for higher order logic programs
with cut that poses the bases for our definitions and paves the way to a
mechanization of the determinacy checking algorithm.

\subsection{Motivating examples}

A first motivating example is the \elpiIn{map} predicate, widely used in
all Elpi applications. The first line gives the signature of \elpiIn{map}:
given a predicate between any types \elpiIn{A} and \elpiIn{B},
it relates a list of \elpiIn{A} with a list of \elpiIn{B}. Elpi
follows the $\lambda$Prolog convention ($\lambda$-0calculus actually) of
writing application with no parentheses, e.g. \elpiIn{map F L R}
can be understood as the atom \elpiIn{map(F, L, R)}.

\begin{elpicode}
pred map i:(pred i:A, o:B), i:list A, o:list B.
map _ [] [].
map F [X|XS] [Y|YS] :- F X Y, map F XS YS.
\end{elpicode}

This code happens to compute a function of the first two arguments, that we
consider inputs, if and only if the higher order predicate \elpiIn{F} is a
function and if the first list is ground. We want to author of this code
to be able to ascribe a more precise signature on the code above, namely:

\begin{elpicode}
func map (func A -> B), list A -> list B.
\end{elpicode}

The syntax \elpiIn{func name? inputs -> outputs} asserts that name is
a function of the inputs (before the arrow) to the outputs if all the
requirements on the inputs are satisfied, in this specific case if the
first input is a functional, binary, predicate. We need this ascription to
be given on an existing code base where \elpiIn{map} is potentially
called by passing a relation as the first argument, say \elpiIn{P}.
In this case the call to \elpiIn{map P L R} has to be accepted but not
coinsidered to be functional for the analisys of the surrounding code.


\begin{elpicode}
func mask! list A, list A -> list bool.
mask! Bad L R :- map (x\y\ if mem! x Bad then y = ff else y = tt) L R. % ok

func mask list A, list A -> list bool. % error
pred mask i:list A, i:list A, o:list bool. % ok
mask Bad L R :- map (x\y\ if mem X Bad then Y = ff else Y = tt) L R.
\end{elpicode}

The cut oeprator is the privileged way to impose functionality on a relation
by committing to its first result.

\begin{elpicode}
func once (pred) -> .
once P :- P, !.

pred mem i:A, i:list A.
mem X [X|_].
mem X [_|XS] :- mem X XS.

func mem! A, list A -> .
mem! X XS :- once (mem X XS).
\end{elpicode}

The signature of \elpiIn{once} states that the higher order argument
\elpiIn{P} can be a predicate, but still \elpiIn{once P} acts as a function.
The determinacy analysis we present tracks functionality from the inputs
to outputs, that can be themselves predicates.

    
\begin{elpicode}
func force (pred) -> (func).
force P (once P).

func foo (pred), list A -> list B.
foo P L R :- force P F, map L F R.
\end{elpicode}

Here the output of \elpiIn{force} is a function \elpiIn{F} and in turn
makes \elpiIn{map L F R} produce a single value for \elpiIn{R} out
of \elpiIn{P} and \elpiIn{L}, making \elpiIn{foo} itself a function.

The higher order term in Elpi (and $\lambda$Prolog) also applies to data
via the so called $\lambda$-tree syntax (also known as HOAS~\cite{dalemechaniz}).
 
\begin{elpicode}
kind tm type.
type app tm -> tm -> tm.
type lam (tm -> tm) -> tm.

func copy tm -> tm.
copy (app A B) (app C D) :- copy A C, copy B D.
copy (lam F) (lam G) :- pi x\ copy x x => copy (F x) (G x).

func whd tm -> tm.
whd (app H A) R :- whd H (lam F), !, pi x\ copy x A => copy (F x) R.
whd X X.
\end{elpicode}

todo explain, can we have the ad hoc rule for pi x?



% We have a rule based language integrated in Coq. Rules are useful
% to model a grown knowledge base (extend existing programs) and manage
% the context in HOAS.

% We want to add to it some of the benefits of functional programming,
% eg statically enforce the absence of global backtracking. Eg twice the same
% rule can turn linear into exponential. Even worse two overlapping rules can
% inadvertently change the meaning when loading two libraries.

% checking or inferring functionality of relations is studied in the literature,
% but does only partially cover HO programming but no HOAS nor homoiconicity, all features
% that are widely used. The former, as in FP, to reuse code via HO iterators,
% eg map. The second for manipulating syntax with binders. The third to
% have programs that extend themselves by synthesizing rules.

% in practice a function is a relation where 1) we identify the arguments
% that are seen as input 2) we prove that the outputs are uniquely determined
% by the inputs. The first part is called mode analysis. We study both
% in the HO setting, eg $\lambda$Prolog, in the dialect of Elpi that has
% a special runtime input mode.

% \section{Motivating examples explained}

% We give a short intro to Elpi and functional analysis with examples
% that cover the use cases we want to cover.

% Convention that capitals are parameters, programs are written in rules
% preceeeded by a signature. prop is the type of predicate, eg code that runs.
% List syntax is bla bla. :- separates the head from the body, the head
% is unified with the goal, then each premise is executed.

% \subsection{Higher order programming}

% A typical HO predicate is map that takes a relation  in A x B
% to a relation in list A x list B.

% \begin{elpicode}
% type map (A -> B -> prop) -> list A -> list B -> prop.
% map _ [] [].
% map F [X|XS] [Y|YS] :- F X Y, map F XS YS.
% \end{elpicode}

% Explain that unlike functional languages, command and expressions are not
% mixed, and prop stuff is executable, hence you don't put F X in place of Y
% but rather run F X Y.

% Also explain that a relation that can go both sides, but
% that this feature is not very useful, for example it does not always
% work and if one calls passing a wrong relation, it is easy to diverge. 

% find a simple example.

% \begin{elpicode}
% filter P [] []
% filter P [X|XS] [X|YS] :- P X, !, filter P XS YS.
% filter P [_|XS] YS :- filter P XS YS.
% \end{elpicode}
  
% We want to annotate with usage info.

% \begin{elpicode}
% pred map i:(pred i:A, o:B), i:list A, o:list B.
% \end{elpicode}

% Explain syntax (pred [name] X, ... = X -> .. -> prop).

% This first step seems to be stringent for little reason, since map can work
% both ways. But it necessary to further refine the annotation with functionality
% assertions.

% \begin{elpicode}
% func map (func A -> B), list A -> list B.
% \end{elpicode}

% Explain syntax (func [name] X, ... -> Y, .. = pred i:X, .. , o:Y, ..).

% Note (map succ) is func list int -> list int (assuming succ is a function).

% \subsection{Higher Order Abstract Syntax programming}

% In HOAS, we want out analysis to accept this code that adds a dynamic rule

% \begin{elpicode}
% kind tm type.
% type app tm -> tm -> tm.
% type lam (tm -> tm) -> tm.

% func copy tm -> tm.
% copy (app F A) (app G B) :- copy F G, copy A B.
% copy (lam F) (lam G) :- pi x\ copy x x => copy (F x) (G x).

% func whd tm -> tm.
% whd (app H A) R :- whd H (lam F), whd (F A) R.
% whd (lam _ as X) X.

% kind ty type.
% type arr ty -> ty -> ty.

% func of tm -> ty.
% of (app F A) T :- of F (arr S T), of A S.
% of (lam F) (arr S T) :- pi x\ of x S => of (F x) T.
% \end{elpicode}

% \subsection{Homoiconicity}

% Since the language is homoiconic we also want this to pass

% \begin{elpicode}
% func comp (func A -> B), (func B -> C) -> (func A -> C).
% comp F G X Y :- F X Tmp, G Tmp Y.

% func fuse (func A -> B) -> (func A -> B).
% fuse (comp (map F) (map G)) (map H) :- fuse (comp F G) H.
% fuse X X.
% \end{elpicode}

% But we don't want to break code that uses map as a relation

% \begin{elpicode}
% func do list (func) -> .
% do [].
% do [P|PS] :- P, do PS.

% pred do-with-trace i:list (pred).
% do-with-trace Code :- map spy Code InstrumentedCode, do Code.
  
% func spy (pred) -> (pred).
% spy P (do [print "before", P, print "after"]).
% \end{elpicode}

% Finally, we want to take advantage of cut across iterators

% \begin{elpicode}
% func once (pred) -> .
% once P :- P, !.

% func do! list (pred) -> .
% do! [] [].
% do! [P|PS] :- once P, do! PS.

% % alternative

% func tcut (pred) -> (func).
% tcut R F :- F = once R.

% func do! list (pred) ->.
% do! LR :- map tcut LR LF, do LF.
% \end{elpicode}

% Before providing a formal definition of determinacy and mode analysis, we
% first describe the language we are working with, namely a dialect of \lamprolog
% called \elpi.

% Since we have structured our paper into two main sections: 1) the first
% discusses a basic version of \elpi with only Horn clauses and the cut operator,
% and 2) the second extends this version by introducing hereditary-Harrop
% formulas. Below we give a simple presentation of some basic 
% objects of our language, they will be expanded in next section when needed.
% \\

% This section lays the foundation for our static analysis of Elpi by
% focusing the fragment of Horn clauses with cut.

\section{Preliminaries}
\label{sec:basic}
\todo{check for dead cod here if we hide the proofs}

We use $\EmptyList$ for the empty list and $x :: xs$ for prepending an element $x$
to a list $xs$; $@$ for list concatenation;
$[e|x \in \vec{x}]$ for list comprehension and $[e\ \mathbf{if}\ p|x \in \vec{x}]$ 
for list filtering. We shall fold $f$ over a list of pairs
using the combinators \fold and \map defined by these equations:
$$
\begin{array}{lll}
\fold\ \_\ \_\ \bot = \bot & \fold\ f\ (x :: xs)\ a = \fold\ f\ xs\ (f\ x\ a) & \fold\ \_\ \EmptyList\ a = a \\
\map\ \_\ \EmptyList = \EmptyList & \map\ f\ (x :: xs) = f\ x :: \map\ f\ xs
\end{array}
$$

We asume a set $P$ of predicate names, a set $K$ of term constructors (e.g. \textit{nil}
and \textit{cons}), a set \X of variables and by convention we use capitals for
unification variables and small letters for bound variables.
As this paper does not really focuses on type checking (the interested reader can
refer to~\cite{1992nadathur}) we collaps all non-predicate types to
a single node \texttt{exp}. We also rule out predicates inside data,
eg a list of predicates, for space constraints.

We focus on predicate signatures: By convention we say that predicates take first
all input arguments and then output arguments and we separate them with $\funsep$,
omitting the symbol when both are absent. \detI{} stands for deterministic (the
precise meaning is given in~\cref{sec:thm}) while \relI{} for non-deterministic.

We write $\vecL{t}$  for $t_1 \ldots t_n$ (a possibly empty list of $t$s)
and we write $\vecL{tu}$ for the zipping of $\vecL{t}$ and $\vecL{u}$
into a list of pairs.

\newcommand{\syntaxFig}{
  \begin{subfigure}[b]{1\textwidth}
    \centering
  $$
  \begin{array}{rlr}
    P & ::= p, q, \ldots & \mathrm{predicate}\\
    K & ::= c, f, \ldots & \mathrm{data\ constructor}\\
    \F & ::= \detI \mid \relI & \mathrm{determinacy} \\
    \X & ::= \mathrm{X}, \mathrm{Y} \ldots x, y, \ldots & \mathrm{variable}\\
    \A & ::= \cut \mid P\ \vecL{\T} \mid \X\ \vecL{\T} \mid \piImplCmd[\texttt{exp}]{\X}{\CL}{\A} & \mathrm{atom} \\
    \T & ::= K\ \vecL{\T} \mid P \ \vecL{\T} \mid \X\ \vecL{\T} \mid \lambda \X\!\!: ty \bs\ \T & \mathrm{term}\\
    \CL & ::= \clauseCmd{P}{\vecL{\T}}{\vecL{\A}} & \mathrm{clause} \\
    \V & ::= \mathrm{A}, \mathrm{B}, \ldots & \mathrm{type\ variable} \\
    sig & ::= \F\ \vecL{ty} \funsep \vecL{ty} & \mathrm{predicate\ signature} \\
    ty   & ::= \texttt{exp} \mid sig \mid \V & \mathrm{type} \\
  \end{array}
  $$
  \caption{Syntax of Elpi}
  \label{fig:syntax}
\end{subfigure}
}

\begin{figure}
  \fbox{\syntaxFig}
\end{figure}

The user provides a context \ctx{} that assigns a
signature $sig$ to each predicate symbol \pred{}. For example, looking back at section~\ref{sec:examples},
we have that $\ctx~\texttt{likes} = \dtype{\relI}{\texttt{exp}}{\texttt{exp}}$
and $\ctx\ \texttt{true} = \relI$.
We single out the cut operator since it has a dedicated treatment
in the operational semantics. Unification of \elpiIn{s} with \elpiIn{t} can be
encoded as \elpiIn{eq s t} for a predicate with a single clause
$\prog\ \texttt{eq} =$\elpiIn{(eq X X :- .)} with signature
$\ctx\ \texttt{eq} = \dtype{\detI}{}{A\ A}$: a deterministic predicate
with two outputs of the same type.
Given a call $\pred\ \vecL{t}$ to a n-ary predicate where
$\ctx\ \pred = \dtype{\relI}{{ty}_1 \ldots {ty}_k~}{~{ty}_{k+1} \ldots {ty}_n}$ we
write $\vecL{t_i}$ for the input arguments ${t}_1 \ldots {t}_k$
and $\vecL{t_o}$ for the output arguments ${t}_{k+1} \ldots {t}_n$.

$\lambda$ and \texttt{pi} are binders in the $\lambda$-calculus sense. We write
$t[x/y]$ the usual, capture avoiding, operation or replacing the variable $y$
with $x$ inside $t$.
We say that $\vars\ t \subseteq \mathcal{P}(\X)$ is the set
of free variables occurring in $t$. When $\vars\ t = \emptyset$ we say that
$t$ is \ground.

% \begin{minipage}{0.48\textwidth}%
%   \vspace{-1em}
%   \begin{align}
%     tm     & ::= \cut \mid \predVar\ \vecL{tm} \mid tm = tm \label{eq:tm} \\
%     ty     & ::= data \mid \dtype{\relI}{data^\ast}{data^\ast} \ \label{eq:ty} \\
%     data   & ::= \texttt{c}\ data^\ast \mid \predVar \label{eq:data}
%   \end{align}  
% \end{minipage}
% \begin{minipage}{0.48\textwidth}
%   \vspace{-1em}
%   \begin{align}
%     clause & ::= \clauseCmd{\pred}{\vecL{tm}}{\vecL{tm}}                  \label{eq:cl}  \\
%     goal   & ::= \goalCmd{\prog}{tm}{\vecL{alt}} \label{eq:goal}                            \\
%     alt    & ::= \texttt{subst} * \vecL{goal} \label{eq:alt}
%   \end{align}  
% \end{minipage}

% In this simple

% Terms are shown in \cref{eq:tm}. They are made by the \cut\ operator
% (usually noted \elpiIn{!}); term application, we use the symbol \predVar\
% to indicate variable names, they can be either predicate names, i.e. constants
% of the program, or quantified variables. An application is followed by a
% vector of terms. Term unification is a special case of term application,
% we prefer to have a special case for it. 
% By convention we differentiate unification variables from constants
% by indicating the former with capital letters and the latters
% with lower case letters.



% Each term in the language has a type (\cref{eq:ty}). The type is either a
% $data$, that is the type of expressions or a $pred$ the type for predicates,
% that is the type of executable piece of code. Predicates are parametrized by
% arguments whose type are of type $data$. $data$ (\cref{eq:data}) are made by
% constants applied to list of data, we have for example the type \elpiIn{int} or
% \elpiIn{list}, and \predVar\ stands for variables, allowing therefore to have
% polymorphism.

% A clause (\cref{eq:cl}) is made of a variable name, its list of arguments and a body made of a
% list of terms. This list of terms are called \textit{premises} and should be
% considered as a list of subgoals in conjunction.

\section{Operational semantics} % Horn Clauses with modes and cut}

A program $\prog = (\aleph, \mathcal{I})$ holds a set of names $\aleph$ and a
mapping $\mathcal{I}$ from predicate names to an ordered list of clauses.
We write $\prog~\pred$ for the clauses of predicate \pred in $\mathcal{I}$
and $(x,h) + \prog$ to add an extra clause $h$ to $\mathcal{I}$ and
name $x$ to $\aleph$.\todo{change order} We write $x \# \prog$ to find a name fresh in $\aleph$.



\newcommand{\runFig}{
  % \begin{subfigure}[b]{1\textwidth}
    \centering

  \ruleBangM{.50}
  \ruleStopM{.45}
  \vspace{0.3em}%

  \ruleCallM{1}
  $$
  \mathcal{F}(\prog, \pred\ \vec{t}, gl, \subst, a) :=
  \bigg[
    %\bigg(
    \subst['],
    %\Big(
    %\underbrace{
      \big[(\prog, g, a) \mid g \in \vecL{g}\big]
    %}_{\mathrm{premises}} @ ~ gl
    \ \mathbf{if}\ \mathcal{H}(\vecL{tu}, \subst) = \subst['] \not= \bot,
    %\Big)
    %\bigg)
    ~\bigg\rvert~
    (\clauseCmd{\pred}{\vec{u}}{\vecL{g}}) \in \prog\ \pred
    \bigg]
  $$
  $$
  \mathcal{H}(\vecL{tu},\ ol,\ \subst) = \fold\ \unify\ \vecL{tu_o}\ (\fold\ \match\ \vecL{tu_i}\ \subst)
  $$

  \ruleCallMF{0.5}
  \ruleCallMFH{0.44}

  \rulePiImplM{.7}
  \ruleCallBeta{0.44}

  \caption{Operational natural semantics}
  \label{fig:basic-interp}
% \end{subfigure}
}

\begin{figure}[t]
  \begin{framed}
  \runFig
  \end{framed}
\end{figure}

% We assume all predicates have at least one clause, in particular
% $\prog\ \texttt{fail} = [\clauseCmd{\texttt{fail}}{\!\!}{0 = 1}]$.
% These clauses are
% disjunctive, and the order in which they are explored follows the chronological
% order of their declaration.
%
% In our approach, we do not include built-in operators, keeping the language as
% minimalist as possible. For instance, there is no built-in \textit{or} operator.
% However, its behavior can be replicated by defining a custom \textit{or}
% predicate with two arguments and providing two implementations: one invoking the
% first argument and the other invoking the second.
%
% A program is a mapping from predicate names to clauses. By hypothesis, we assume
% that every program we work with from now on has been type-checked (see
% \cite{1992nadathur}). One of the role of type-checking is to ensure that
% commands and expressions are not mixed: commands, also called propositions, are
% pieces of executable code, whereas expressions are not. Instead, expressions
% serve to carry pieces of information during the execution of commands.
%
%   \begin{align}
%     clause & ::= \clauseCmd{\pred}{\vecL{tm}}{\vecL{tm}}                  \label{eq:cl}  \\
%     goal   & ::= \goalCmd{\prog}{tm}{\vecL{alt}} \label{eq:goal}                            \\
%     alt    & ::= \texttt{subst} * \vecL{goal} \label{eq:alt}
%   \end{align}  
%
%
A goal $\g \subseteq \prog \times \A \times \vecL{\alt}$
is a triple made of a program, an atom $g$ and a list of \emph{cut-to} alternatives.
An alternative $a \in \alt \subseteq \Sigma \times \vecL{\g}$ is a
pair made of a substitution and a list of goals.
A substitution $\subst : \X \to tm$ is a mapping from unification variables
to terms. 
We write $\subst t$ the application of a substitution \subst to a term $t$,
and we remark that $t$ is \ground{} iff $\forall \subst, \subst t = t$.
We write $\mathrm{dom}\ \subst$ for the set of variables occurring
in the domain or in the codomain of \subst, i.e.
$\dom\ \subst = \{ X\ |\ \subst X \mathrm{\ is\ defined\ } \lor X \in \vars\ (\subst Y) \mathrm{\ for\ some\ Y} \}$.
We write $\EmptySubst$ for the only
substition s.t. $\dom\ \EmptySubst = \emptyset$.
When two substitutions have disjoint domains we write $\sigma_1 \cup \sigma_2$
as the (disjoint) union of two substitutions.

We assume the unifier $\unify : tm \times tm \times \Sigma \to \Sigma \uplus \bot$
such that if $\unifyCmd{t_1}{t_2}{\subst}{\subst[']} \not= \bot$ then $\subst['] t_1 = \subst['] t_2$
and $\subst[']$ is the most general extension $\subst$
as in~\cite{1991miller-pf}.
We assume a matcher $\match : tm \times tm \times \Sigma \to \Sigma \uplus \bot$
such that if $\matchCmd{t}{p}{\subst}{\subst[']} \not= \bot$ then $\subst t = \subst['] p$
and $\subst[']t = \subst t$ and $\subst[']$ is the most general extension of  $\subst$,
i.e. \match{} does not assign variables in $t$ but only in $p$ that acts as a pattern.



The operational semantics of our language is given by a relation
$\run{} \subseteq \alt \times \vecL{\alt} \times (\vecL{\alt} \times \Sigma ~\uplus~ \bot)$.
We write \runCmd{gl}{a}{\subst}{a'}{\subst'}
when a goal list $gl$ under a substitution \subst and alternatives $a$
terminates with a substitution $\subst'$ and a remaining list of
(still unexplored) alternatives $a'$. We write  \runCmdF{gl}{a}{\subst}
when the execution halts: fails to solve one of the given goals and runs out of
alternatives. 
An initial query looks like \runQuery{[\goalCmd{\prog}{\pred\ \vecL{t}}{\EmptyList}]}{\EmptyList}

% In this section the program never changes during execution hence storing it
% in the goal seems useless, but the program will change dynamically
% in section~\ref{sec:hoas} where the implication operator \elpiIn{=>} is
% introduced.


  % $$
  % \mathcal{F}(\prog, \pred\ \vec{t}, gl, \subst, a) :=
  % \bigg[
  %   %\bigg(
  %   \subst,
  %   %\Big(
  %   \underbrace{\big[(\prog, t = u, ~a) \mid (t,u) \in \vecL{t,u}\big]}_{\mathrm{head\ unification}} @
  %   \underbrace{\big[(\prog, g, a) \mid g \in \vecL{g}\big]}_{\mathrm{premises}} @ ~ gl
  %   %\Big)
  %   %\bigg)
  %   ~\bigg\rvert~
  %   (\clauseCmd{\pred}{\vec{u}}{\vecL{g}}) \in \prog\ \pred
  %   \bigg]
  % $$


  % \ruleUnifM{.6}
  % \ruleFailM{.55}
  % \ruleAbortM{.4}
  
  %\vspace{0.3em}%

The rules for \run{} are given in~\cref{fig:basic-interp}.
The first rule to look at is the one for cut (\ref{rule:cut}).
Remark that the cut-to alternatives stored in each goal are a suffix
of the (global) alternatives to that goal: If the atom in the goal
is a cut, then the global alternatives are shortened to the cut-to ones.

This rule goes hand in hand with~\ref{rule:call} that stores
the current alternatives \alts in all the subgoals via
the function $\mathcal{F}$. This function is in charge of creating the
new alternatives $\alts['] :: al$;
the former is directly evaluated while
the rest prepended to the existing set of alternatives.
$\mathcal{H}$ filters applicable clauses by unifying their heads with
the goal. If no clause applies rule ~\ref{rule:callbacktrack} moves
to the next alternative (backtracks to the most recent choice point) if
any, otherwise rule ~\ref{rule:callabort} terminates.
A peculiarity of Elpi is that input arguments are \emph{matched} against
the corresponding terms in the head of the rule, while outputs are unified.

The stop rule~\ref{rule:stop} terminates as there are no more goals to be
solved and produces the current susbtitution and the yet to be explored
alternatives.

Rule~\ref{rule:piimpl} loads into the current program the new clause $h$
after having postulated a fresh symbol $y$ and replaced the bound $x$ for
$y$ everywhere.

The last rule~\ref{rule:beta} deals with the fact that an atom can be
a variable. If that variable is assgined in $\subst$ to a term
that happens to be a predicate, then it proceeds. We write $=_{\beta\eta}$
since the equational theory of $\lambda$Prolog (hence Elpi) requires
that~\cite{1991miller-pf} but plays no role in this paper.

% The rule \ruleImpl handles the atom \implCmd{H}{B}, and its derivation rule
% attempts to solve \( B \) under a program extended with the clause \( H \). The
% rule \rulePi introduces the name of the fresh variable into the program.  


This operational semantisc is essentially a big-step version of
the one given in~\cite{1990Vink} extended to the higher-order
constructs of $\lambda$Prolog and the input-matching behavior of Elpi.
The choice to use operational semantics
rather than denotational semantics (as in \cite{2011king}) stems from our
preference for maintaining a concrete representation of the current choice
points as well as the cut-to ones (the ones obtained when a cut is performed).
This semantics describes an SLD search strategy with a \textit{hard-cut} operator
in the sense of~\cite{2003Andrews}: it is cutting
away not only the later clauses of the same predicate, but also the alternative
clauses for subgoals that appear earlier in the clause premises.

\subsection{Digression on  the eagherness of $\mathcal{H}$}

The function $\mathcal{H}$ eagerly unifies all clauses with the goal while
a more natural (and efficient) semantics would be to just create the
alternatives and prepend to the list of goals unification the same unification
problems. The choice simplifies the formal threatment of determinacy 
in~\cref{sec:thm} and in particular it matches the \mutExclHeads condition.

From a practical stanpoint a
logic programming languages implementation can either index clauses
deep enough, or perform a program transformation consisting in putting
a tail cut in each clause for a deterministic predicate.


% The \textit{call rule} (\ref{rule:call}) deals with goals starting with a predicate
% call, \pred\ is the notation to represent predicate name. In this case the function $\mathcal{F}$ is called with the parameters as
% explained above and if the result of this operation is the list
% \ConsHd{b}\ConsTl{bs}, then a recursive call to \run\ is done by prepending $b$ to
% the list of remaining goals \g and $bs$ is prepended to the list of
% alternatives \alt.

% The combination of the \ref{rule:call} and \ref{rule:cut} rules is crucial for replicating
% the behavior of the cut. When a call to a predicate generates multiple rules as
% new choice points, the function $\mathcal{F}$ creates future disjunctive goals
% where the cut alternatives are set to the current list of alternatives \alt. In
% other words, if the alternatives before reaching a call to a predicate \pred are
% \alt and \prog is the current program, then if ``\prog \pred'' results in
% clauses $c_0, \dots, c_n$, any clause $c_j$ with $i < j \leq n$ will be
% discarded if clause $c_i$ contains a cut. Furthermore, if the body of $c_i$
% consists of the atoms $a_1, \dots, a_k, !, a_{k+2}, \dots, a_m$, then all choice
% points created during the evaluation of $a_1, \dots, a_k$ will also be pruned.
% This pruning consist simply in setting the cut alternatives to \alt.

% It takes program \prog, a predicate name \pred, a list of terms,
% a substitution $\subst$ and a list of alternatives \alt. For each clause
% \clauseCmd{p}{\vec{t'}}{bs}, it builds a new list of pairs where the first argument
% is the substitution $\subst$ and the second is the list of new goals to treat.
% This list is made by all the unification between the terms received at call
% time and the argument of the clause and is followd by the premises
% of the clause that have been transformed in a goal.


% The \textit{fail rule} (\ref{rule:backtrack}) consumes the list of alternatives if the
% first goal fails. A failure occurs if the goal at the head of the list is a call
% to a predicate with no clauses in the associated program or if it is a
% failing unification under the given substitution.
% In fact, this
% rule allows to break loop since it can non-determinalistically applied on any
% configuration, provided that the list of alternatives is not empty. It is
% possible to make the algorithm deterministic by chainging \ruleFail so that it
% is applied if the current goal is a call to a predicate with no alternatives,
% but we prefer to simplify our rule system.


% A variable is a term not yet instantiated. We can assign variables thanks to the
% \unify\ procedure. It is used each time a term like $t_1 = t_2$ is encountered
% while solving a goal. The notation \unifyCmd{t_1}{t_2}{\subst}{\subst[']} is
% the unification and between the terms $t_1$ and $t_2$. It also takes an initial substitution \subst
% which is updated into the final substition \subst[']. A substitution is a mapping from variables name to their
% assignment. An assignment is a term. Unification provides, when possible, a most general unifier
% between the two terms. 


% existing at moment of the creation of
% the goal. A query is a list of goals (noted \g in the following) in conjunction, whereas alternatives
% (\cref{eq:alt}, noted \alt in the following) represent a disjunction of goals.
% In particular,
% \g is evaluated by looking into the clauses inside \prog.
% \alt
% contains the choice-points used by the interpreter during the evaluation of a
% cut, the aim of this piece of data will be clarified further in the section
% dedicated to the interpreter implementation (see \cref{sec:basic-elpi}).

% The interpreter takes a program a query and returns a substitution if the query
% is a consequence of the program. Each time a predicate call has multiple clauses
% implementing it, it commits the first choice and keep the other as global
% alternatives. If the committed choice leads to a failure, the interpreter will try to
% execute the first alternative. This non-deterministic behavior can be controlled
% by the user thanks to the cut operator which allows to cut away unwanted choice
% points.

% We say that a clause applies on the goal if its head unifies with a goal
% and we say that it \textit{successfully} unifies with a goal if the clause
% applies on the goal and all premises in its body succeed.

% \begin{coqcode}
%   Definition pn := string. (*predicate names are strings*)
%   Definition vn := string. (*variable  names are strings*)
%   Inductive tm := 
%     | Call (p:pn) (i:tm) (o:tm)  (*binary predicates taking integers*)
%     | Cut                        (*the cut operator*)
%     | Unify (t1:tm) (t2:tm)      (*unification between t1 and t2*)
%     | Var (n:vname)              (*a variable*)
%     | Lam (x: vn) (b:tm)         (*lam abstraction*)
%     ...                          (*this is extended in section XX*)
%   Inductive clause := Clause (p:pn) (i:tm) (i:tm) (A:list tm).
%   Notation "P I O :- Bo" := (Clause P I O Bo).
%   Notation "t1 = t2" := (Unify t1 t2).
%   Definition sbst := T.
%   Definition prog := pn -> list clause.
%   Inductive goal := Goal (P:prog) (a:tm) (A:list alt) -> goal.
%   with alt := sbst * list goal.
% \end{coqcode}

% The evaluation of a program is done through the \run\ predicate. It is a
% function with the following signature:
% %
% $$run : goal \to \vecL{alt} \to subst \to (\vecL{alt} * subst)$$
%
% \begin{minted}[autogobble]{coq}
%   Inductive run : goal -> list alt -> subst -> list alt -> subst := ...
% \end{minted}

% \coqIn{run} should be understood as a relation between a goal \g, a list of
% alternatives \alt (which are disjunctive with the current goal), and an initial
% substitution \subst. These three components produce a new list of alternatives
% \alt['] and an updated substitution \subst['].
% The notation \runCmd{\mathcal{G}}{\mathcal{A}}{\subst}{\mathcal{A}'}{\subst'}
% represents calls to \run. 



% The idea is that the intepreter stops
% when it finds the first solution, which is a valid substitutions for the query
% wrt the program. In order to find all the solution of query from the current
% program, that is all the valid substutions for the query, it is sufficient to
% iterate over all the alternatives \alt['].

% In the following we use the following notation for call to the run predicate:
% %
% $$ \runCmd{\mathcal{G}}{\mathcal{A}}{\subst}{\mathcal{A}'}{\subst'} $$
% In this first section we start with a first-order version of a logic language.


% of the objects we manipulate. Instead of using
% continuations to store the state of a choice point, we represent this
% information as lists containing the alternatives, along with the substitution
% that existed at the moment the choice point was created.

% As an example, let's consider the following program called \prog:

% \begin{elpicode}
%   p1 X Y :- p2 X Y.           % r1
%   p1 3 3.                     % r2
%   p2 X Y :- p3 X Y, !, Y = 1. % r3
%   p2 1 1.                     % r4
%   p3 1 2.                     % r5
% \end{elpicode}

% {
% \def\goalG{\goalCmd{\prog}{\callCmd{\texttt{p1}}{3\ Z}}{\EmptyList}}

% Let $\mathcal{G} :=\ \goalG$ be a goal, the execution of
% ``\runCmd{\mathcal{G}}{\EmptyList}{\EmptySubst}{?A}{?\subst}'' will apply \ref{rule:call}
% producing a new goal for the rule $r1$ and an alternative list containing $r2$,
% we leave out details concerning unifications of head terms. The execution of
% $r1$ will try to solve $r3$. This will add three goals \elpiIn{r 1 Y, !, Y = 1}
% with cut-alternatives equal to $r2$ whereas the new alternatives will be $r4 @
%   r2$. The execution of $r3$ will solve \elpiIn{p3 1 Y} with substitution $s :=
%   \{X \gets 1; Y \gets 2\}$. The \elpiIn{!} will cut away the alternative $r4 @
%   r2$ and will keep the cut-alternative $r2$. The failing premise \elpiIn{Y = 1}
% will cause a backtracking thanks to \ref{rule:backtrack} and try to apply $r2$ from the
% empty substitution. This last unification succeed with final substution $?\subst :=
%   \{X \gets 3; Y \gets 3\}$ and final list of alternatives $?A := \EmptyList$.

% % \begin{myRule}{1}
% %   \AxiomC{}
% %   \RightLabelM{\ruleCall}
% %   \UnaryInfC{\runCmd{\mathcal{G}}{\EmptyList}{\EmptySubst}{?A}{?S}}
% % \end{myRule}

% }

% \subsection{The cut operator}
% \label{sec:cut}

% Logic programs are known for their non-deterministic behavior: there can be
% multiple distinct ways to derive a query from a knowledge base, and logic
% programs aim to find all of these solutions. While non-determinism is a key
% feature, it is sometimes important to allow the user to control if and when
% alternatives should be rejected. The cut operator is designed to address this
% problem.

% Since we use the elpi dialect of \lamprolog, it is important to clarify how
% the cut behaves in our development. It is that case that each prolog-ish language
% has its own cut implementation, for example in the official page of
% swi-prolog (\href{www.swi-prolog.org}{www.swi-prolog.org}) we see that
% there are two different cut implementations: the \textit{soft cut} noted with
% ``\texttt{C *-> T ; E}'' runs E if C has no solution otherwise the result is the
% same as running the conjunction of C and T. The \textit{hard cut} noted with the
% ``\texttt{!}'', "discards all choice points created since entering the predicate
% in which the cut appears".

% The \mercury system \cite{1996Somogy} is meant to be a pure programming language
% extended with types, modes, determinism and module systems. In order to satisfy
% its purity, \Mercury only accepts a version of the soft cut.

% It is also interesting to see that further representations of cut may exists,
% such as the firm cut explained in \cite{2003Andrews} where they provide a
% restricted version of the hard cut which has some concistency properties.

% \Elpi implementation uses the hard cut definition. Even though the hard cut have
% no equivalent representation from pure logic, we are convinced that, from a
% programming perspective, it is quite practical. It allows us to eliminate not
% only alternative implementations of a predicate but also to prevent backtracking
% of all choice points born from the begin of the clause-body to the current cut
% position.

% For example, in \elpi, the \textit{if-then-else} construct, which is not a\todo{wrong if pred are bin + no prop as arg}
% primitive of the language, can be impelented as follows:
% \begin{elpicode}
%   if C T E :- C, !, T. % if1
%   if _ _ E :- E.       % if2
% \end{elpicode}
% Due to the hard cut, the query ``\elpiIn{if C T E}'' executes the \texttt{if1} rule.
% If \elpiIn{C} has $n > 0$ solutions then only the first is kept, the other being
% cut away as the rule \texttt{if2}. The call then succeeds only if \elpiIn{T} has
% a solution. The rule \texttt{if2} will be run only if \elpiIn{C} has no
% solution.

% For example, in the following database:

% \begin{elpicode}
%   p 1.
%   p 2.
%   q 2.
% \end{elpicode}
% The query ``\elpiIn{if (p X) (q X) (0 = 1)}'' fails. Note that the usage of the
% soft-cut would make the same query to have a solution, since the backtracking in
% the condition would be authorized.
% \todo{cut di teyjus}


\section{The meaning of a predicate signature}

It is one of our objectives that the determinacy checker accepts
predicates that are \textit{wrongly} called so that we can, incrementally,
make the predicate signatures more precise.


The \mySub\ relation subsumes signature equality and
extends the intuitive inclusion of mathematical functions
into relations to the recursive nature of predicate signatures, and rejects the
converse inclusion. 
In the spirit of subtyping the input arguments of predicate signature
are compared in a contravariant way. 

\begin{equation}
  f_1 \mySub f_2 =\ 
  \begin{cases}
   \ \bot                                                          & \text{if } f_1 = \dtype{\relI}{\_}{\_}\ \land\ f_2 = \dtype{\detI}{\_}{\_} \\
    % (\forall i, a'_i \mySub a_i) \land (\forall j, b_j \mySub b'_j)      & \text{else if } f_1 = \detI\ a\ b \land f_2 = \detI\ a'\ b' \\
    % (\forall i, a'_i \mySub c_i) \land (\forall j, b_j \mySub c_{|a|+j}) & \text{else if } f_1 = \detI\ a\ b \land f_2 = \relI\ c      \\
    % \forall i, a_i \mySub b_i                                            & \text{else if } f_1 = \relI\ a\ \land f_2 = \relI\ b        \\
    % (\forall i, c_i \mySub a_i) \land (\forall j, b_j \mySub d_j) & \text{else if } f_1 = [\detI\mid \relI]\ a\ b \land f_2 = [\detI\mid \relI]\ c\ d \\
   \ (\forall i, c_i \mySub a_i)\ \land\ (\forall j, b_j \mySub d_j) &
    \!\!\!
    \begin{array}{l} \text{if } f_1 = \dtype{\func}{a}{b}\ \land\ f_2 = \dtype{\func}{c}{d} \ \lor    \\
     \ \ \  f_1 = \dtype{\detI}{a}{b}\ \land\ f_2 = \dtype{\relI}{c}{d} 
    \end{array}    \\
    % (\forall i, c_i \mySub a_i) \land (\forall j, b_j \mySub d_j) & \text{if } f_1 = \relI\ a\ b \land f_2 = \relI\ c\ d \\
   \ \top                                                          & \text{if } f_1 = \expI\ \land\ f_2 = \expI
  \end{cases}
  \label{eq:sub-def}
\end{equation}

\newcommand{\sem}[1]{\ensuremath{[\![#1]\!]}}
The intuition is that the meaning of a 
predicate signature is given by the following formula where
\isdet\ is an uninterpreted predicate (the definition we give in section~\ref{sec:thm}
does not matter).
\begin{definition}[Meaning of a signature ($\sem{\func}\pred$ for $\ctx\ \pred = \func$)]\label{def:sem}
$$
\begin{array}{rl}
\sem{\dtype{\func_1}{\func_2}{\func_3}}\ \pred{}\!\!& =\ \forall \vecL{t_i}\ \vecL{t_o}, \bigwedge \vecL{\sem{\func_2}\ t_i} \Rightarrow \sem{\func_1}\ (\pred\ t_i\ t_o) \land \bigwedge \vecL{\sem{\func_3}\ t_o}\\
% \sem{\dtype{\func_1}{\func_2}{\func_3}}\ \pred & = \forall t\ u, \overbrace{\sem{\func_2}\ t}^{\mathrm{pre-conditions}} \Rightarrow \overbrace{\sem{\func_1}\ (\pred\ t\ u) \land \sem{\func_3}\ u}{\mathrm{post-conditions}}\\
\sem{\detI}\ \pred{}\!\!& =\ \isdet\ \pred \\
\sem{\relI}\ \pred{}\!\!& =\ \top \\
\sem{\expI}\ \pred{}\!\!& =\ \top 
\end{array}
$$
\end{definition}
The first equation states tht if the preconditions of a signature are satisfied
then the call to a predicate of that signature is
functional and the post conditions on the outputs hold.
The relation \mySub\ is such that if $f_1 \mySub f_2$ then $\sem{f_1} \Rightarrow \sem{f_2}$,
hence the contravariance:
the post conditions of $f_1$ can be stronger than the ones of $f_2$ while
the pre conditions of $f_1$ must be wekaer than the ones of $f_2$.

\section{Signature checking}

The checking process performs a form of dataflow analysis:
It stores the predicate
signature of each variable in \ctx\ and uses it to verify whether the variable
satisfies the precondition of a predicate call or not.
Additionally, it may update the
variable's signature to match the postcondition of
the predicate call.

Updates use the dual operators \(\minT\) and \(\maxT\).  
In their mutual definition below, we encode them thourgh the function \( f \),
parameterized by the determinacy \( d \) (either \(\detI\) or \(\relI\)).
Specifically, \( f_\detI \) represents \(\minT\), while \( f_\relI \)
corresponds to \(\maxT\). We use \( \overline{d} \) to denote the negation of \(
d \), which inverts its determinacy value. The definitions of these functions
are as follows:  

%
\begin{equation}
  f\ s_1\ s_2 =\ 
  \begin{cases}
    %\ \dtype{\detI}{(\vecL{g\ a\ c})}{(\vecL{f\ b\ d})} & \text{if } s_k = \dtype{\relI}{a}{b}\ \land\ s_{3-k} = \dtype{\detI}{c}{d} \\
    \ \dtype{d}{(\map\ f_{\overline{d}}\ \vecL{ac})}{(\map\ f_d\ \vecL{bd})} & \text{if } s_k = \dtype{\relI}{\vec a}{\vec b}\ \land\ s_{3-k} = \dtype{\detI}{\vec c}{\vec d} \\
    %\ \dtype{D}{(\vecL{g\ a\ c})}{(\vecL{f\ b\ d})}     & \text{if } s_1 = \dtype{D}{a}{b}\ \land\ s_2 = \dtype{D}{c}{d}             \\
    \ \dtype{\func}{(\map\ f_{\overline{d}}\ \vecL{ac})}{(\map\ f_d\ \vecL{bd})}     & \text{if } s_1 = \dtype{\func}{\vec a}{\vec b}\ \land\ s_2 = \dtype{\func}{\vec c}{\vec d}             \\
    % \relI\ (\vecL{g}\ a\ c)\ (\vecL{f}\ b\ d) & \text{if } s_1 = \relI\ a\ b \land s_2 = \relI\ c\ d \\
    \ \expI                                                     & \text{if } s_1 = \expI\ \land\ s_2 = \expI
  \end{cases}
  \nonumber
  \label{eq:min}
\end{equation}
\vspace{0.3em}

We assume each clause has been type checked by an algorithm that
traverses terms including variables in \X\ that
represent predicates, e.g. the higher-order parameter of \elpiIn{map}.
Since we want to keep type checking separated from determinacy checking,
the former identifies \relI\ with \detI\ and in particular assigns
to these variables the signature obtained by weakening their
declared signature:
It is up to the determinacy analsysis to assume or prove stronger signatures.
The \maximize\ function is defined mutually with its
dual \minimize. Similarly to the \minT and \maxT, we encode them with the
function $f$ parametrized by a determinacy $d$:
%
% $$
%   \minT\ f_1\ f_2 = \begin{cases}
%     \maxT\ a'\ a \myTo \minT\ b\ b' & \text{if } f_1 = a \myTo b \land f_2 = a' \myTo b' \\
%     f_1                             & \text{else if } f_1 \mySub f_2                     \\
%     f_2                             & \text{otherwise}
%   \end{cases}
%   \label{eq:maximize}
% \end{equation}
% \begin{equation}
%   \minimize\ f =
%   \begin{cases}
%     \dtype{\detI}{(\vecL{\maximize}\ a)}{(\vecL{\minimize}\ b)} & \text{if } f = \dtype{\_}{a}{b} \\
%     \expI                                                       & \text{if } f = \expI
%   \end{cases}
% $$
%
% The main usage of the \minT\ and \maxT\ comes with the fact that unification is
% possibile between variable with different determinacy annotation. For example,
% let \elpiIn{get-det} and \elpiIn{get-rel} be two relations returning
% respectively a function and a relation (both with zero arguments). In a query like \elpiIn{get-det X,
%   det-rel Y, X = Y} we have three goals to solve, the first allows to label
% \elpiIn{X} with the tag \detI, the second allows to label \elpiIn{Y} with
% the tag \relI. The third goal is interesting since we are unifying two\todo{not true if output are flex}
% predicates with different determinacy labels. Now, the only way that this unificaton
% succeeds at runtime is that \elpiIn{Y} is a function.
% Said in another way, a unification may change the labels of the two terms being
% unified. In the previous example, \detlab\ \elpiIn{X} $=$ \detlab\ \elpiIn{Y} $=$
% \minT\ $($\detlab\ \elpiIn{X}$)$ $($\detlab\ \elpiIn{Y}$)$.
%
% finally when a predcate is miscalled we need a way to deal with its post conditions
%
%
\begin{equation}
  f_d\ s =\ 
  \begin{cases}
    \ \dtype{d}{(\map\ f_{\overline{d}}\ a)}{(\map\ f_d\ b)} & \text{if } s = \dtype{\_}{a}{b} \\
    \ \expI                                                       & \text{if } s = \expI
  \end{cases}
  \label{eq:maximize}
\end{equation}
% \begin{equation}
%   \minimize\ f_1 =
%   \begin{cases}
%     \dtype{\detI}{(\vecL{\maximize}\ a)}{(\vecL{\minimize}\ b)} & \text{if } f = \dtype{\func}{a}{b} \\
%     \expI                                                       & \text{if } f = \expI
%   \end{cases}
%   \label{eq:minimize}
% \end{equation}
%
% The two fonctions respectively return the maximal and minimal dtype of a given
% dtype. This is used \dots
%
% $$---$$
%
% and therefore
% The base cases of \mySub\ are quite intuitive: a relation is not a function. The
% interesting case is the \coqIn{arr} constructor. Here, we are comparing two
% propositions and therefore some attention should be paid for its inputs and
% outputs: the inputs of the left memeber should be at least as restrictive as the
% right one and the outputs of the left memeber are at most as restrictive as the
% right one.
%
% As an example consider two terms $t_1 \coloneq \texttt{idF}$ and $t_2 \coloneq
%   \texttt{once}$. 
%
% What may seem weird at first sight is that in the example above, we call
% \texttt{idF} with a relation in input position. This creates a mismatch
% between the dtype of the signature and the dtype of the argument.\todo{dtype := determinacy type}
% However, we want our system to be as expressive as possible. This means that
% we allows the user the wrongly call a predicate. In that particular case the dtype of the
% outputs will not be guaranteed to have the expected determinacy: they will be labeled
% with the less restrictive determinacy for its type. To continue our example,
% the query \elpiIn{idF (divisor 3 X) Y} will label \elpiIn{Y} with the \relI\
% tag, since the determinacy of \elpiIn{divisor 3 X} is \relI\ and $\relI \not\mySub \detI$.
%
% Thanks to the \mySub\ relation we can define a min (resp. max) relation between
% two determinacy relations.
%
%
% \begin{coqcode}
%   Fixpoint min a b = match a, b with
%     | a ~\myTo~ b, a' ~\myTo~ b' => max a' a && min b b'
%     | a, b -> if a ~\mySub~ b then a else b
%   end with max a b = match a, b with
%     | a ~\myTo~ b, a' ~\myTo~ b' => min a' a && max b b'
%     | a, b -> if a ~\mySub~ b then b else a
%   end.
% \end{coqcode}
%
% These two functions become particularly useful
% when trying to determine the determinacy of unification variables. Let's
% take the following example.
%
% \begin{elpicode}
%   pred give_rel i:int, o:pred.
%   func give_fun i:int, o:func.
%   give_rel _ (divisors 5 X).
%   give_rel _ (succ 1 X).
%   give_fun _ (succ 1 X).
% \end{elpicode}
%
% In the snippet above, both \txt{give\_rel} and \txt{give\_fun} are functions
% that take a dummy integer as input (we conventionally work with binary
% predicates) and return, respectively, a relation and a function.
%
% Now, consider the following query: \elpiIn{give_fun 1 Y, give_rel 1 Z, Z = Y}.
% The first goal tells us that \elpiIn{Z} is \relI, while the second tells us that
% \elpiIn{Y} is \detI. After unifying the two variables, what determinacy label
% should they have? Unification succeeds if the two terms are unifiable, meaning
% that, in the end, \elpiIn{Z} and \elpiIn{Y} will be the same term and will
% therefore share the same \detlab. Their label will be updated and set to the
% minimum determinacy between them, which is \coqIn{min Rel Fun} $=$ \detI.
%
% If we take back the query \elpiIn{give_fun 1 Y, give_rel 1 Z, Z = Y}, we may
% want to statically determine if it create choice points. The determinacy of each
% goal is \detI, \relI, \detI, the determinay of the query is the maximum of them,
% that is \relI.
%
% With this, we now have all the ingredients needed to propose the full static
% determinacy checker for the higher-order setting. This checker is implemented in
% \detCheck, which ensures that a deterministic predicate has at most one solution
% per call. Furthermore, it is extended to certify that each output of a clause
% has the expected determinacy, which is crucial since outputs can be used as new
% queries in the program.
%
\inferFig
%
\assumeFig
%
\paragraph{The static checker} We can now exibit the static checker.
The first rules we present allow to \infer\ (\cref{fig:det-inference}) or
\assume\ (\cref{fig:det-assume}) the signature of a predicate and finally to
\checkk\ (\cref{fig:static-check}) if a rule matches a declared signature.
%
% $$\infer : \ctx \to (\A + \T) \to \func \times \B$$
% $$\dapp : \ctx \to \vecL{\func} \to \vecL{\func} \to \vecL{tm} \to \B \to \func \times \func \times \B$$

% TODO: rule for lambda assumes lambdas are ordered according to
% the convetion of input first and that their mode is somehow given.
% We have an heuristic in the type checker for that\todo{Davide?}.

% TODO: rule for app handles partial application like
% the \elpiIn{commit likes V} in the intro and check if
% sigs are respected.

The \infer\ procedure 
%is responsible for inferring the signature of a term. It
returns a pair containing the signature of the given
atom and a boolean value indicating whether
it contains miscalled predicates.
%  Since predicate
% calls are part of the term syntax, they can also be miscalled.  

The rules for \cut\ (\ref{rule:infer-cut}) and functors (\ref{rule:inferK})
are straightforward. Rule \ref{rule:infer-piimpl} introduces a new mapping in
the context, associating the binder with its signature and performing a
recursive call to \infer. $\lambda$-abstractions are handled by
\ref{rule:infer-lam}. We distinguish between binders corresponding to inputs and
those corresponding to outputs. Binders in input positions are added to \ctx\
and mapped to their corresponding signature.  

The rule \ref{rule:infer-app}, combined with the \dapp\ procedure,
handles call when the head is a variable or a predicate
and  accounts for partial applications.  

For example given \elpiIn{func once (pred i:A, o:B), A -> B} as in
the introduction, the partial application \elpiIn{(once likes)}
has the inferred signature
\elpiIn{~\PYG{k+kd}{func}\PYG{+w}{ }\PYG{k+kt}{guest}\PYG{+w}{ }\PYG{k+kMode}{\PYGZhy{}\PYGZgt{}}\PYG{+w}{ }\PYG{k+kt}{dish}~}.
%
% $t =\ $\elpiIn{map likes!}. Its inferred
% signature is obtained by calling \ref{rule:infer-app}. The signature of
% \elpiIn{map} expects a binary function as the first input argument. Thanks to
% \ref{rule:inferAppInputOK}, we can infer the signature of \elpiIn{likes!}, which
% falls into the category of partial application (in fact, this term is not
% applied at all). Ultimately, the signature of \elpiIn{likes!} is a subtype of
% the expected signature for the first argument of \elpiIn{map}. The result of the
% call to \infer\ on $t$ is the pair $(\dtype{\detI}{\expI}{\expI},\top)$: the
% Boolean value indicates that the predicate call is correct, and the first
% position of the pair expresses the signature of the partially applied term.  
% 
% $$\assume : \func \to (\A + \T) \to \ctx \to \ctx$$  

The \assume\ procedure strenghtens the signature
of variables occurring in the given term by
updates a context \ctx\ according to \func. The rule
governing this change is \ref{rule:assume-hd}, which states that whenever a
non-applied variable is encountered, its signature is set to the \minT\ between
its current signature in \ctx\ and \func.
The intuition is given by definition~ref{def:sem}: when the
preconditions are satisfied we know the post conditions hold. 
% The use of \minT\ is justified by the
% fact that \assume\ represents runtime unification.
% For unification to succeed,
% the variable's signature must be the strongest between the one in the context and
% the expected one.  \todo{enrico help}

The rules \ref{rule:assume-cut} and \ref{rule:assume-expr} are intuitive, as
they do not require any modifications to the context. In rules
\ref{rule:assume-lam} and \ref{rule:assume-piimpl}, we extend the context by
mapping the binder to its signature. % and making a recursive call to \assume.\todo{rule for lambda mh}
Finally, \ref{rule:assume-app} performs a fold over the input arguments of the
predicate being called, under the assumption that the determinacy of the
predicate is smaller than the received one.  
% 
% $$\checkk : \A \to (\ctx \times \F) \to (\ctx \times \F)$$  

\checkTermFig  

The \checkk\ procedure (in \cref{fig:static-check})
tries to prove, in the sense of~\ref{def:sem} the determinacy
of an atom by checkig if the pre conditions and deducing new
knowledge from the post conditions.

% combines both \infer\ and \assume
% to determine if an atom is fu

% . Its purpose is to
% check the execution of a query under a given context and determinacy. The result
% is a new context and an updated determinacy.  

% For example, as shown in \ref{rule:check-cut}, if the term to be executed is a
% \cut, then the context remains unchanged, and the determinacy becomes \detI. If
% a predicate $p$ is called (\ref{rule:check-callOK} and \ref{rule:check-callKO}),
% we first verify whether the call is correct -- i.e., whether the inferred
% signature of the inputs is at least as strong as the expected one. If the call
% is correct, the terms in the output position are assigned the signature of $p$'s
% outputs. If $p$ is miscalled, then the behavior of the call is
% non-deterministic, and we output \relI.  

% If the call is correct, the behavior of the call is determined by the maximum of
% the predicate's determinacy and the input determinacy. For example, if we are in
% a deterministic context and correctly call a relation, then the resulting
% behavior of the query is the maximum of the two, i.e., relational.  

% The ~ref{rule:check-clause} rule concerns the validation of
% an entire clause: It begins by assuming that the input terms
% have the determinacy specified in the signature and
% it analyzes all premises in order, accumulating in \ctx
% all post conditions \checkk\ was able to prove.
% Finally it checks that the determinacy of the body
% matches the declared one, as well as that all output
% signatures are consequences of the body.
% \todo{sucks}

%  Then, we analyze the premises
% of the clause, assuming its behavior is deterministic. Finally, we infer the
% type of the output terms, ensuring that all have their boolean value set to
% $\top$ and that their signature is \mySub compared to the expected determinacy.
% Additionally, we verify that the determinacy of the body (\func['] in the rule)
% is \mySub relative to the expected determinacy (\func\ in the rule).

% xxx

% Rule \ref{rule:check-piimpl} behaves similarly to \ref{rule:assume-piimpl}. It
% also makes some check on the hypothesis $H$. But we delay it to the next
% section.


% If the term is a unification the determinacy \func\ is transmitted from the
% input to the output. What may change is the context, since, during unification
% some variables may be assigned. In \ref{rule:check-unif} we are in the case
% where it is possible to successfully deduce the dtype of the left and the right
% terms of the equality operator. At runitime, unification succeeds if the two
% terms are unifiable. This means that in the end the two terms will share the
% same determinacy. The determinacy the will have is the minimum between the
% determinacy of the two terms. Imagine that we are unifying a function with a
% relation, then the more restrictive condition for both terms should be taken
% into account. Therefore, the \assume\ procedure is called on both terms so that
% their determinacy is set to the minimum. If on the contrary, it is not possible
% to \infer\ a valid dtype for at least one of the two terms (rule
% \ref{rule:check-unifFail}), we cannot deduce much more on the variables in the
% two terms, therefore, we do nothing on the context \ctx, which is returned as it
% was in input.

% The \checkk\ rule for application (see \ref{rule:check-callOK}) starts by
% deducing the type of the input term. If its dtype \func[_i'] can be derived
% (i.e. the booleans is $\top$) and if \func[_i'] is less or equal than the
% expected one in the signature, then the output can be assumed to be the one
% inscribed the signature. If these two conditions are not validated, then the
% rule context is not modified. Concerning determinacy, if the call is valid, then
% the determinacy is the \maxT\ between the one received in entry and the
% determinacy of the head of the application. If, otherwise, the call is wrong,
% then the output determinacy is set to relational.

% $$\checkk : \A \times (\ctx \times \F) \to (\ctx \times \F)$$

% \textbf{TODO: make an example}

% \subsection{Determinacy checking for clauses}

% \staticCheckFig
% \todo{fai con fold}

% We provide an example of this checker in the following database.

% \begin{elpicode}
%   give-fun X (succ X Y).
%   give-fun X (~\texttt{pred}~ X Y).
%   good :- give-fun 3 Res, !, Res.
% \end{elpicode}

% The signatures of the predicates in the snippet above are the following:
% %
% \begin{align*}
%   \texttt{succ}     & : \dtype{\detI}{[\expI]}{[\expI]} & \texttt{pred} & : \dtype{\detI}{[\expI]}{[\expI]} \\
%   \texttt{give-fun} & : \dtype{\relI}{[\expI]}{[\detI]} & \texttt{good} & : \detI
% \end{align*}

% The static checker for determinacy starts by analysing the first rule for
% \texttt{give-fun}. The input is an expression, the body is empty, so in the end
% we only have to check that the output has the output and the body have the right
% determinacy. Now, the determinacy of the body is $\detI \mySub \relI$, the
% determinacy (deduced) for the ouput term is $\detI \mySub \detI$. Therefore the
% rule respect the contract of the signature: the rule is valid. A very similar
% reasonment can be do on the second rule for \texttt{give-fun}. It is now more
% interesting to analyze the clause for \texttt{good}. We have no input, thereofre
% we keep analyze the body. The context on which \checkL\ is launched is $\ctx =
%   \{\text{\elpiIn{Res}}\mapsto \relI\}$ since in the wrost case the determinacy of
% a term with type \texttt{prop} is \relI. The iteration on the terms in the body
% starts by analysing ``\elpiIn{give-fun 3 Res}'' which makes the determinacy of
% the term relational and sets updates the context to
% $\{\text{\elpiIn{Res}}\mapsto \detI\}$, since it is a correct call and its
% output is \assume d to have the functionality of the output in the dype of
% \elpiIn{give-fun}. The \cut\ operator, makes \func\ to become \detI: intuitively
% the \cut discards all choice points, that's way \ref{rule:check-cut} puts the
% determinacy to \detI. Finally we execute a term which, by
% \ref{rule:check-callOK} is deterministic. We conclude the execution of \checkL\
% with $\func = \detI$ and $\ctx = \{\text{\elpiIn{Res}}\mapsto \detI\}$. This is
% exactly what the signature of \texttt{good} expects and therefore the rule
% passes the check.

% $$-----$$

% For instance, we would like that the call \elpiIn{idF divisors X} is valid for
% determinacy, but, since the input has not the expected determinacy, then the
% output will not be a deterministic predicate. Indeed, we propose the following
% ordering relation between deterministic and relational predicates.

% TODO: presentare l'algoritmo di static check of determinacy with varaible flow.

% Dare un esempio in cui l'input ground non è garantito e in cui il checkin viene
% invalidato

\newcommand{\pgc}{\texttt{pgCheck}}
\newcommand{\pgcC}{\texttt{pgCheckC}}
\newcommand{\pgcP}{\texttt{pgCheckP}}

% \subsection{Mode checking: propositional-ground mode}
% \label{sec:mc}

% \begin{leftbar}{red}

% The last but not least important point to treat is the mode checker. Similarly
% to the determinacy checker in the literature, the determinacy checker extended
% with higher-order propositional variable need a guarantee on the terms that are
% manipulated during the resolution of a goal. If we take back the predicate
% \elpiIn{idF}, defined at the beginning of \cref{sec:vars} and the predicate
% \elpiIn{give-fun}, defined and the end of the previous paragraph, we can create
% a list of premises contradicting the hypothesis the static checker performs on
% terms.

% The list of goal we may be interested in is \elpiIn{idF X X, X = give-fun 3 Y}.
% This goal naturally succeeds at runtime, but what breaks the checker is the
% hypothesis made on the variable \elpiIn{X}. If we perform a \checkk\ on the
% first atom we realize that 

% As we have sketched in \cref{sec:basic-elpi}, we don't want to perform 

% \begin{definition}[Proposiotions ground check (\pgc)] 
%   ground check: \pgc.
% \end{definition}

% \begin{definition}[Clause prop ground check (\pgcC)]
%   Given a clause, we use the same algorithm as the mode ground checker from
%   the literature, but, instead of using the ground condition on terms, we
%   use the definition of input ground check (\pgc)
% \end{definition}

% \begin{definition}[Program ground check (\pgcP)] 
%   Given a program \prog,
%   $$\forall p\ c, c \in \prog p \to \pgcC\ c$$
% \end{definition}
% \end{leftbar}


% \subsection{The final theorem}

% The final theorem we want to prove is the following.

% \begin{theorem}[Determinacy checking property]
%   Given a context \ctx, a program \prog and a deterministic predicate \pred
%   % \begin{align*}
%   %   &\pgcP\ \prog \to \detCheck\ \prog\ \to \\ 
%   %     &\quad \forall i\ o, \pgc\ i\ \to \isdet\ \prog\ \pred\ i\ o \\
%   %       &\qquad\land \pgc\ o \land \infer\ o = \get{o}p
%   % \end{align*}
%   \begin{align*}
%     \hyperref[def:det-check-ho]{\detCheckHO}\ \prog\ \to
%     (\forall p\ c, c \in \prog\ p \to \hyperref[fig:static-check]{\checkc}\ \ctx\ c) \to \forall \vec{t}, \hyperref[def:is-det]{\isdet}\ \prog\ (\pred\ \vec{t}) %\land \infer\ \vec{t}_o = \get{o}p
%   \end{align*}
%   \vspace{-20pt}
%   \label{th:det3}
% \end{theorem}

% % $$\isdet := \isDetCmd{i}$$

% \begin{proof}
%   We reason on the derivations in \cref{fig:basic-interp} extended with
%   \cref{fig:interp-match,fig:interp-piimpl}. The interesting case is the rule
%   \ref{rule:call} since we have an application in our call the \run.
%   By
% \end{proof}

% \subsection{Mix data with ty}

% \begin{align}
%   data & ::= \texttt{c}\ ty^\ast \mid data \to data \mid \predVar \label{eq:data11}
% \end{align}

% \subsection{Determinacy polymorphism}

% Il problema è ad esmepio la funzione identità

% \begin{elpicode}
%   pred id i:A, o:A.
% \end{elpicode}

% Se id viene chiamato con una deterministic-predicate, allora l'output sarà
% anch'esso deterministic. Il problema è che non è sempre vero che la determinacy
% è passata tra due oggetti che hanno lo stesso tipo.

% Per esempio:

% \begin{elpicode}
%   pred wrong_id i:A, o:A.
%   wrong_id fun rel.
% \end{elpicode}

% Per l'identià si potrebbe definire versioni specializzate:

% \begin{elpicode}
%   pred id1 i:(func A -> A) (func A -> A).
%   id1 X X.
% \end{elpicode}

% Ma ovviamente questo è tedioso, ci vorrebbe la possibilità di avere un
% polimorphismo di determinacy.
% Qualcosa tipo:

% \begin{elpicode}
%   pred id2 i:X~$^\texttt{Y}$~, o:X~$^\texttt{Y}$~.
% \end{elpicode}

% Dove $^\texttt{Y}$ trasmette l'informazione che i due argomenti di id2 hanno
% la stessa funzionalità.

% \paragraph{\textbf{Digression: inference v.s. checking}}

% Several works \cite{king2005, king2006, 2011king}
% study the inference of determinacy in \prolog system with \cut.
% They explain that determinacy inference subsume determinacy checking.
% We think that an inference algorithm does not fit
% well with our language.

% % The first motivation is that since an \elpi
% % %is an interpreted dialect of \lamprolog,
% % program can change dynamically due to the \impl operator.
% % This means that rules can be added \textit{à la volée} making impossible to infer
% % determinacy as the user expects.
% % It's
% % up to the user to choose what should be the behavior of a predicate wrt
% % determinacy. The checker is meant to assist the user by rejecting, a predicate,
% % that does not respect the declared determinacy.

% As explained in some of the previously cited papers, determinacy (and so
% functionality) checking is an undecidable problem. The checking property is
% sound: if a predicate passes the analysis then it is for sure deterministic.
% However, it is not complete: there could exists false negatives. We are
% convinced, that completeness, is not a big deal, since, as also claimed in ...
% every deterministic predicate not passing the determinacy check,
% can be rewritten in a equivalent way
% so that it is no more classified as non-deterministic.



\section{Determinacy analysis}\label{sec:thm}

% \begin{definition}[Functionality (\isfunc)]\label{def:is-func}
%   Given a program \prog and a term $t$,
%   \isdet\ is defined as follows:
%   $$\forall a\ \subst, \isFuncCmd{t}$$
% \end{definition}

We are interested in identifying predicates that leave no choice points
called \emph{semidet} in \cite{1996henderson}, or in the wording of \cite{nakamura1986}
are \emph{operationally deterministic}.

\begin{definition}[Operational determinacy (\isdet)]\label{def:is-det}
  Given a program \prog and a predicate $\pred$,
  \isdet\ is defined as follows:
  $$\forall a\ \vec{t}\ \subst, \isDetCmd{\pred\ \vec{t}}$$
\end{definition}

% The essence of the definition is that when interpreting any call to a given
% predicate \pred within a program \prog, starting from an empty substitution and
% an empty list of alternatives, no new choice points should be generated. This means
% that the returned alternatives must be an empty list.

\noindent
We now give some \emph{sufficient} conditions for a predicate to be
operationally deterministic.

\begin{definition}[Unifiable (\unifiable)]
We say that $\unifiable\ t_1\ t_2$ holds iff
  $\exists\ \subst\ s.t.\ \unifyCmd{t_1}{t_2}{\EmptySubst}{\subst}$.
% We say that $\nUnify\ t_1\ t_2$ iff
%   $\forall \subst \ \unifyCmd{t_1}{t_2}{\subst}{\bot}$ (equivalently, since
%   \unify is complete, $\neg \exists \subst\ s.t. \subst t_1 = \subst t_2$).
\end{definition}

% Thanks to this first definition, we can state a key property of a predicate to
% be deterministic. We need that the head of its clauses in the program are
% mutually exclusive (or equivalently non-overlapping), i.e. at most one clause
% can be used on a given query.


% Before explaining the interaction between these two, we need to take
% some time to talk about mutual clauses exclusiveness.
% %
% % \paragraph{Mutual exclusive clauses}
% From \cite{1989Warren}, we know that at most one clause can be applied
% successfully for any deterministic predicate.

% In order to satisfy this (necessary by not sufficient) condition, we need 
% that all the clauses of a deterministic predicate $p$ are mutually exclusive.
% Mutual exclusiveness can be stated as follows:

% \begin{definition}[Non-overlapping clauses]
%   For any pair of clauses of the same predicate ``\clauseCmd{p}{\vec{x}}{b1}''
%   and ``\clauseCmd{p}{\vec{y}}{b2}'', we say that they are non-overlapping if
%   there exists an input position $i$ such that $\forall \sigma, \sigma\ x_i \neq
%   \sigma\ y_i$
%   \label{def:mut-excl}
% \end{definition}

% \Cref{th:det} ensures that at most one clause can be applied to any given
% predicate, provided that the input terms are ground. However, since we no longer
% perform mode-checking and our inputs are not necessarily ground, the theorem is
% no longer valid.
%
% Thanks to mode checking, any call is validated only if its terms in input
% positions are ground. Groundness % (modulo \cut, see \cref{def:mut-excl+cut})
% guarantees that at most one clause can be executed successfully for a given
% call.
%
% We also emphasize that mode checking ensures that output terms become ground.
% This is a fundamental property; otherwise, outputs would be meaningless—if an
% output does not become ground, it cannot serve as the input for another
% predicate call.
%
% In our setting, we slightly extend this definition so that the \elpi's \uvar
% keyword is taken into account.
%
% \begin{definition}[Mutual exclusiveness with \uvar]
%   A clause with an input term marked with the \elpiIn{uvar} keyword,
%   does not overlap with any other rigid-head term.
%   \label{def:mut-excl-uvar}
% \end{definition}
%
% This means that a term marked with the \uvar keyword in the head of a clause
% overlaps only with unification variables or with another term marked with
% \uvar.
%
% In our first-order \elpi, we do not perform any static mode analysis. Instead,
% we rely on the \match operation, which is dynamically applied to input arguments at runtime.

% \begin{lemma}[Substitution union] Given two substitutions $\subst_1$ and $\subst_2$
%   such that $\dom\ \subst_1 \cap \dom\ \subst_2  = \varnothing$, then
%   $\subst_1\ t_1 = \subst_2\ t_2$ implies that $(\subst_1 \cup \subst_2)\ t_1 = (\subst_1 \cup \subst_2)\ t_2$.
%   \end{lemma}
  
  
% We start therefore to state a new version of
% \cref{th:mut-excl-head,th:mut-excl}
% %BBB

\begin{definition}[Mutually exclusive heads (\mutExclHeads) at index $d$]
  \label{def:mut-excl-head}
  % \begin{coqcode}
  %   Definition ~\customlabel{mutexcl}{\texttt{mutual\_exclusive}}\prog \pred\!\!~:
  %     ~$\forall\ i1\ o1\ bo1\ i2\ o2\ bo2$~ (H1: ~\clauseCmd{p}{i1\ o1}{bo1}~ \in ~$\prog\ p$~) (H2: ~\clauseCmd{p}{i2\ o2}{bo2}~ \in ~$\prog\ p$~),
  %         not (exists ~\subst\!\!~, ~\unifyCmd{i1}{i2}{\EmptySubst}{\subst}~)
  % \end{coqcode}
  Given a context \ctx\, an index $d$ and two clauses $c_1$ and $c_2$ with head,
  respectively, $\pred\ u_1 \ldots u_n$ and $\pred\ v_1 \ldots v_n$.
  If $\ctx\ \pred = \dtype{\detI}{i_1 \ldots i_k}{\_}$,
  then we say that
  $\mutExclHd\ c_1\ c_2\ d$ holds iff $1 \leq d \leq k \land\ \lnot(\unifiable\ u_d\ v_d)$.
\end{definition}

\noindent
Usually (see for example~\cite{1989Warren,1996Somogy}) static mode checking
plays a crucial role in most determinacy analsys algorithms: two mutually
exclusive heads may still unify with with the same query if it is flexible.
By tracking the flow of ground terms from the query to every predicate call
a mode checker can rule out this problematic situation.
However, since Elpi performs matching
rather than full unification on input arguments, the groundness condition is not
necessary.

\begin{lemma}[Mutually exclusive matching of split heads]
  Given a context \ctx\ and two clauses $c_1$ and $c_2$ with heads,
  respectively,  $\pred\ \vec{u}$ and $\pred\ \vec{v}$ then
  % forall vector of \textit{ground} terms $\vec{v}$,
  $$\forall d, \mutExclHeads\ \ctx\ c_1\ c_2\ d \Rightarrow
    \lnot (\exists t\ \subst_1\ \subst_2\ s.t.\ 
    \matchCmd{t}{u_d}{\EmptySubst}{\subst_1} \land
    \matchCmd{t}{v_d}{\EmptySubst}{\subst_2})
    % \forall t, \nUnify\ t\ \vec{u}_d \lor \nUnify\ t\ \vec{v}_d 
    $$
  \vspace{-2em}
  \label{th:mut-excl-head}
\end{lemma}

% \begin{proof}
%   By contradiction, let $t$ be a term matching with both $h_1$ and $h_2$. By
%   \cref{th:match-right}, we have that $\subst_1 h_1 = \subst_2 h_2$, this
%   implies that $(\subst_1 \cup \subst_2) h_1 = (\subst_1 \cup \subst_2) h_2$\todo{prove this}. By
%   the hypothesis $\mutExclHeads\ \ctx\ c_1\ c_2$, and by \cref{def:mut-excl}, it does
%   not exists a substitution allowing to unify the head of the two clause. This
%   is a contradiction.
% \end{proof}

% \begin{theorem}[Mutually-exclusive clauses property]
%   Given any $\ctx$ and any predicate $\prog$ that holds two clauses $c_1$ and
%   $c_2$ for $\pred$ such that $\mutExclHeads\ \ctx\ c_1\ c_2\ d$,
%   then % any $\wellModed\ (\pred\ \vec{t})$ is such that
%   $$
%     \mathcal{F}(\prog, \pred\ \vec{t}, \subst, \alts) = [\alts[_1], a_2]
%     \Rightarrow
%     \runCmdQ{a_1}{\EmptyList}{\subst}{a'} \Rightarrow
%     \lnot \exists \subst' a'', \runCmdQ{a_2}{\EmptyList}{\subst[']}{a''}
% $$
    
%   \label{th:mut-excl-head}
% \end{theorem}
% % \begin{lemma}[Mutually-exclusion with \match property]
% %   Given a context \ctx\ and a program \prog, if $\mutExcl\ \ctx\ \prog$ then for
% %   any call $\pred\ \vec{t}$, it exists at most one clause that can be
% %   successfully applied on $\pred\ \vec{t}$. 
% %   \label{th:mut-excl1}
% % \end{lemma}

% \begin{proof}
%   By \cref{th:mut-excl-head1,th:mut-excl+cut}
% \end{proof}

\noindent
We relax the condition above by accepting clauses that overlap in the head
if the following condition holds.

\begin{definition}[Mutually exclusive bodies\ (\mutExclCut)]
  Given two clauses
  $c_1$ and $c_2$ such that $c_1$ is defined before $c_2$,
  a \cut\ is in the body of $c_1$.
  \label{def:mut-excl-cut}
\end{definition}

% We can finally combine \cref{def:mut-excl-head} and \cref{def:mut-excl-cut}
% to obtain the definition of mutually exclusive clauses in a program.

\begin{definition}[Mutually exclusive clauses (\mutExcl)]\label{def:mut-excl}
  We say that $\mutExcl\ \ctx\ \prog$ holds
  iff for all predicate \pred such  that $\ctx\ \pred = \dtype{\detI}{\_}{\_}$
  and for any two clauses $c_1$ and $c_2$ in \prog\ \pred the following holds:
  % \vspace{-0.5em}
  $$
  \mutExclCut\ c_1\ c_2
  \ \lor\ 
  \exists d, \mutExclHeads\ \ctx\ c_1\ c_2\ d
  $$
\end{definition}

\noindent
Mutual exclusion ensures that at most one clause is applied on a query, however,
we still have to ensure no clause body leaves choice points.

\begin{definition}[Call to a deterministic predicate (\detAtom)]
  We say that an atom $t$ is a call to
  a deterministic predicate iff $t = \pred\ \vec{u}\ \land\ \ctx\ \pred = \dtype{\detI}{\_}{\_}$
\end{definition}

\begin{definition}[Deterministic premises (\detPrem)] \label{def:det-prem}
  We say that $\detPrem\ \ctx\ \prog$ holds iff for all predicate \pred such that
  $\ctx\ \pred = \dtype{\detI}{\_}{\_}$ and forall clause
  $(\clauseCmd{\pred}{\vec{t}}{b_1\dots b_n}) \in \prog\ \pred$ the
  following holds:
  $$(\exists j\ s.t.\  b_j = \cut \land \forall k > j, \detAtom\ \ctx\ b_k) \lor (\forall j, \detAtom\ \ctx\ b_j)$$
\end{definition}

\noindent
Since one can load clauses dynamically vial the implication operator we
need to ensure that these hypothetical clauses are mutually exclusive
with the ensiting ones. Recall that hypothetical clauses are added at
the top of the program, before any other clause.
\todo{write this somewhere}

\begin{definition}[Mutually exclusive hypothetical clauses (\locExcl)]
  We say that $\locExcl\ \ctx\ \prog$ iff
  for all clause $c = \clauseCmd{\pred}{\vec{v}}{t_1, \dots, t_n}$ in \prog
  such that $\ctx\ \pred = \dtype{\detI}{\_}{\_}$ occurring to the
  left of \impl\ (i.e. \piImplCmd{x}{c}{\_}) we have that
  $$\exists i\ s.t.\ t_i = \cut \land \forall k > i, \detAtom\ t_i$$
  % \vspace{-2em}
\end{definition}

% \noindent
%  local clauses should have a cut if they are clauses of a
% deterministic predicate and after this cut all premises left are call to
% deterministic predicates. This definition is a combination of
% \cref{def:mut-excl-head,def:mut-excl-cut} in the context of local clauses.

% Finally, the two definitions we need to be verified on a program are the one
% that uses the \checkc\ and \checkk\ procedure in
% \cref{fig:static-check,fig:det-assume} so that we have the guarantee that
% the body of all clauses respect the determinacy declared in the signature, the
% same check is performed on the outputs. Moreover we define what is the meaning
% of a good call to a predicate.

\begin{definition}[Determinacy check (\staticcheck)]
  $\staticcheck\ \ctx\ \prog$ holds iff
  for all clause $c$ in \prog, $\checkc\ \ctx\ c$ holds.% (see Fig~\ref{fig:full-det-check-rules}).
\end{definition}

\begin{definition}[Deterministic program (\detprog)]
 $\detprog\ \ctx\ \prog$ holds iff
 $\staticcheck\ \ctx\ \prog \land \mutExcl\ \ctx\ \prog \land \detPrem\ \ctx\ \prog
  \land \locExcl\ \ctx\ \prog$ holds.
\end{definition}

\begin{definition}[Deterministic query (\staticcheckq)]
 $\staticcheckq\ ctx\ \prog\ (\pred\ \vec{t})$ holds iff
  $\detAtom\ \ctx\ (\pred\ \vec{t})\ \land\ \checkCmd{\ctx}{\detI}{t}{\_}{\detI}$.% (see Fig~\ref{fig:full-det-check-terms}).
\end{definition}

% NOTA:
% \begin{definition}[Mutually-exclusive local clauses (\locExcl)]
%   Given a clause $c$, for any subterm in $c$ with the shape
%   $\piImplCmd{x}{(\clauseCmd{\pred}{i\ o}{B})}{D}$ then $x = i$ or a \cut is in
%   $B$. 
% \end{definition}
% È sbagliato per la mutual exclusion: controesempio
% programma `f Z 3`
% goal: pi x\ f x 1 :- body_with_no_bang => f x R
% Per backtracking potrei istanziare R a 1 o 3


% \begin{definition}[Mutual-exclusion in HOAS (\mutExclHO)]
%   Given a program \prog, forall deterministic predicate \pred, mutual exclusion is defined as
%   follows:
%   $$\mutExcl\ \prog \land (\forall \pred[']\
%   c, c\in \prog\ \pred['] \Rightarrow \locExcl\ c\ \pred)$$
%   \vspace{-2em}
%   \label{def:det-check-ho}
% \end{definition}


\begin{theorem}[Deterministic execution] \label{th:det2}
  $$
  \detprog\ \ctx\ \prog \land
  \staticcheckq\ \ctx\ \prog\ (\pred\ \vec{t}) \Rightarrow
  \isdet\ \prog\ (\pred\ \vec{t})
  $$
\end{theorem}

% \begin{proof}
%   The proof is similar to \cref{th:det1}. We only need to prove that the
%   property holds in a program where some rules for  \pred\ may be loaded locally
%   during the exectution of a subgoal. By definition of \ref{rule:piimpl}, any new
%   added local clause $c$ for \pred has the highest priority, therefore, to
%   ensure that \pred remains deterministic we need to consider two cases: 1) $c$
%   successfully applies on the current goal and 2) $c$ does not successfully
%   apply on the goal. The latter case is easy to prove since it is the induction
%   hypothesis. On the other hand, if $c$ successfully applies then, by the
%   hypothesis, we know that a \cut\ is in the body of $c$, this means that all
%   choice points are cut away, moreover, since, by the same hypothesis, all terms
%   after this cut are deterministic, it means that at most on solution is
%   returned for the call to \pred.
% \end{proof}


% \begin{theorem}[Determinacy checking with \match property]
%   Given a context \ctx\ and a program \prog, forall deterministic predicate \pred in \ctx,
%   we have
%   $$\hyperref[def:det-check]{\detCheck}\ \ctx\ \prog \Rightarrow \forall \vec{t}, \hyperref[def:is-det]{\isdet}\ \prog\ (\pred\ \vec{t})$$
%   \vspace{-2em}
%   \label{th:det1}
% \end{theorem}

% \begin{proof}
%   We proceed by induction on the derivations in
%   \cref{fig:basic-interp,fig:interp-match}. The structure of the proof closely
%   follows the one in \cref{th:det}, with a few notable differences. In
%   particular, this proof does not rely on the \wellModed\ hypothesis for the
%   program, nor does it assume the groundness of input arguments in the call to
%   \pred. The crucial insight lies in the use of the \match\ procedure, combined
%   with the result from \cref{th:mut-excl-head1}, which strengthens the proof.
%   %The
%   % most significant case to consider is \ref{rule:call}. Let $\vec{c}$ denote the
%   % result of $\prog\ \pred$. The function $\mathcal{F}$ produces a list of
%   % alternatives $\alt$, where each $a_i \in \alt$ consists of the unification
%   % between each term in $\vec{t}$ and the corresponding terms in the head of $c_i$,
%   % followed by the body of $c_i$.
%   % 
%   % Let $\alt = a_1, \dots, a_s, \dots, a_n$, where $a_s$ is the first alternative that
%   % can be successfully applied to the initial goal. We distinguish between two
%   % cases:  
%   % 1) $a_s$ contains a \cut.
%   % 2) $a_s$ does not contain a \cut.
%   % 
%   % In the first case, by hypothesis $\detPrem\ \ctx\ \prog$, the alternative
%   % $a_s$ has the form $b_1, \dots, b_x, \dots, b_m$, where $b_x = \cut$, and for
%   % all $b_j \in b_{x+1},\dots,b_m$, $b_j$ is a call to a deterministic predicate.
%   % The presence of the \cut\ discards the alternatives $a_{s+1},\dots,a_n$ as
%   % well as all choice points generated by the execution of the goals
%   % $b_1,\dots,b_{x-1}$. By the induction hypothesis, the execution of the goals
%   % $b_{x+1},\dots,b_m$ produces alternatives with no solution. Consequently, the
%   % goal is proved in this case.
%   % 
%   % In the second case, where $a_s$ does not contain a \cut, the hypothesis
%   % guarantees that all goals in $c_i$ are calls to deterministic predicates. By
%   % the induction hypothesis, the execution of these goals produces alternatives
%   % $\alt[']$ with no solution. The final list of alternatives returned by the
%   % call to \run is the concatenation of $\alt[']$ and $a_{s+1},\dots,a_n$. We need
%   % to prove that:
%   % $$\forall a_i \in \alt['] @ (a_{s+1},\dots,a_n), \lnot (\exists \alt\
%   % \subst['], \runCmd{a_i}{\EmptyList}{\EmptySubst}{\alt}{\subst[']})$$
%   % 
%   % 
%   % This holds for the alternatives in $\alt[']$. For any alternative $a_k \in
%   % (a_{s+1},\dots,a_n)$, the mutual exclusion hypothesis \mutExcl\ \ctx\ \prog\
%   % ensures that no alternative derived from clauses of \pred\ declared
%   % chronologically after $c_s$ can unify with the goal $\pred\ \vec{t}$, since
%   % every term in input position in $\vec{t}$ is ground. This completes the proof.
% \end{proof}

% The main difference between this last definition and \cref{def:mut-excl} is the
% absence of the \coqIn{HG} hypothesis and the usage of the \match procedure
% instead of \unify in the conclusion.

% \begin{theorem}
%   The \elpi input/ouput modes guarantee that for any
%   predicate $p$ whose clauses respect
%   \cref{def:emut-excl}, there exists at most
%   one succeeding clause for any call to $p$.
% \end{theorem}

% \begin{proof}
%   Without loss of generality, we take a program \prog with only binary
%   predicates representing respectively an input and an output. Let $p$ be a
%   predicate in \prog such that all clauses respect
%   \cref{def:mut-excl}. Let
%   ``$p\ t_1\ t_2$'' be a valid call for $p$. Let ``$c_1 :=
%   \clauseCmd{p}{t_1'\ t_2'}{b_1}$'' and ``$c_2 := \clauseCmd{p}{t_1''\
%   t_2''}{b_2}$'' be two clauses implementing $p$. 
%   Note that the absence of
%   groundness check avoid us from saying that $t_1$, which is the input of the
%   call, is a ground term. We reason by induction on the shape of $t_1$ and show
%   that it cannot \match simultaneously with $t_1'$ and $t_1''$, i.e. at most one
%   between $c_1$ and $c_2$ can be applied on the goal.
%   \begin{itemize}
%     \item Case 1: $t_1$ is a constant. A constant, in input position, matches with
%           the same constant or a variable. By the definition of \match
%           $t_1$ only matches with
%           the same constant or a variable.
%           Due to \cref{def:mut-excl}, $t_1'$ and $t_1''$ cannot be neither the
%           constant $t_1$ nor a unification variable nor a combination of the
%           two. Therefore $c_1$ and $c_2$ cannot be applied both of the call to
%           $p$.
%     \item Case 2: $t_1$ is a variable. A variable, in input position, matches
%           only with another variable. 
%           By \cref{def:mut-excl}, $t_1'$ and $t_1''$ cannot be both
%           unification variables. This means that at most one of the two clauses 
%           can be applied on the call to $p$.
%     \item Case 3: $t_1$ is a compond term: a term starting with rigid head with
%           potentially flexible subterms. If the heads of $t_1'$ and $t_1''$ have
%           the same head as $t_1$ then the unification of $t_1$ proceed on the
%           subterms, but, by induction hypothesis, only one between $t_1'$ and
%           $t_1''$ can unify with $t_1$. If the heads of $t_1'$ and $t_1''$ are
%           different then only we are sure that at most one of the two clause
%           can be applied on the call.
%   \end{itemize}
% \end{proof}


% As explained in \cite{1989Warren}, thanks to the (hard-)cut operator, mutual
% exclusiveness can be relaxed.

% \begin{definition}[Mutual exclusiveness with \cut]
%   Two clauses for the same predicate are mutually exclusive if the
%   chronological antecedent has a cut in its body.
%   \label{def:mut-excl-cut}
% \end{definition}

% This ensures that if we reach the cut
% operator in the first clause, the second clause is not considered as a choice
% point. Conversely, if one of the premises before the cut fails, then the second
% clause will be tried. In both situations, the two clauses cannot be applied
% simultaneously to the same predicate call.

% This definition allows overlapping clauses to
% exist in a database under the condition that the antecedent has a cut
% guaranteeing that at most one clause can be applied on a predicate call. 

% \begin{definition}[Mutual-exclusion + \cut]
%   Same as \cref{def:det-prem-cut}
% \end{definition}

% % \paragraph{deterministic clauses after last \cut}
% The second, but no less important, condition for a predicate to be deterministic
% is the following:

% \begin{definition}[deterministic premises after last \cut]
%   In each clause of a deterministic predicate, the premises after the last \cut
%   operator are only calls to deterministic predicates. 
%   \label{def:det-prem-cut}
% \end{definition}

% This guarantees that any output produced is uniquely determined, i.e. no two
% solutions can be produced on the same call.

% \begin{definition}[Deterministic predicate in \elpi]
%   \begin{coqcode}
%     Definition ~\customlabel{edetpred}{\texttt{edet\_pred}}~(~\prog~: prog) (p: pn) :=
%       forall ~$i$~ ~$o$~ ~$a$~
%         (H : ~\runCmd{[\goalCmd{\prog}{\callCmd{p}{i}{o}}{\EmptyList}]}{\EmptyList}{\EmptySubst}{a}{\subst}~), ~$a$~ = ~\EmptyList~.
%   \end{coqcode}
%   \label{def:edt-pred}  
% \end{definition}

% Our definition of deterministic predicate in \elpi (called \ref{edetpred} with a
% leading \coqIn{e} for \elpi) changes from \ref{detpred} in \cref{sec:det}: we do
% not need the \coqIn{HG} hypothesis: the usage of \elpi modes allows to pass any
% (even not ground) term in input position. The derivation rule \ruleCall allows
% makes the difference between terms that should be unified with the \unify or
% the \match procedure at runtime.

% \begin{definition}[Determinacy checking in \elpi]
%   Determinacy checking (noted \coqIn{edet_check}) on a program
%   \prog is equivalent by the combination of 
%   \cref{def:emut-excl,def:mut-excl+cut,def:det-prem}
%   \label{def:det-check}
% \end{definition}

% The following lemma says that in a determinacy-checked program, if \pred is
% a deterministic-annotated program and the \run\ of a call to \pred gives a 
% solution, then the same solution is returned by a run of the same goal
% in a program where all of the clauses of \pred are rewritten such
% that thier last atom is a cut.

% \begin{lemma}
%   Let \tailcut be a function taking a program \prog and predicate \pred
%   returning a new program \prog['] such that the bodies of all clauses of \pred
%   in \prog have been added a \cut\ as last atom.

%   Let \pred be a deterministic-annotated predicate,
%   \begin{coqcode}
%     Lemma det_tail_cut ~\prog \alt~:
%       forall i o a ~\subst \subst[']~ (H: det_check ~\prog\!\!~)
%         (HR: ~\runCmd{[\goalCmd{\prog}{\callCmd{\pred}{i}{o}}{\alt}]}{[]}{\subst}{a}{\subst'}~),
%           ~\runCmd{[\goalCmd{(\tailcutCmd{\prog}{\pred})}{\callCmd{\pred}{i}{o}}{\alt}}{[]}{\subst}{a}{\subst'}~.
%   \end{coqcode}
%   \label{lemma:prog-all-cut}
% \end{lemma}

% \def\clauseL{\ensuremath{\mathcal{L}}\xspace}
% \begin{proof}
%   We reason by induction on \coqIn{HR}: $5$ cases should be taken
%   one per derication rule in \cref{fig:basic-interp}.
%   \begin{itemize}
%     \item Case \ruleStop: cannot be applied since the list of goals is not empty.
%     \item Case \ruleFail: the \texttt{fail} hypothesis tells that no
%           there is no implementation for the predicate \pred, therefore
%           adding a tail-cut to the rules of \pred does not change the behaviour
%           of the program.
%     \item Case \ruleUnif: cannot be applied since the first goal is not a
%           unification or a match.
%     \item Case \ruleBang: same problem as before with the \cut operator.
%     \item Case \ruleCall: there exists at least one rule implementating \pred.  
%           Due to the \tailcut function the list of new goal, together with the
%           alternatives have a \cut\ has their last atom. Moreover, the list
%           of cut-alternatives in each of these atoms is the empty list.
%           By the hypothesis \coqIn{H}, we know that ...
%   \end{itemize}
% \end{proof}

% % Thanks to \cref{th:all-cut}, we can give a common structure to all the clauses
% % of a deterministic predicate: we are free to assume that $c$ always has at
% % least one cut.

% The following lemma says that if all clauses of a predicate \pred have a cut as
% last atom in their body, if the \run\ of a call to this predicate as a goal with
% an empty list of alternatives gives an output, then the alternatives of the
% ouput are the empty list.

% \begin{lemma}[Tail-cut and cut-alternatives]
%   $$
%   \begin{array}{l}
%   \forall \subst\ \subst[']\ i\ o\ p,\\
%   \runCmd{[\goalCmd{(\tailcutCmd{\prog}{\pred})}{\callCmd{p}{i}{o}}{a}]}{[]}{\subst}{x'}{\subst'} \rightarrow [] = x'
%   \end{array}
%   $$
%   \label{lemma:cut-cat-alt}
% \end{lemma}

% \begin{proof}
%   INTUITION: The cut-alternative of the last cut is the empty list
% \end{proof}

% % \begin{corollary}
% %   If all clauses of a predicate have a cut, then at most one of
% %   these clauses can be applied successfully on the goal, the other being
% %   cut away.
% %   \label{cor:only-one-clause}
% % \end{corollary}

% % \begin{proof}
% %   By a slightly modified version of \cite{1989Warren}
% % \end{proof}

% %
% \begin{theorem}[determinacy check $\Rightarrow$ deterministic pred]
%   For any deterministic-annotated predicate \pred,
%   \begin{coqcode}
%     Theorem det_check_det_pred ~\prog~ (H: edet_check ~\prog\!\!~):
%       ~\ref{edetpred} \prog \pred~.
%   \end{coqcode}
% \end{theorem}

% % Before giving the proof, we just want to point out that using input/output modes
% % of \elpi, no hypothesis on the groundness of terms can be done. 

% \begin{proof}
%   By \cref{lemma:prog-all-cut} and the hypothesis \coqIn{H}, the conclusion can
%   be rewritten such that all clauses of \pred have a cut as last atom in their
%   bodies. Finally, thanks to \cref{lemma:cut-cat-alt} we can conclude the proof.
% \end{proof}


\subsection{Digression on \isdet}

\begin{definition}[Functionality (\isfunc)]\label{def:is-func}
  Given a program \prog and a term $t$,
  \isdet\ is defined as follows:
  $$\forall a\ \subst, \isFuncCmd{t}$$
\end{definition}

We insist on \isdet{} rather than \isfunc{} or functional as in~\cite{1989Warren}
becase the former gives the programmer a more precise indication of the
run time behavior of a predicate labelled as \func. If \texttt{mem} and \texttt{len}
are flagged as such, then the programmer knows that this silly piece of
code terminate in linear time.

\begin{elpicode}
silly L V :- mem 1 L, len L N, N = V.
\end{elpicode}

Upon failure under a notation of functional as in~\cite{1989Warren} the complexity
could be quadratic since a mem (without a cut) is still a function. If mem
was \isfunc{} the code would still be linear but with the cost of a choice point.
With \isdet{} the code is operationally equivalent to the following code where
the complexity, even in case of failure, is very apparent.

\begin{elpicode}
once P :- P, !.
silly L V :- once (mem 1 L), once (len L N), N = V.
\end{elpicode}

We are so concerned with failure in case of performance because of Elpi's
application as the runtime for type class resolution in the interactive prover
Rocq: the user experience crytically relies on quick feedback from the system.

% A definition of functional as in warren, or like 
% \begin{definition}[Functionality (\isfunc)]\label{def:is-func}
%   Given a program \prog and a term $t$,
%   \isdet\ is defined as follows:
%   $$\forall a\ \subst, \isFuncCmd{t}$$
% \end{definition}
% does not have this guarantee.

% \begin{coqcode}
% Definition ~\customlabel{detpred}{\texttt{det\_pred}}~(~\prog~: prog) (p: pn) :=
%   forall ~$i$~ ~$o$~ ~$a$~ ~\subst~(HG: ground i) 
%     (H : ~\runCmd{[\goalCmd{\prog}{\callCmd{p}{i}{o}}{\EmptyList}]}{\EmptyList}{\EmptySubst}{a}{\subst}~), ~$a$~ = ~\EmptyList~.
% \end{coqcode}

% The definition above explains what it means for a predicate to be deterministic. 
% The idea is that in a given program \prog and a predicate $p$,
% if for any call to $p$ with arbitrary input and output terms
% starting with the empty substitution and the empty list of alternatives,
% we have a solution, i.e. the couple $(a, \subst)$ then the list of
% alternatives $a$ is empty. This essentially means that any functional
% predicate produces no choice points.

% Following the literature, this claim is proven true:

% \begin{theorem}
%   Let \pred be a deterministic-annotated predicate, the following holds
%   \begin{coqcode}
%     Theorem ~\customlabel{is_detpred}{\texttt{det\_pred\_prop}}~(~\prog~: prog) (p: pn) :=
%       forall (HM : well_moded ~\prog\!\!~) (HD: det_check ~\prog\!\!~),
%           det_pred ~\prog~p
%   \end{coqcode}    
% \end{theorem}

% This to say that for any well-moded (hypothesis \coqIn{HM}) and
% determinacy-checked (hypothesis \coqIn{HD}) program and a predicate \pred such,
% then \pred is a deterministic predicate.

% \subsection{Contributions and paper structure}

% In our paper we will provide a description about a new dynamic mode and
% a new static determinacy
% checkers to verify that clauses are consistent wrt the signature the predicate has. We
% introduce these concepts with the \elpi programming language. We start by a
% light version of \elpi in the first-order setting, i.e. with no higher-order variables.
% In this part we explain our notion of modes, which have an inpact
% on the dynamic interpretation of the program. This will force us to slightly
% modify the interpreter in \cref{fig:basic-interp}. Moreover, thanks to these modes, we show that
% to statically guarantee the determinacy property of deterministic predicates
% we don't need any static mode analysis.

% In the second part of the paper we will extend the intepreter so that it can\todo{there are 3 parts}
% work with higher-order variables, hereditary-arrop clauses insertion and local
% $\forall$-quantified variable declaration. Here predicates will also take
% propositions as arguments. Thanks to this extention we will propose a new mode
% checking algorithm working with a new definition of groundness, called
% \textit{input-ground}. The signature of a predicate can be annotated so that
% output arguments are guaranteed to be deterministic. This will make the
% determinacy checker to analyse not only deterministic clauses but also relational
% one, since a check should be performed wrt the determinacy annotation of its
% outputs.

\subsection{Digression on  the eagherness of $\mathcal{H}$}

The function $\mathcal{H}$ eagerly unifies all clauses with the.
A more natural and possibly more efficient semantics would be to just create the
alternatives and prepend to the list of goals the same unification
problems. Our choice simplifies the formal threatment of determinacy 
in~\cref{sec:thm} and in particular it matches the \mutExclHeads\ condition.

From a practical stanpoint a implementation of the semantics
above with efficiency concerns can make $\mathcal{H}$ lazy
without compromising the \mutExclHeads\ condition
can either index clauses deep enough
or perform a program transformation consisting in inserting
\elpiIn{once} around each call to a deterministic predicate.


% \section{Prolog with higher-order variables in datatype}
\label{sec:hoas}

An improvement of the current version of the interpreter comes with the fact
that the \elpi accepts hareditary-harrop formulas and higher-order pattern
unification (\cite{1991miller-pf}), noted \pf. This extension allows to
dynamically modify the program by inserting fresh variables and clauses.
Moreover, in order to express using the higher-order abstract syntax (HOAS,
\cite{1988pfenning}) we also add $\lambda$-abstractions to the language.
% 
% In order to pass to the full \elpi language, we need to change the way terms and clauses are 
% represented. We give, therefore, their implementation below.
% 
% \begin{coqcode}
%   Inductive tpos := 
%     | a (i:base)            
%     | impl (c:tneg) (b:tpos)   
%     | pi   (n:string) (b: tpos)
%     | conj (l:list tpos)       
%     | cut                      
%     | unif  (x:any) (y:any)    
%     | match (x:any) (y:any)    
%   with tneg :=
%     | hb  (h: base) (b: list tpos)
%     | piN (x: string) (b : tneg)  
%   with base :=
%     | c (c : pn) (l:list any)
%     | v (v : vn) (l:list any)
%   with any := 
%     | p (i:tpos) 
%     | n (i:tneg) 
%     | b (i:base) 
%     | s (i:string).
% \end{coqcode}
% 
% NOTA: mi sembra sia più facile avere predicati di arità qualsiasi
% 
% \begin{coqcode}
%   | PiImpl (x:vn) (c:clause) (b:tm)
% \end{coqcode}
%
\begin{align}
  data &::= ... \mid data \to data\\
  tm &::= ... \mid \piImplCmd[type]{\predVar}{clause}{tm} \mid \lambda \predVar: type.tm
\end{align}

The term \piImplCmd[ty]{x}{H}{B} introduces $x$ as a local fresh variable with
type $ty$ inside \implCmd{H}{B}. Then the clause $H$ is added in the scope of
$B$ with highest priority. Note that this constructor can be split in two, one
talking about the $\pi$, the other talking about the $\impl$ operator, we prefer
however to have a sole constructor. It is always possible, in fact, to quantify
dummy variables and hypothesis, reproducing exactly the behaviors of the single
operators. Finally, the $\lambda x: ty.t$ construct allows to abstract the
variable $x$ with type $ty$ inside $t$. We also have extended the non-terminal
$data$ so that the type of $\lambda$-expressions can be represented.

\subsection{Interpreter for prolog with higher-order variables in datatype}


To account for the new constructor in the language, we extend our interpreter.
The updated rule is shown in \cref{fig:interp-piimpl}.  

\begin{figure}
  \rulePiImplM{.85}
  
  % \rulePiM{.85}

  % \ruleLamM{.85}
  \caption{Dynamic semantics: rule for higher-order prolog}
  \label{fig:interp-piimpl}
\end{figure}

We can remark that the terms accepted by the interpreter are still terms in the
canonical form, i.e. no lambda abstraction can appear in the head of a
application. Moreover, we intentionally keep out from the interpreter
application with head being variables. This essentially means that variables are
still not executable piece of code.  Variables will play an important role in
\cref{sec:vars}.


We modify the signature of a \goal, which now takes a context \ctx instead of
just a program \prog. This context, denoted as \ctx, consists of a pair: a
program \prog and a set of local variables \env. Consequently, all occurrences
of \prog in \cref{fig:basic-interp} should now be read as \ctx.  

When the function $\mathcal{F}$ searches for the implementation of a predicate,
it should extract the first component of \ctx\ to obtain the expected output.

The notation $H + \prog$ is the insertion of the clause $H$ 
in the program \prog so that $H$ has
the highest priority in \prog. The notation
$x \uplus \env$\todo{penso si possa togliere}
is the insertion of the local variable $x$ inside the set of local binders \env
provided that $x \notin \env$.

% The rule \ruleImpl handles the atom \implCmd{H}{B}, and its derivation rule
% attempts to solve \( B \) under a program extended with the clause \( H \). The
% rule \rulePi introduces the name of the fresh variable into the program.  

The rule \ref{rule:piimpl} in \cref{fig:interp-piimpl} is interpreted by running the
goal under the \pi\ and the \impl\ charging both the local variable and clause
inside its scope.

The \unify and \match procedures now consider \env\ to perform \pf\
unification. \env is required to know if a term is in the pattern fragment.

% Finally the \ruleLam makes a $\beta$-reduction step and keep solving the goal.

\subsection{Determinacy checker for prolog with higher-order variables in datatype}

The insertion of the new rule makes the determinacy checker to change
accordingly. In particular, the rule $H$ charged by \impl\ in
\cref{fig:interp-piimpl} may break \cref{th:mut-excl}. This is because the rule
being added may overlap with other existing rules. This means that the execution of $B$ may
produce several results which is not the desired behavior if the $B$ should be
deterministic.

In order to address this problem, we need to ensure that the dynamic insertion
of the new clause $H$ in the program is such that if $H$ successfully applies on a goal
then all the other clause fails. To ensure this propery to following condition
should be granted.

\begin{definition}[Local clauses: mutually-exclusion + det. premises (\locExcl)]
  Given a clause $c$ and a predicate \pred, for any subterm $s$ of $c$, then
  $$s = \piImplCmd{x}{(\clauseCmd{\pred}{i\ o}{t_1, \dots, t_n})}{D} \to \exists
  i, t_i = \cut \land (\forall k, k > i \to \detAtom\ a_i)$$
  \vspace{-2em}
\end{definition}

% NOTA:
% \begin{definition}[Mutually-exclusive local clauses (\locExcl)]
%   Given a clause $c$, for any subterm in $c$ with the shape
%   $\piImplCmd{x}{(\clauseCmd{\pred}{i\ o}{B})}{D}$ then $x = i$ or a \cut is in
%   $B$. 
% \end{definition}
% È sbagliato per la mutual exclusion: controesempio
% programma `f Z 3`
% goal: pi x\ f x 1 :- body_with_no_bang => f x R
% Per backtracking potrei istanziare R a 1 o 3

Thanks to this we can define the mutual-exlusion definition in the higher-order
setting.

\begin{definition}[Mutual-exclusion in HOAS (\mutExclHO)]
  Given a program \prog, forall deterministic predicate \pred, mutual exclusion is defined as
  follows:
  $$\mutExcl\ \prog \land (\forall \pred[']\
  c, c\in \prog\ \pred['] \to \locExcl\ c\ \pred)$$
  \vspace{-2em}
  \label{def:det-check-ho}
\end{definition}


\begin{theorem}[Determinacy checking in HOAS]
  Given a program \prog and a predicate \pred
  \begin{align*}
    &\detCheckHO\ \prog\ \pred \to \forall i\ o,\runCmd{[\goalCmd{\prog}{\callCmd{p}{i\ o}}{\EmptyList}]}{\EmptyList}{\EmptySubst}{a}{\subst} \to a = \EmptyList    
  \end{align*}
  \vspace{-2em}
  \label{th:det2}
\end{theorem}

\begin{proof}
  The proof is similar to \cref{th:det1}. We only need to prove that the
  property holds in a program where some rules for  \pred\ may be loaded locally
  during the exectution of a subgoal. By definition of \ref{rule:piimpl}, any new
  added local clause $c$ for \pred has the highest priority, therefore, to
  ensure that \pred remains deterministic we need to consider two cases: 1) $c$
  successfully applies on the current goal and 2) $c$ does not successfully
  apply on the goal. The latter case is easy to prove since it is the induction
  hypothesis. On the other hand, if $c$ successfully applies then, by the
  hypothesis, we know that a \cut\ is in the body of $c$, this means that all
  choice points are cut away, moreover, since, by the same hypothesis, all terms
  after this cut are deterministic, it means that at most on solution is
  returned for the call to \pred.
\end{proof}


% check that
% the body of each local clauses (even those appearing inside the body of
% \lam-abstraction) has a \cut\ in its body. Therefore, the goal
% $(\lamCmd{x}{\implCmd{x}{t_1}}) t_2$, where the local clause is loaded inside
% the body of the \lam-abstraction is rejected: the static checker does not
% perform any reduction in the code, nor it knows the shape of $t_2$. On the other
% hand, the \mutExclHeads\ procedure will accept the term
% $(\lamCmd{x}{\implCmd{(\clauseCmd{f}{x\ 1}{!})}{t_1}}) t_2$.

% This condition is sufficient to make \cref{th:mut-excl} to work again: since the
% checker ensures that all clauses have a cut in their body, and since by
% hypothesis, all the clauses of the program satisfy \cref{def:mut-excl}, then the
% lemma is valid.
% Attenzione, guardare che la regola abbia alemno una variabile local in posizione
% di input non è sufficiente: esempio `pi x\ f x 1 => f x 2 => ...

% To address the second problem, we need to ensure that if a predicate is
% deterministic and has a $\beta$-reduction ``$(\lamCmd{x}{B})T$'' after the last
% \cut, then the reduced term is still a call to a deterministic predicate. This
% check can be performed by looking to the shape of $B$. We know that, by
% construction, $B$ may contains a tower of \pi and hypothetical clauses lodaded
% with \impl. At the bottom of this tower we will have either a \cut, which does
% not impact \cref{def:det-prem}, or a call. This call may use $B$ has the head of
% the predicate, e.g. take $(\ruleLam{x}{x 1 2}{t_1})$. In the system of
% derivation rules, due to the lack of an interpretation for variables, we can
% safely suppose that $T$ ($t_1$ in the example) is a rigid term. Therefore,
% by looking to its determinacy-annotation, we know if the $\beta$-reduced term is
% deterministic or not. On the other hand, 
%\section{Determinacy analysis of Horn Clauses with input matching}
\label{sec:basic-elpi}

As anticipated in the end of \cref{sec:modes}, the hypothesis that input
argument are ground, such as in \cref{th:det} is a too strong condition for our
use case. In automatic proof search, the input arguments may contain flexible
terms. They will actually assigned by the search engine. Similarly, proof search
may produce non-ground output terms. This means that \cref{def:well-moded}
should be relaxed.

In our first-order prolog system we work with the same data structures described
in the previous section, however, we change the unification algorithm wrt the
mode of the terms that are being unified.

\subsection{Modes and \match}
\label{sec:elpi-modes}

% Mode checking in this setting is performed using the classical groundness
% verification where term groundness is derived from their flow in the body of a
% clause. We start with the hypothesis that a term in input position in the head
% of a clause is ground. The premises in the body representing calls to predicates
% must be called with ground input terms, while output are supposed to become
% ground. At the end of the body analyse, we check that the output terms in the
% head of the clause has become ground. If this is not the case, or if a call in
% the body is done with a non-ground term in input position a mode-checking error
% is raised.

The modes we authorize in this setting are two, even though we call them
the input and the output
mode, ther interpretation we attach to them is quite different. 
% As explained at the very end of \cref{sec:modes}, we want predicates
% to be called with non-ground input terms, therefore
% the meaning we attach to our modes is quite particuar wrt what
% can be found in the literature.

We start from the concept of terms read and write as
sketched in \cite{1991ait-wam}. Essentially, in a call, a term in input position is an
object received in read mode. This means that, if the entry
term is a variable and the corresponding term in the head of the clause is
rigid, unification fails: we cannot write in input in the head of a clause.

On the other hand, it is always possible to unify a non-flexible input term with
a variable if that variable appears in the head of the clause: we are consuming
the pieces of information of the input to select the clause in the program. 

% Finally, no
% constraints are imposed on output terms: they are used to produce information
% and therefore they can be written. In a call an output is not forced to be a
% variable to be instantiated, but also a compound term.

We refer to this special type of unification as \match. Similar to the \unify
procedure, we use the notation \matchCmd{t_1}{t_2}{\subst}{\subst[']} to denote
the \match\!ing between two terms $t_1$ and $t_2$, transforming, if
possibile, an initial substitution \subst into a final substitution \subst['].

\def\vars{\texttt{vars}}

\begin{definition}[The \match procedure]
  Let \coqIn{vars(t)} be a function that returns the variables in a term
  \coqIn{t}. Then \match is defined as follows:
  \vspace{-1em}
  $$\forall\ t_1\ t_2\ \subst\ \subst['], \unifyCmd{t_1}{t_2}{\subst}{\subst[']} \land \vars\ (\sigma t_1) = \vars\ (\subst['] t_1)$$
  \vspace{-2em}
  % \begin{coqcode}
  %   Definition ~\matchCmd{t_1}{t_2}{\subst}{\subst[']}~ :=
  %     ~\unifyCmd{t_1}{t_2}{\subst}{\subst[']}~ /\ 
  %       ~$\forall$~ v, v \in vars(~$t_1$~) -> is_var ~(\subst['] $t_1$)~
  % \end{coqcode}
\end{definition}

As a mean of example, consider following program.

\begin{elpicode}
  pred p i:int, o:int.
  p 1 1.          % p1
  p X 2 :- X = 7. % pX
\end{elpicode}


The predicate \elpiIn{p} is decorated with a type\&mode signature. It tells not
only that the first two arguments should be of type \elpiIn{int}, but also that
the first argument is an input (\modeAlone{i}) and the second an output
(\modeAlone{o}). This means that the query \elpiIn{p Y Z} succeed exactly once:
the rule \elpiIn{p1} could not be applied, since the variable \elpiIn{Y} is in
input (i.e. read mode) and cannot be instantiated to $1$ in the head of the
clause. Note, that the application of \elpiIn{pX} on the query succeed and
producing the substitution $\subst = \{Y \mapsto 7, Z \mapsto 2\}$: \elpiIn{Y}
can be instantiated in the body of a clause. Moreover, the query
\elpiIn{p 1 Z} has one solution with $\subst = \{Z \mapsto 1\}$ due to the
application of \elpiIn{p1}. Note that the rule \elpiIn{pX} is a valid choice
point for the query: the head unifies with the query assigning the local
$\exists$variable \elpiIn{X} to $1$, but a failure will occur in the body
while unifying the value of \elpiIn{X} and \elpiIn{7}.

\begin{corollary}[\match is not reflexive]
  The \match operator is not reflexive:
\end{corollary}
\begin{proof}
  A counter-example: \matchCmd{1}{X}{\EmptySubst}{\{X \mapsto 1\}},
  but \matchCmd{X}{1}{\EmptySubst}{\text{\xmark}}.
\end{proof}

\begin{lemma}[\match: unification of right terms]
  Given three terms $t$, $t_1$, and $t_2$:
  $$\forall\ \subst_1\ \subst_2, 
    \matchCmd{t}{t_1}{\EmptySubst}{\subst_1} \to 
    \matchCmd{t}{t_2}{\EmptySubst}{\subst_2} \to 
    \subst_1\ t = \subst_2\ t $$
  \vspace{-2em}
  \label{th:match-right}
\end{lemma}

\begin{proof}
  By hypothesis, $\subst_1\ t = \subst_1\ h_1$ and $\subst_2\ t = \subst_2\
  h_2$. Due to the definition of \match and since $t$ is the term passed in
  input in both \match, no variables in $t$ is assigned in neither
  $\subst_1$ nor $\subst_2$, therefore $\subst_1 t = t$ and $\subst_2 t = t$.
  Now, after some rewriting, we have $\subst_1\ h_1 = \subst_2\ h_2$, which 
  concludes the proof.
\end{proof}

% Moreover, \elpi provides a way to further control the unification on input
% arguments. The user can put the \uvar keyword in front of a variable name in a
% input argument. This tells the unification engine to unify this head term with
% only flexible term. For example, we can add the rule ``\elpiIn{pUX} := \elpiIn{p
% (uvar X) 2 :- X = 7}'' to the previous database. The \uvar keyword does not
% allow the call \elpiIn{p 1 Z} to be unified with \elpiIn{pUX} since \elpiIn{1}
% is not a variable.

% We think that the combination of the input mode and the \uvar keyword give
% the user a powerful tool to  control how unification is dynamically performed
% at runtime.

We add the new term to the language:
% \begin{coqcode}
%   | UnifyM (t1:tm) (t2:tm) : atom
% \end{coqcode}
$$tm ::= ... \mid tm =_m tm$$

The introduction of dynamic modes requires modifying the interpreter so that the
\match procedure is called when unification is performed on input terms. To
achieve this, we need to slightly adjust the $\mathcal{F}$ function used by the
\ref{rule:call}, as follows:\todo{riscrivere con indici diversi}%
% $$
% \mathcal{F}(\prog, \pred, i, o, \subst, \alt) := 
%   [(\subst, (
%       \Cons{(\prog, i =_m i', \alt)}
%         \Cons{(\prog, o = o', \alt)}
%           {[(\prog, g, \alt) \mid g \in bs]})) \mid \clauseCmd{\pred}{i'\ o'}{bs} \in \prog\ \pred]
% $$
%
\begin{align*}
  \mathcal{F}(\prog, \pred, \vec{t}, \subst, \alt) :=
  \Bigg[\bigg(\subst,
    \Big(
    &\big[(\prog, (\vec{t}_i)_x =_m (\vec{u}_i)_x, \alt) \mid 0 \leq x < \text{len }\vec{t}_i\big] @ \\
    &\quad\big[(\prog, (\vec{t}_o)_x = (\vec{u}_o)_x, \alt) \mid 0 \leq x < \text{len }\vec{t}_o\big] @\\
    &\qquad\big[(\prog, g, \alt) \mid g \in bs\big]
    \Big)\bigg) \mid \clauseCmd{\pred}{\vec{u}}{bs} \in \prog\ \pred\Bigg]
\end{align*}

The main difference is that the usage of the $=_m$ to indicate that the
unification between input arguments in the goal and in the head of the clause
should be performed using the \match procedure.

Finally the derivation system in \cref{fig:basic-interp} is extended with the
rule in \cref{fig:interp-match}.

\begin{figure}
  
  \ruleMatchM{1}
  
  \caption{Dynamic semantics: rule for \match}
  \label{fig:interp-match}
\end{figure}


% \textcolor{red}{\textbf{TODO}}: add the derivation rule for the $=_m$ atom
% Recall:

% The \textit{unify rule} (\ruleUnif) is applied when a unification $t_1 =_m t_2$
% occurs in the head of the current goal list. Depending on $m$, which can be
% either $i$ or $o$, it calls respectively the \unify or the \match (see
% \cref{sec:elpi-modes}) algorithm between the two terms $t_1$ and $t_2$ from the
% substitution \subst and returns the updated substitution \subst[']. Finally the
% \run\ procedure is called the list of remaining goals \g and the new
% substitution \subst['].

\subsection{Determinacy checker for prolog with \match}

Mode checking is an essential ingredient for ensuring determinacy.

% Before explaining the interaction between these two, we need to take
% some time to talk about mutual clauses exclusiveness.
% %
% % \paragraph{Mutual exclusive clauses}
% From \cite{1989Warren}, we know that at most one clause can be applied
% successfully for any deterministic predicate.

% In order to satisfy this (necessary by not sufficient) condition, we need 
% that all the clauses of a deterministic predicate $p$ are mutually exclusive.
% Mutual exclusiveness can be stated as follows:

% \begin{definition}[Non-overlapping clauses]
%   For any pair of clauses of the same predicate ``\clauseCmd{p}{\vec{x}}{b1}''
%   and ``\clauseCmd{p}{\vec{y}}{b2}'', we say that they are non-overlapping if
%   there exists an input position $i$ such that $\forall \sigma, \sigma\ x_i \neq
%   \sigma\ y_i$
%   \label{def:mut-excl}
% \end{definition}

\Cref{th:det} ensures that at most one clause can be applied to any given
predicate, provided that the input terms are ground. However, since we no longer
perform mode-checking and our inputs are not necessarily ground, the theorem is
no longer valid.
%
% Thanks to mode checking, any call is validated only if its terms in input
% positions are ground. Groundness % (modulo \cut, see \cref{def:mut-excl+cut})
% guarantees that at most one clause can be executed successfully for a given
% call.
%
% We also emphasize that mode checking ensures that output terms become ground.
% This is a fundamental property; otherwise, outputs would be meaningless—if an
% output does not become ground, it cannot serve as the input for another
% predicate call.
%
% In our setting, we slightly extend this definition so that the \elpi's \uvar
% keyword is taken into account.
%
% \begin{definition}[Mutual exclusiveness with \uvar]
%   A clause with an input term marked with the \elpiIn{uvar} keyword,
%   does not overlap with any other rigid-head term.
%   \label{def:mut-excl-uvar}
% \end{definition}
%
% This means that a term marked with the \uvar keyword in the head of a clause
% overlaps only with unification variables or with another term marked with
% \uvar.
%
% In our first-order \elpi, we do not perform any static mode analysis. Instead,
% we rely on the \match operation, which is dynamically applied to input arguments at runtime.

We start therefore to state a new version of
\cref{th:mut-excl-head,th:mut-excl}
%BBB

\begin{lemma}[Mutually-exclusive heads with \match property]
  Given a context \ctx and two clauses $c_1 = \clauseCmd{\predVar}{\vec{t}}{b_1}$ and $c_2 = \clauseCmd{\predVar}{\vec{u}}{b_2}$ then
  let $\vec{t}_i$ and $\vec{u}_i$ be the inputs arguments of $\vec{t}$ and $\vec u$ respectively.
  % forall vector of \textit{ground} terms $\vec{v}$,
  $$\mutExclHeads\ \ctx\ c_1\ c_2 \to \forall \subst_1\ \subst_2, \lnot (\exists \vec{v},
    \matchCmd[1]{v}{t_i}{\EmptySubst}{\subst_1} \land
    \matchCmd[1]{v}{u_i}{\EmptySubst}{\subst_2})$$
  \vspace{-2em}
  \label{th:mut-excl-head1}
\end{lemma}

\begin{proof}
  By contradiction, let $t$ be a term matching with both $h_1$ and $h_2$. By
  \cref{th:match-right}, we have that $\subst_1 h_1 = \subst_2 h_2$, this
  implies that $(\subst_1 \cup \subst_2) h_1 = (\subst_1 \cup \subst_2) h_2$\todo{prove this}. By
  the hypothesis $\mutExclHeads\ \ctx\ c_1\ c_2$, and by \cref{def:mut-excl}, it does
  not exists a substitution allowing to unify the head of the two clause. This
  is a contradiction.
\end{proof}

\begin{lemma}[Mutually-exclusion with \match property]
  Given a context \ctx\ and a program \prog, if $\mutExcl\ \ctx\ \prog$ then for
  any call $\pred\ \vec{t}$, it exists at most one clause that can be
  successfully applied on $\pred\ \vec{t}$. 
  \label{th:mut-excl1}
\end{lemma}

\begin{proof}
  By \cref{th:mut-excl-head1,th:mut-excl+cut}
\end{proof}

\begin{theorem}[Determinacy checking with \match property]
  Given a context \ctx\ and a program \prog, forall deterministic predicate \pred in \ctx,
  we have
  $$\hyperref[def:det-check]{\detCheck}\ \ctx\ \prog \to \forall \vec{t}, \hyperref[def:is-det]{\isdet}\ \prog\ (\pred\ \vec{t})$$
  \vspace{-2em}
  \label{th:det1}
\end{theorem}

\begin{proof}
  We proceed by induction on the derivations in
  \cref{fig:basic-interp,fig:interp-match}. The structure of the proof closely
  follows the one in \cref{th:det}, with a few notable differences. In
  particular, this proof does not rely on the \wellModed\ hypothesis for the
  program, nor does it assume the groundness of input arguments in the call to
  \pred. The crucial insight lies in the use of the \match\ procedure, combined
  with the result from \cref{th:mut-excl-head1}, which strengthens the proof.
  %The
  % most significant case to consider is \ref{rule:call}. Let $\vec{c}$ denote the
  % result of $\prog\ \pred$. The function $\mathcal{F}$ produces a list of
  % alternatives $\alt$, where each $a_i \in \alt$ consists of the unification
  % between each term in $\vec{t}$ and the corresponding terms in the head of $c_i$,
  % followed by the body of $c_i$.
  % 
  % Let $\alt = a_1, \dots, a_s, \dots, a_n$, where $a_s$ is the first alternative that
  % can be successfully applied to the initial goal. We distinguish between two
  % cases:  
  % 1) $a_s$ contains a \cut.
  % 2) $a_s$ does not contain a \cut.
  % 
  % In the first case, by hypothesis $\detPrem\ \ctx\ \prog$, the alternative
  % $a_s$ has the form $b_1, \dots, b_x, \dots, b_m$, where $b_x = \cut$, and for
  % all $b_j \in b_{x+1},\dots,b_m$, $b_j$ is a call to a deterministic predicate.
  % The presence of the \cut\ discards the alternatives $a_{s+1},\dots,a_n$ as
  % well as all choice points generated by the execution of the goals
  % $b_1,\dots,b_{x-1}$. By the induction hypothesis, the execution of the goals
  % $b_{x+1},\dots,b_m$ produces alternatives with no solution. Consequently, the
  % goal is proved in this case.
  % 
  % In the second case, where $a_s$ does not contain a \cut, the hypothesis
  % guarantees that all goals in $c_i$ are calls to deterministic predicates. By
  % the induction hypothesis, the execution of these goals produces alternatives
  % $\alt[']$ with no solution. The final list of alternatives returned by the
  % call to \run is the concatenation of $\alt[']$ and $a_{s+1},\dots,a_n$. We need
  % to prove that:
  % $$\forall a_i \in \alt['] @ (a_{s+1},\dots,a_n), \lnot (\exists \alt\
  % \subst['], \runCmd{a_i}{\EmptyList}{\EmptySubst}{\alt}{\subst[']})$$
  % 
  % 
  % This holds for the alternatives in $\alt[']$. For any alternative $a_k \in
  % (a_{s+1},\dots,a_n)$, the mutual exclusion hypothesis \mutExcl\ \ctx\ \prog\
  % ensures that no alternative derived from clauses of \pred\ declared
  % chronologically after $c_s$ can unify with the goal $\pred\ \vec{t}$, since
  % every term in input position in $\vec{t}$ is ground. This completes the proof.
\end{proof}

% The main difference between this last definition and \cref{def:mut-excl} is the
% absence of the \coqIn{HG} hypothesis and the usage of the \match procedure
% instead of \unify in the conclusion.

% \begin{theorem}
%   The \elpi input/ouput modes guarantee that for any
%   predicate $p$ whose clauses respect
%   \cref{def:emut-excl}, there exists at most
%   one succeeding clause for any call to $p$.
% \end{theorem}

% \begin{proof}
%   Without loss of generality, we take a program \prog with only binary
%   predicates representing respectively an input and an output. Let $p$ be a
%   predicate in \prog such that all clauses respect
%   \cref{def:mut-excl}. Let
%   ``$p\ t_1\ t_2$'' be a valid call for $p$. Let ``$c_1 :=
%   \clauseCmd{p}{t_1'\ t_2'}{b_1}$'' and ``$c_2 := \clauseCmd{p}{t_1''\
%   t_2''}{b_2}$'' be two clauses implementing $p$. 
%   Note that the absence of
%   groundness check avoid us from saying that $t_1$, which is the input of the
%   call, is a ground term. We reason by induction on the shape of $t_1$ and show
%   that it cannot \match simultaneously with $t_1'$ and $t_1''$, i.e. at most one
%   between $c_1$ and $c_2$ can be applied on the goal.
%   \begin{itemize}
%     \item Case 1: $t_1$ is a constant. A constant, in input position, matches with
%           the same constant or a variable. By the definition of \match
%           $t_1$ only matches with
%           the same constant or a variable.
%           Due to \cref{def:mut-excl}, $t_1'$ and $t_1''$ cannot be neither the
%           constant $t_1$ nor a unification variable nor a combination of the
%           two. Therefore $c_1$ and $c_2$ cannot be applied both of the call to
%           $p$.
%     \item Case 2: $t_1$ is a variable. A variable, in input position, matches
%           only with another variable. 
%           By \cref{def:mut-excl}, $t_1'$ and $t_1''$ cannot be both
%           unification variables. This means that at most one of the two clauses 
%           can be applied on the call to $p$.
%     \item Case 3: $t_1$ is a compond term: a term starting with rigid head with
%           potentially flexible subterms. If the heads of $t_1'$ and $t_1''$ have
%           the same head as $t_1$ then the unification of $t_1$ proceed on the
%           subterms, but, by induction hypothesis, only one between $t_1'$ and
%           $t_1''$ can unify with $t_1$. If the heads of $t_1'$ and $t_1''$ are
%           different then only we are sure that at most one of the two clause
%           can be applied on the call.
%   \end{itemize}
% \end{proof}


% As explained in \cite{1989Warren}, thanks to the (hard-)cut operator, mutual
% exclusiveness can be relaxed.

% \begin{definition}[Mutual exclusiveness with \cut]
%   Two clauses for the same predicate are mutually exclusive if the
%   chronological antecedent has a cut in its body.
%   \label{def:mut-excl-cut}
% \end{definition}

% This ensures that if we reach the cut
% operator in the first clause, the second clause is not considered as a choice
% point. Conversely, if one of the premises before the cut fails, then the second
% clause will be tried. In both situations, the two clauses cannot be applied
% simultaneously to the same predicate call.

% This definition allows overlapping clauses to
% exist in a database under the condition that the antecedent has a cut
% guaranteeing that at most one clause can be applied on a predicate call. 

% \begin{definition}[Mutual-exclusion + \cut]
%   Same as \cref{def:det-prem-cut}
% \end{definition}

% % \paragraph{deterministic clauses after last \cut}
% The second, but no less important, condition for a predicate to be deterministic
% is the following:

% \begin{definition}[deterministic premises after last \cut]
%   In each clause of a deterministic predicate, the premises after the last \cut
%   operator are only calls to deterministic predicates. 
%   \label{def:det-prem-cut}
% \end{definition}

% This guarantees that any output produced is uniquely determined, i.e. no two
% solutions can be produced on the same call.

% \begin{definition}[Deterministic predicate in \elpi]
%   \begin{coqcode}
%     Definition ~\customlabel{edetpred}{\texttt{edet\_pred}}~(~\prog~: prog) (p: pn) :=
%       forall ~$i$~ ~$o$~ ~$a$~
%         (H : ~\runCmd{[\goalCmd{\prog}{\callCmd{p}{i}{o}}{\EmptyList}]}{\EmptyList}{\EmptySubst}{a}{\subst}~), ~$a$~ = ~\EmptyList~.
%   \end{coqcode}
%   \label{def:edt-pred}  
% \end{definition}

% Our definition of deterministic predicate in \elpi (called \ref{edetpred} with a
% leading \coqIn{e} for \elpi) changes from \ref{detpred} in \cref{sec:det}: we do
% not need the \coqIn{HG} hypothesis: the usage of \elpi modes allows to pass any
% (even not ground) term in input position. The derivation rule \ruleCall allows
% makes the difference between terms that should be unified with the \unify or
% the \match procedure at runtime.

% \begin{definition}[Determinacy checking in \elpi]
%   Determinacy checking (noted \coqIn{edet_check}) on a program
%   \prog is equivalent by the combination of 
%   \cref{def:emut-excl,def:mut-excl+cut,def:det-prem}
%   \label{def:det-check}
% \end{definition}

% The following lemma says that in a determinacy-checked program, if \pred is
% a deterministic-annotated program and the \run\ of a call to \pred gives a 
% solution, then the same solution is returned by a run of the same goal
% in a program where all of the clauses of \pred are rewritten such
% that thier last atom is a cut.

% \begin{lemma}
%   Let \tailcut be a function taking a program \prog and predicate \pred
%   returning a new program \prog['] such that the bodies of all clauses of \pred
%   in \prog have been added a \cut\ as last atom.

%   Let \pred be a deterministic-annotated predicate,
%   \begin{coqcode}
%     Lemma det_tail_cut ~\prog \alt~:
%       forall i o a ~\subst \subst[']~ (H: det_check ~\prog\!\!~)
%         (HR: ~\runCmd{[\goalCmd{\prog}{\callCmd{\pred}{i}{o}}{\alt}]}{[]}{\subst}{a}{\subst'}~),
%           ~\runCmd{[\goalCmd{(\tailcutCmd{\prog}{\pred})}{\callCmd{\pred}{i}{o}}{\alt}}{[]}{\subst}{a}{\subst'}~.
%   \end{coqcode}
%   \label{lemma:prog-all-cut}
% \end{lemma}

% \def\clauseL{\ensuremath{\mathcal{L}}\xspace}
% \begin{proof}
%   We reason by induction on \coqIn{HR}: $5$ cases should be taken
%   one per derication rule in \cref{fig:basic-interp}.
%   \begin{itemize}
%     \item Case \ruleStop: cannot be applied since the list of goals is not empty.
%     \item Case \ruleFail: the \texttt{fail} hypothesis tells that no
%           there is no implementation for the predicate \pred, therefore
%           adding a tail-cut to the rules of \pred does not change the behaviour
%           of the program.
%     \item Case \ruleUnif: cannot be applied since the first goal is not a
%           unification or a match.
%     \item Case \ruleBang: same problem as before with the \cut operator.
%     \item Case \ruleCall: there exists at least one rule implementating \pred.  
%           Due to the \tailcut function the list of new goal, together with the
%           alternatives have a \cut\ has their last atom. Moreover, the list
%           of cut-alternatives in each of these atoms is the empty list.
%           By the hypothesis \coqIn{H}, we know that ...
%   \end{itemize}
% \end{proof}

% % Thanks to \cref{th:all-cut}, we can give a common structure to all the clauses
% % of a deterministic predicate: we are free to assume that $c$ always has at
% % least one cut.

% The following lemma says that if all clauses of a predicate \pred have a cut as
% last atom in their body, if the \run\ of a call to this predicate as a goal with
% an empty list of alternatives gives an output, then the alternatives of the
% ouput are the empty list.

% \begin{lemma}[Tail-cut and cut-alternatives]
%   $$
%   \begin{array}{l}
%   \forall \subst\ \subst[']\ i\ o\ p,\\
%   \runCmd{[\goalCmd{(\tailcutCmd{\prog}{\pred})}{\callCmd{p}{i}{o}}{a}]}{[]}{\subst}{x'}{\subst'} \rightarrow [] = x'
%   \end{array}
%   $$
%   \label{lemma:cut-cat-alt}
% \end{lemma}

% \begin{proof}
%   INTUITION: The cut-alternative of the last cut is the empty list
% \end{proof}

% % \begin{corollary}
% %   If all clauses of a predicate have a cut, then at most one of
% %   these clauses can be applied successfully on the goal, the other being
% %   cut away.
% %   \label{cor:only-one-clause}
% % \end{corollary}

% % \begin{proof}
% %   By a slightly modified version of \cite{1989Warren}
% % \end{proof}

% %
% \begin{theorem}[determinacy check $\to$ deterministic pred]
%   For any deterministic-annotated predicate \pred,
%   \begin{coqcode}
%     Theorem det_check_det_pred ~\prog~ (H: edet_check ~\prog\!\!~):
%       ~\ref{edetpred} \prog \pred~.
%   \end{coqcode}
% \end{theorem}

% % Before giving the proof, we just want to point out that using input/output modes
% % of \elpi, no hypothesis on the groundness of terms can be done. 

% \begin{proof}
%   By \cref{lemma:prog-all-cut} and the hypothesis \coqIn{H}, the conclusion can
%   be rewritten such that all clauses of \pred have a cut as last atom in their
%   bodies. Finally, thanks to \cref{lemma:cut-cat-alt} we can conclude the proof.
% \end{proof}

TODO: add the uvar to the terms




\section{Conclusion and related works}

\begin{table}
  \centering
  \begin{tabular}{c|c|c|c|c|c|c}
    Paper         & Mode check & Hard Cut & HO prog         & HHF    & HO unif & SubT.           \\
    \hline
    Warren et al. & Assumed    & \cmark   & \xmark          & \xmark & \xmark  & \xmark          \\
    Mercury       & \cmark     & \xmark   & \nicefrac{1}{2} & \xmark & \xmark  & \nicefrac{1}{2} \\
    Mixtus        & \cmark     & \cmark   & \xmark          & \xmark & \xmark  & \nicefrac{1}{2} \\
    RedAlert      & \cmark     & \cmark   & \xmark          & \xmark & \xmark  & \xmark          \\
    Elpi          & \xmark     & \cmark   & \cmark          & \cmark & \cmark  & \cmark          \\
  \end{tabular}

  \caption{Comparison}
  \label{tab:comparison}
\end{table}


Several studies have explored determinacy analysis in logic programs.
\Cref{tab:comparison} summarizes the main differences between our approach to
determinacy checking and existing work from the literature.

Determinacy analysis can be applied to \textit{open} or \textit{closed}
programs. Closed programs allow determinacy inference by analyzing clauses and
tracking predicate calls throughout the program. In a closed-world setting, it
is sometimes possible to compile a program into a more efficient form using the
\textit{super-homogeneous} transformation, where all clauses of a predicate are
collapsed into a single clause containing a disjunction for each replaced
clause. This approach is central to works such as
\cite{1996Somogy,king2006,1991Sahlin}.

In contrast, a \textit{higher-order} setting introduces the implication
operator, making programs \textit{open} -- i.e., local clauses may be
dynamically loaded. This is the case in Elpi, where a static determinacy checker
is necessary to handle such scenarios. Our operational semantics closely follows
\cite{1990Vink}, interpreting logic programs with the non-logical \cut\
operator. This approach allows us to work with concrete objects, making it
easier to define the meaning of a deterministic predicate. This choice contrasts
with \cite{2011king}, where a denotational semantics is used.

Regarding modes, Elpi's integration with an interactive theorem prover like Rocq
prevents us from relying on ground input terms. However, the non-standard
unification dynamically applied to input terms allows our theorem to dispense
with the requirement of working with a well-moded program -- a fundamental
assumption in works such as \cite{1989Warren,1996Somogy,2011king}.



% SE NON HO MATCH
% \todo{fare P! + well moded}

% SE FACCIAMO PI IMPL:
% Unlike other prolog system, such as \mercury \cite{1996Somogy}, \elpi is not a
% compiled language and we do not transform the list $\mathcal{L}$ of clauses of a
% predicate $p$ in super-homogenous form, i.e. a sole clause containing the
% disjunction of $\mathcal{L}$. This is mainly due to the fact that \elpi is an
% homoiconic language and its program definition can change during the evaluation
% of the code.

% In \cite{1991Sahlin}, and more formally in \cite{1996mogensen}, determinacy is
% used to work with \mixtus, a partial evaluator of \prolog. In that case,
% determinacy is inferred so that it is possible to derive a new specialiezed,
% and therefore more efficient, version of the original program under the
% guarantee that the two program share the same semantics.

% Finally, in \cite{1996henderson}, a determinacy checker for \mercury
% captures different behaviours of a predicate. A predicate can return
% exactly zero and/or one solution, zero and/or multiple solution. In \mercury the
% user is allowed to annotate predicates with determinacy information. A
% non-annotated predicate will be inferred with the lowest derived tag.
% Determinacy, in \mercury, besides giving a guarantee on the program, allows to
% specialized it so that a faster routine can be used in the compiled program.

% Determinacy, as previously mentioned, is the property of a predicate that
% returns at most one solution per call. Such a predicate behaves like a function,
% which is why we refer to it as a deterministic predicate or simply a function.
% Determinism checking statically ensures that the clauses implementing a
% deterministic predicate adhere to this condition. The literature contains
% numerous discussions on this topic, offering various descriptions and
% applications of determinism.

% In \cite{1989Warren}, the authors describe a property subsuming
% determinism: they describe functionality. A predicate is
% functional if it produces at most one \textit{distinct} solution per predicate
% call. The keyword here is \textit{distinct}, since, in the determinacy setting,
% a predicate call producing the same solution twice is not considered as
% deterministic, while, it is functional. %In the paper they explain that mutual
% exclusivness of clauses can be improved not only by looking at the head and at
% the presence of the cut but also by instructing the checker that premises can
% put clauses in mutual exclusivness.

% \todo{look at 1,2,5 of \url{https://citeseerx.ist.psu.edu/document?repid=rep1&type=pdf&doi=b52fc2c62a2f78b8565e96f97dc7cbf6c86b45d4}}

% TABLING cache entry for func only requires 2 cases, ongoing and done with 1 solution



% \subsection{Future work}

% \paragraph{Problem of charging local clauses with weaker conditions}
% Overlapping check che rompe:

% \begin{elpicode}
%   pi x\ (pi Y\ f x Y) => (pi y => f x y) => Bo
% \end{elpicode}

% Dove i modi per f sono input, input.

% In questo esempio se Bo = ``f x y'' allora ci sono due soluzioni,
% in quanto entrambe le regole caricate colla freccia si applicherebbero

%\section{Introduction}

We are interested in the static analsysis of Elpi programs, in particular
in checking their determinacy. Elpi is a higher oreder logic programming
language, a dialect of $\lambda$Prolog~\cite{dale} well suited to manipulate
incomplete syntax trees with binders~\cite{lpar,journal}.
Elpi finds applications as an extension
language for The Rocq\footnote{formerly known as Coq} Interactive prover, where
Elpi has been used for program and proof synthesis~\cite{derive1,derive2,hb} and more recently
as the target language for type class resolution~\cite{coqws,ppdp}.
Type class resolution is typically used to implement overloading~\cite{haskell,ms}
where the solution to a query provides the meaning of the overloaded symbol.
In this context it is of paramount importance that this solution is unique,
i.e. non ambiguous. A static check for this property is of particular
interest in the context of the platform of Rocq libraries: code developed
by different teams is combined together, reused in order to lower the cost
of mechanization.

As of typday Elpi comes with a quite standard type checker~\cite{pf} but
features no mode nor determinacy analisys. The literature provides
many works on the subject~\cite{robadescrittadopo} but none of these
works can be applied to Elpi due to its higher order nature, inherited from
$\lambda$Prolog, and its nonstandard notion of input.

Our contributions is a  determinacy checking algorithm that
\begin{itemize}
\item covers logic programs with cut
\item covers higher order logic programming constructs such as first class predicates and clauses
\item can be applied to pre-existing code bases, i.e. tracks miscalled functions rather than aborting
\end{itemize}
Last but not least we provide an operational semantics for higher order logic programs
with cut that poses the bases for our definitions and paves the way to a
mechanization of the determinacy checking algorithm.

\subsection{Motivating examples}

A first motivating example is the \elpiIn{map} predicate, widely used in
all Elpi applications. The first line gives the signature of \elpiIn{map}:
given a predicate between any types \elpiIn{A} and \elpiIn{B},
it relates a list of \elpiIn{A} with a list of \elpiIn{B}. Elpi
follows the $\lambda$Prolog convention ($\lambda$-0calculus actually) of
writing application with no parentheses, e.g. \elpiIn{map F L R}
can be understood as the atom \elpiIn{map(F, L, R)}.

\begin{elpicode}
pred map i:(pred i:A, o:B), i:list A, o:list B.
map _ [] [].
map F [X|XS] [Y|YS] :- F X Y, map F XS YS.
\end{elpicode}

This code happens to compute a function of the first two arguments, that we
consider inputs, if and only if the higher order predicate \elpiIn{F} is a
function and if the first list is ground. We want to author of this code
to be able to ascribe a more precise signature on the code above, namely:

\begin{elpicode}
func map (func A -> B), list A -> list B.
\end{elpicode}

The syntax \elpiIn{func name? inputs -> outputs} asserts that name is
a function of the inputs (before the arrow) to the outputs if all the
requirements on the inputs are satisfied, in this specific case if the
first input is a functional, binary, predicate. We need this ascription to
be given on an existing code base where \elpiIn{map} is potentially
called by passing a relation as the first argument, say \elpiIn{P}.
In this case the call to \elpiIn{map P L R} has to be accepted but not
coinsidered to be functional for the analisys of the surrounding code.


\begin{elpicode}
func mask! list A, list A -> list bool.
mask! Bad L R :- map (x\y\ if mem! x Bad then y = ff else y = tt) L R. % ok

func mask list A, list A -> list bool. % error
pred mask i:list A, i:list A, o:list bool. % ok
mask Bad L R :- map (x\y\ if mem X Bad then Y = ff else Y = tt) L R.
\end{elpicode}

The cut oeprator is the privileged way to impose functionality on a relation
by committing to its first result.

\begin{elpicode}
func once (pred) -> .
once P :- P, !.

pred mem i:A, i:list A.
mem X [X|_].
mem X [_|XS] :- mem X XS.

func mem! A, list A -> .
mem! X XS :- once (mem X XS).
\end{elpicode}

The signature of \elpiIn{once} states that the higher order argument
\elpiIn{P} can be a predicate, but still \elpiIn{once P} acts as a function.
The determinacy analysis we present tracks functionality from the inputs
to outputs, that can be themselves predicates.

    
\begin{elpicode}
func force (pred) -> (func).
force P (once P).

func foo (pred), list A -> list B.
foo P L R :- force P F, map L F R.
\end{elpicode}

Here the output of \elpiIn{force} is a function \elpiIn{F} and in turn
makes \elpiIn{map L F R} produce a single value for \elpiIn{R} out
of \elpiIn{P} and \elpiIn{L}, making \elpiIn{foo} itself a function.

The higher order term in Elpi (and $\lambda$Prolog) also applies to data
via the so called $\lambda$-tree syntax (also known as HOAS~\cite{dalemechaniz}).
 
\begin{elpicode}
kind tm type.
type app tm -> tm -> tm.
type lam (tm -> tm) -> tm.

func copy tm -> tm.
copy (app A B) (app C D) :- copy A C, copy B D.
copy (lam F) (lam G) :- pi x\ copy x x => copy (F x) (G x).

func whd tm -> tm.
whd (app H A) R :- whd H (lam F), !, pi x\ copy x A => copy (F x) R.
whd X X.
\end{elpicode}

todo explain, can we have the ad hoc rule for pi x?



% We have a rule based language integrated in Coq. Rules are useful
% to model a grown knowledge base (extend existing programs) and manage
% the context in HOAS.

% We want to add to it some of the benefits of functional programming,
% eg statically enforce the absence of global backtracking. Eg twice the same
% rule can turn linear into exponential. Even worse two overlapping rules can
% inadvertently change the meaning when loading two libraries.

% checking or inferring functionality of relations is studied in the literature,
% but does only partially cover HO programming but no HOAS nor homoiconicity, all features
% that are widely used. The former, as in FP, to reuse code via HO iterators,
% eg map. The second for manipulating syntax with binders. The third to
% have programs that extend themselves by synthesizing rules.

% in practice a function is a relation where 1) we identify the arguments
% that are seen as input 2) we prove that the outputs are uniquely determined
% by the inputs. The first part is called mode analysis. We study both
% in the HO setting, eg $\lambda$Prolog, in the dialect of Elpi that has
% a special runtime input mode.

% \section{Motivating examples explained}

% We give a short intro to Elpi and functional analysis with examples
% that cover the use cases we want to cover.

% Convention that capitals are parameters, programs are written in rules
% preceeeded by a signature. prop is the type of predicate, eg code that runs.
% List syntax is bla bla. :- separates the head from the body, the head
% is unified with the goal, then each premise is executed.

% \subsection{Higher order programming}

% A typical HO predicate is map that takes a relation  in A x B
% to a relation in list A x list B.

% \begin{elpicode}
% type map (A -> B -> prop) -> list A -> list B -> prop.
% map _ [] [].
% map F [X|XS] [Y|YS] :- F X Y, map F XS YS.
% \end{elpicode}

% Explain that unlike functional languages, command and expressions are not
% mixed, and prop stuff is executable, hence you don't put F X in place of Y
% but rather run F X Y.

% Also explain that a relation that can go both sides, but
% that this feature is not very useful, for example it does not always
% work and if one calls passing a wrong relation, it is easy to diverge. 

% find a simple example.

% \begin{elpicode}
% filter P [] []
% filter P [X|XS] [X|YS] :- P X, !, filter P XS YS.
% filter P [_|XS] YS :- filter P XS YS.
% \end{elpicode}
  
% We want to annotate with usage info.

% \begin{elpicode}
% pred map i:(pred i:A, o:B), i:list A, o:list B.
% \end{elpicode}

% Explain syntax (pred [name] X, ... = X -> .. -> prop).

% This first step seems to be stringent for little reason, since map can work
% both ways. But it necessary to further refine the annotation with functionality
% assertions.

% \begin{elpicode}
% func map (func A -> B), list A -> list B.
% \end{elpicode}

% Explain syntax (func [name] X, ... -> Y, .. = pred i:X, .. , o:Y, ..).

% Note (map succ) is func list int -> list int (assuming succ is a function).

% \subsection{Higher Order Abstract Syntax programming}

% In HOAS, we want out analysis to accept this code that adds a dynamic rule

% \begin{elpicode}
% kind tm type.
% type app tm -> tm -> tm.
% type lam (tm -> tm) -> tm.

% func copy tm -> tm.
% copy (app F A) (app G B) :- copy F G, copy A B.
% copy (lam F) (lam G) :- pi x\ copy x x => copy (F x) (G x).

% func whd tm -> tm.
% whd (app H A) R :- whd H (lam F), whd (F A) R.
% whd (lam _ as X) X.

% kind ty type.
% type arr ty -> ty -> ty.

% func of tm -> ty.
% of (app F A) T :- of F (arr S T), of A S.
% of (lam F) (arr S T) :- pi x\ of x S => of (F x) T.
% \end{elpicode}

% \subsection{Homoiconicity}

% Since the language is homoiconic we also want this to pass

% \begin{elpicode}
% func comp (func A -> B), (func B -> C) -> (func A -> C).
% comp F G X Y :- F X Tmp, G Tmp Y.

% func fuse (func A -> B) -> (func A -> B).
% fuse (comp (map F) (map G)) (map H) :- fuse (comp F G) H.
% fuse X X.
% \end{elpicode}

% But we don't want to break code that uses map as a relation

% \begin{elpicode}
% func do list (func) -> .
% do [].
% do [P|PS] :- P, do PS.

% pred do-with-trace i:list (pred).
% do-with-trace Code :- map spy Code InstrumentedCode, do Code.
  
% func spy (pred) -> (pred).
% spy P (do [print "before", P, print "after"]).
% \end{elpicode}

% Finally, we want to take advantage of cut across iterators

% \begin{elpicode}
% func once (pred) -> .
% once P :- P, !.

% func do! list (pred) -> .
% do! [] [].
% do! [P|PS] :- once P, do! PS.

% % alternative

% func tcut (pred) -> (func).
% tcut R F :- F = once R.

% func do! list (pred) ->.
% do! LR :- map tcut LR LF, do LF.
% \end{elpicode}
% 
% \section{Classic mode and functional analysis}

% two steps, overlap + func body, with refinement for cut

% mode analysis to deduce groundness and hence sustain overlapping check hypothesis

% \subsection{Peculiarity of Elpi's input mode}

% dynamic effect of input make the ground

% \section{Higher Order programming}

% give the meaning of a type

% \begin{verbatim}
% ty_ := k ty | m:_ -> _ | prop f
% ty := all (ty_ -> ty) | mono ty_
% f  := F | R
% m  := i | o
% tm := tm tm | c
% rule := tm :- tm*
% \end{verbatim}

% \begin{verbatim}

% \end{verbatim}

% \begin{verbatim}
% spec (c : tm) (e : logic) ty : logic := match ty with
%   | prop R => True /\ e
%   | prop F => functional c
%   | data => True
%   | i:l -> r => forall x, spec x True l -> spec (c x) e r
%   | o:l -> r => forall x, -> spec x (spec x True l /\ e) r

% spec map True ... =
% forall c0, (forall c1, forall c2, func (c0 c1 c2)) → forall c1, forall c2, func (map c0 c1 c2) 
% \end{verbatim}

% in other words $func ~ F \to func~ (map~ F)$.

% TODO: how do we do partial application?

% \section{Higher order Abstract syntax}

% \begin{verbatim}
%   ty_ := k ty | m:_ -> _ | prop f
%   ty := all (ty_ -> ty) | mono ty_
%   f  := F | R
%   m  := i | o
%   tm := c | tm tm | x\ tm | tm => tm | pi x\ tm
%   rule := tm :- tm*
%   \end{verbatim}

% \section{Homoiconicity}

% not sure we need to change the syntax.
% what was the example where the skema
% parameters had to carry a mode?
\bibliography{bib}
\end{document}
