\section{Preliminaries}

Before providing a formal definition of functionality and mode analysis, we
first describe the language we are working with, namely a dialect of \lamprolog
called \elpi.

Since we have structured our paper into two main sections: 1) the first
discusses a basic version of \elpi with only Horn clauses and the cut operator,
and 2) the second extends this version by introducing hereditary-Harrop
formulas, we will keep the presentation of the types and functions we need
simple, expanding them further when needed.

Data are represented by terms. For now, a term is an opaque data representing
terms living in the datatype predicate are binary relations and
have type \\\elpiIn{int -> int -> prop}. The two arguments are respectively 
considered as its input and output to better represent mode analysis
(see \cref{sec:modes}).

TODO: dare la definizione di termine: per ora è sufficiente dire che un termine
è un predicate applicato a due interi? Altrimenti potremmo avere un argomenti
di tipo prop che rompono la functionality. + dire che Una variabile è un termine

An atom is either a fully applied predicate or the cut operator, noted
\elpiIn{!}.

TODO: spiegare meglio la unify: quali sono le sue proprietà, ad esempio
transitività.

A variable is a term not yet instantiated. We can assign variables thanks to the
\texttt{unify} procedure. The notation $t_1 = t_2$ is the unification between
the terms $t_1$ and $t_2$. We store the result of variable assignment into
substitutions. A substitution is a mapping from variable name to its assignment
$t$. We considered that the returned assignment is the most dereferenced one,
ie. $t$ is the dereferenced assignement for the variable $X$ in the subsitution
$s$ if $X$ maps to $t'$ in the mapping and all variables in $t'$ 
are dereferenced, if the variable does not appear in the substitution then
the same variable is returned.

By convention variables are indicated with capital letters, whereas predicate
names and constants are indicated with lower case letters.

A clause is made of a predicate name, its two arguments and a body made of a
list of atoms. The list of these atoms are called premises and should be
considered as a list of subgoals in conjunction. By convention a clause is noted
with ``\elpiIn{p i o :- b}'' where p is the predicate name i and o are its
parameters and b is the body. It should be read as follows: if \elpiIn{b} holds
then \elpiIn{p i o} holds. 

A program is a mapping from predicate names to clauses. These clauses are in
disjunction and the order of their visit depend on the chronological order in
which clauses have been declared. We have no builtin \textit{or} operator but
it's behavior can be replicated by adding several implementation of the same
rule. There is no builtin \elpiIn{or} operator, it can be however represented by
adding multiple rules for the same predicate.

Unlike other prolog system, such as \mercury \cite{1996Somogy}, \elpi is not a
compiled language and we do not build super-homogenous clauses for each
predicate. This is mainly due to the fact that \elpi is an homoiconic language
and its program definition can change during the evaluation of the code.

A program is a mapping from predicate name to clauses.

A goal is a triple made of a program \prog, an atom
$\mathcal{A}$ and a list of alternatives $\mathcal{B}$. In particular,
$\mathcal{A}$ is evaluated by looking into the clauses inside \prog. $s$ is the
substitution existing at the creation of the current goal. $\mathcal{B}$
contains the choice-points used by the interpreter during the evaluation of a
cut, the aim of this piece of data will be clarified further in the section
dedicated to the interpreter implementation (see \cref{sec:basic-elpi}).

The interpreter takes a program a query and returns a substitution if the query
is a consequence of the program. Each time a predicate call has multiple clauses
implementing it, it commits the first choice and keep the other as global
alternatives. If that choice leads to a failure, the interpreter will try to execute
the first alternative. This non-deterministic behavior can be controlled by
the user thanks to the cut operator which allows to cut out alternatives (see
\cref{sec:cut}).

To be more precise, we give below the type structure of each cunstruct we have
defined above:

\begin{coqcode}
  Definition tm := Set.
  Definition pn := string. (*predicate names are strings*)
  Definition vn := string. (*variable  names are strings*)
  Inductive atom := 
    | Call (p:pn) (i:tm) (o:tm) : t  (*binary predicates taking integers*)
    | Cut : t                        (*binary predicates taking integers*)
    ...                              (*this is extended in section XX*)
  Inductive clause := Clause (p:pn) (i:tm) (i:tm) (A:list atom) : Clause.
  Notation "P I O :- Bo" := (Clause P I O Bo) (at level 200, right associativity).
  Definition sbst := T.
  Definition prog := pn -> list clause.
  Inductive goal := Goal (P:prog) (a:atom) (A:list alt) -> goal.
  with alt := sbst * list goal.
\end{coqcode}

Below we provide the signature of the intepreter which is expressed by
the \run predicate. Its implementation will be provided thanks to derivation
rules in \cref{sec:basic-elpi} and \cref{sec:full-elpi}.
%
\begin{minted}[autogobble]{coq}
  Inductive run : goal -> list alt -> subst -> list alt -> subst := ...
\end{minted}

This predicate should be read as a relation between a goal \g, a list of
alternative \alt (which are in disjunction with the current goal) and an initial
substitution \subst. These three objects produce a new list of alternatives
\alt['] and a new substitution \subst[']. The idea is that the intepreter stop
when it finds the first solution, which is a valid substitutions for the query
wrt the program. In order to find all the solution of query from the current
program, that is all the valid substutions for the query, it is sufficient to
iterate over all the alternatives \alt['].

\subsection{The cut operator}
\label{sec:cut}

Logic programs are known for their non-deterministic behavior: there could be
several distinct ways to derive a query from a base of knowledge and logic
programs aims to find all of these solutions. 
% We have put quotations around
% the words non-deterministic since of course the interpreter of the prolog system
% has to make some choice about the order in which the rules of the program are
% treated. In our case, we consider that the intepreter takes the rules in a
% chronological way wrt the order in which rules are inserted in the database.

If non-determinism is a key feature, sometimes it is important to let the user
control if and when alternatives have to be rejected. The cut operator is meant
to address this problem.

Since we use the elpi dialect of \lamprolog, it is important to clarify how
the cut behaves in our development. It is in case that each prolog-ish language
has its own cut implementation, for example in the official page of
swi-prolog\footnote{\href{www.swi-prolog.org}{www.swi-prolog.org}} we see that
there are two different kind of cut implementation: the \textit{soft cut} noted with
``\texttt{C *-> T ; E}'' runs E if C has no solution otherwise the result is the
same as running the conjunction of C and T. The \textit{hard cut} noted with the
``\texttt{!}'', "discards all choice points created since entering the predicate
in which the cut appears".

The \mercury system \cite{1996Somogy} is meant to be a pure programming language
extended with types, modes, determinism and module systems. In order to satisfy
its purity, \Mercury only accepts a version of the soft cut.

It is also interesting to see that further representations of cut may exists,
such as the firm cut explained in \cite{2003Andrews} where they provide a
restricted version of the hard cut which has some concistency properties.

\Elpi implementation uses the hard cut definition. Even though the hard cut have
no equivalent representation from pure logic, we are convinced that, from a
programming point of view, it is quite practical to cut away not only the
alternative implementation of a predicate but also to hinder backtracking of all
choice points born from the begin of the clause-body to the current cut
position.

For example, in \elpi, the \textit{if-then-else} construct, which is not a
primitive of the language, can be impelented as follows:
\begin{elpicode}
  if C T E :- C, !, T. % if1
  if _ _ E :- E.       % if2
\end{elpicode}
Due to the hard cut, the query ``\elpiIn{if C T E}'' executes the \texttt{if1} rule.
If \elpiIn{C} has $n > 0$ solutions then only the first is kept, the other being
cut away as the rule \texttt{if2}. The call then succeeds only if \elpiIn{T} has
a solution. The rule \texttt{if2} will be run only if \elpiIn{C} has no
solution.

For example, in the following database:

\begin{elpicode}
  p 1.
  p 2.
  q 2.
\end{elpicode}
The query ``\elpiIn{if (p X) (q X) (0 = 1)}'' fails. Note that the usage of the
soft-cut would make the same query to have a solution, since the backtracking in
the condition would be authorized.

\subsection{Concerning modes}
\label{sec:modes}

Before talking about functionality and the impact that hard-cut has on it. We
prefer to spend some few words about the definition of modes, because as
explained in \cite{1989Warren,1996Somogy} modes are the basis for determinim.

The general idea is that a logic program allows to express relation, but,
sometimes it is also needed to express directed relation where arguments are
differenciated between inputs and output. Said in an other way, output arguments
can be derived from input arguments. It is often the case that the user of a
logic programming langauge tend, in most of cases, to express directed
predicates. The typical example of this need is the \texttt{append} predicate
aiming to concatate two lists.

In a prolog-ish program without modes, the following program:

\begin{elpicode}
append [X|Xs] L [X|L'] :- append Xs L L'. % app1
append [] L L.                            % app2
\end{elpicode}

\noindent
makes the query ``\elpiIn{append Y [] L, 0 = 1}'' to diverge. This is a
classical case where the variable \elpiIn{Y} is unified with a list of infinite
length, the rule \texttt{app1} being continously applied.

Modes are there to avoid this situation. In the literature we find several
papers \cite{1989Warren,1996Somogy} distinguishing two kinds of modes, an input
mode and an output mode. The idea is that each predicate is decorated with a
mode signature. Then the static mode checker verifies that the clauses of the
program are consistent with the mode declrations of the predicates. Essentially,
in the head of a clause, 1) terms appearing in input position are supposed to be
ground whereas 2) terms in output position must become ground in the body of the
clause. In each premise of the body, 3) terms appering in output position become
ground while 4) terms appearing in input position are checked to be ground. If
condition 2) or 4) are not satisfied a mode compilation error is raised.

In \twelf \cite{1996pfenning}, which is an implementation of \lamprolog, mode
checking is extended to work with local nominal variables and higher-order
variable unification within the pattern fragment.

It is interesting to note that mode checking under the groundness condition is
sometime too restrictive. In some situation it is in fact sufficient to work
with term that are "enough" ground to guarantee the wanted properties. As an
example, both \cite{1994apt-oc} and \cite{2002overton}, even though they
researches concern different areas, present a version of this mode checking with
"enough" instantiated terms together with some condition of variable linearty.

In our paper, we will combine all of these ideas, as \elpi is used as a
meta-language for proof verification
\cite{tassi2018,tassi2019,tassi2023,fissore2023,fissore2024}. In particular, we
need to work with a higher-order setup where inputs (and outputs) may not be
fully instantiated: the role of the search engine is to potentially assign
unification variables. In the first part of the paper, mode verification is not
particularly relevant since we present a first-order version of the language.
However, in the second part, we allow predicate arguments to receive not only
objects living in the datatype space but also propositions.

\subsection{Concerning functionality}


Determinacy, as anticipated, is the property of a predicate that returns at most
one solution per call. Determinism checking statically ensures that a predicate
marked as determinate respects this condition with respect to the clauses
implementing it. In the literature, several papers discuss this topic, although
they provide different descriptions and different usages of it.

Determinacy checking (or inference) is statically performed on a program on
predicates annotated (or inferred) as determinate. A predicate is functional if
all of the clauses implmenting it have mutually-exclusive heads, that is they
cannot be applied at the same time. By the way, non-mutually exclusive clauses
$c_0, \dots, c_n$ are accepted if all clauses $c_i$ with $0 \leq i < n$ have
a cut. Thanks to these two conditions we are sure that at most one clause can
be applied successfully per predicate call. The second important condition
for determinacy is that all the premises of a clauses after the last cut are
call to determinate predicate, and, if the clause has no cut all its premises
are call to determinate predicate.

In \cite{1989Warren}, Debray and Warren describe a property subsuming
determinism: they describe functionality. A predicate (clause or query) is
functional if it produces at most one \textit{distinct} solution per predicate
call. The key word here is \textit{distinct}, since, in the determinacy setting,
a predicate call producing the same solution twice is not considered as
determinate, while, it is functional. In the paper they explain that mutual
exclusivness of clauses can be improved not only by looking at the head and at
the presence of the cut but also by instructing the checker that premises can
put clauses in mutual exclusivness.1`'

In \cite{1991Sahlin}, and more formally in \cite{1996mogensen}, determinacy is
used to work with \mixtus, a partial evaluator of \prolog. In that case,
determinacy was inferred so that it was possible to derive a new specialiezed,
and therefore more efficient, version of the original program under the
guarantee that the two program share the same semantics.

Finally, in \cite{1996henderson}, a determinacy checker for \mercury allows to
capture different kind of behaviour of a predicate. A predicate can return
exactly zero and/or one solution, zero and/or multiple solution. In \mercury the
user is allowed to annotate predicates with determinacy information. A
non-annotated predicate will be inferred with the lowest derived tag.
Determinacy, in \mercury, besides giving a guarantee on the program, allows to
specialized it so that a faster routine can be used in the compiled program.

Several work of King et al \cite{king2005, king2006, 2011king} had been pursued
in order to build an inference model for determinacy in \prolog system with cut.
Moreover, they explain that mode inference subsume mode checking. This claim is
quite intuitive, however, we will present a determinacy checker.

Our idea it's up to the user to choose what should be the behavior of a
predicate wrt determinacy. The checker, in fact, assists the user by rejecting,
a predicate, that, for example, is not to detected as determinate. Moreover,
since \elpi is an interpreted language, it seems quite useless a mode inference
engine. This is because local rules can be charged \textit{à la volée} during
the evaluation of the program. 

As explained in some of the previously cited papers, determinacy (and so
functionality) checking is an undecidable problem. The checking property is
sound: if a predicate passes the analysis then it is for sure determinate.
However, it is not complete since there could exists false negatives. We are
convinced, that completeness, is not a big deal, since, as also claimed in ...
the clases of every determinate predicate can be rewritten in a equivalnet form
so that they are no more classified as non-determinate.

\begin{coqcode}
Definition determinate_pred (~\prog~: prog) (p: pn) :=
  forall ~$i$~ ~$o$~ ~$a$~ ~\subst~(H : ~\runCmd{[\goalCmd{\prog}{\callCmd{p}{i}{o}}{\EmptyList}]}{\EmptyList}{\EmptySubst}{a}{\subst}~), ~$a$~ = ~\EmptyList~.
\end{coqcode}

The definition above explains what it means for a predicate to be determinate. 
The idea is that in a given program \prog and a predicate $p$,
if for any call to $p$ with arbitrary input and output terms
starting with the empty substitution and the empty list of alternatives,
we have a solution, i.e. the couple $(a, \subst)$ then the list of
alternatives $a$ is empty. This essentailly means that any functional
predicate produces no choice points.

\subsection{Contributions and paper structure}

In our paper we will provide a description about a new mode and functionality
checkers to verify that clauses are consistent wrt the user annotation. We
introduce these concepts with the \elpi programming language. We start by a
light version of \elpi in the first-order setting, i.e. with no $\lambda$-terms.
Here we provide the derivation system allowing to evaluate a query. Few words
will be spent on the mode checker in this subset of the \elpi language. These
derivation rule, together with the mode checker, are used to prove that any
functional predicate can be rewritten so that all of its rules have at least one
cut by keeping the semantics of the original program. 

In the second part of the paper we will extend the intepreter so that it can
work with higher-order variables, hereditary-arrop clauses insertion and local
$\forall$-quantified variable declaration. Here predicates will also take
propositions as arguments. Thanks to this extention we will propose a new mode
checking algorithm working with a new definition of groundness, called
\textit{input-ground}. The signature of a predicate can be annotated so that
output arguments are guaranteed to be determinate. This will make the
determinacy checker to analyse not only determinate clauses but also relational
one, since a check should be performed wrt the determinacy annotation of its
outputs.