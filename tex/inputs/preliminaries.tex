\section{Preliminaries}

Before providing a formal definition of functionality and mode analysis, we
first describe the language we are working with, namely a dialect of \lamprolog
called \elpi.

Since we have structured our paper into two main sections: 1) the first
discusses a basic version of ELPI with only Horn clauses and the cut operator,
and 2) the second extends this version by introducing hereditary-Harrop
formulas, we will keep the presentation of the types and functions we need
simple, expanding them further when needed.

Data are represented by integers. For now, predicate are binary relations and
have type \\\elpiIn{int -> int -> prop}. The two areguments are respectively 
considered as its input and output (see \cref{sec:modes}).

An atom is either a fully applied predicate or the cut operator, noted
\elpiIn{!}.

A variable is a term not yet instantiated. We can assign variables thanks to the
\texttt{unify} procedure and we note that 2 terms $t_1$ and $t_2$ have to be
unifed with the notation $t_1 = t_2$. We store the result of variable assignment
into substitutions, that are mapping from variable name to their assignment $t$.
We considered that the returned term is the most dereferenced one, ie. $t$ is a
the dereferenced assignement for the variable $X$ in the subsitution $s$ if any
variable appearing as a subterm of $X$ is not assigned in $s$.

By convention variables are indicated with capital letters, whereas predicate
names and constants are indicated with lower case letters.

A clause is made of a predicate name, its two arguments and a body made of a
list of atoms. The list of these atoms are called premises and should be
considered as a list of subgoals in conjunction. By convention a clause is noted
with \elpiIn{p i o :- b} where p is the predicate name i and o are its
parameters and b is the body. It should be read as follows: if \elpiIn{b} holds
then \elpiIn{p i o} holds. 

A program is a mapping from predicate name to clauses. These clauses are in
disjunction and the order of their visit depend on the chronological order in
which clauses have been declared. We have no builtin \textit{or} operator but
it's behavior can be replicated by adding several implementation of the same
rule. There is no builtin \elpiIn{or} operator, it can be however represented by
adding multiple rules for the same predicate.

Unlike other prolog system, such as \mercury \cite{1996Somogy}, \elpi is not a
compiled language and we do not build super-homogenous clauses for each
predicate. This is mainly due to the fact that \elpi is an homoiconic language
and its program definition can change during the evaluation of the code.

A program is a mapping from predicate name to clauses and a goal is an atom and
a query is made by a list of goal. 

The interpreter takes a program a query and returns a substitution if the query
is a consequence of the program. Each time a predicate call has multiple clauses
implementing it, it commits the first choice and keep the other as global
alternatives. If choice leads to a failure, the interpreter will try to execute
of the global alternatives. This non-deterministic behavior can be controlled by
the user thanks to the cut operator which allows to cut out alternatives (see
\cref{sec:cut}).

To be more precise, we give below the type structure of each cunstruct we have
defined above:

\begin{coqcode}
  Definition pn := string. (*predicate names are strings*)
  Definition vn := string. (*variable  names are strings*)
  Inductive atom := 
    | PN pn nat nat : t. (*binary predicates taking integers*)
    | CUT : t.           (*binary predicates taking integers*)
    ...                  (*this is extended in section XX*)
  Definition clause := pname * int * int * list (atom).
  Definition sbst := vname -> atom.
  Definition prog := pname -> list clause.
  Definition goal := prog * atom * list alt
  with alt := sbst * list goal.
\end{coqcode}

Since we have to make backtracking and cut change the 

A goal is a triple made of a program \prog, a substitution $s$, an atom
$\mathcal{A}$ and a list of alternatives $\mathcal{B}$. In particular,
$\mathcal{A}$ is evaluated by looking into the clauses inside \prog. $s$ is the
substitution existing at the creation of the current goal. $\mathcal{B}$
contains the choice-points used by the interpreter during the evaluation of a
cut, the aim of this piece of data will be clarified further in the section
dedicated to the interpreter implementation.


Prolog interpreter has a \texttt{run} procedure relating a program, a goal, a list of 
alternatives to a new list of alternative and a substition

\begin{minted}[autogobble]{coq}
  Inductive run : prog -> goal -> list alt -> subst -> list alt :=
    ...

  Definition functional_goal (p: prog) :=
    forall g s, run p g [::] s -> s = [::].
\end{minted}


\subsection{The role of the cut}
\label{sec:cut}

Here we give a definition of our cut: \cite{2003Andrews}

Logic programs are known for their non-deterministic behavior. The could be
several distinct ways to derive a query from a base of knowledge and logic
programs aims to find these all of this solution. 
% We have put quotations around
% the words non-deterministic since of course the interpreter of the prolog system
% has to make some choice about the order in which the rules of the program are
% treated. In our case, we consider that the intepreter takes the rules in a
% chronological way wrt the order in which rules are inserted in the database.

If non-determinism is a key feature, sometimes it is important to let the user
control if and when alternatives have to be rejected. The cut operator is meant
to address this problem. It is often used, for example, to provide an
implementation for the \textit{if-then-else} construct. 

Since we use the elpi dialect of lambda-prolog, it is important to clarify how
the cut behaves in our development. It is in case that each prolog-ish language
has its own cut implementation, for example in the official page of
swi-prolog\footnote{\href{www.swi-prolog.org}{www.swi-prolog.org}} we see that
there are two different kind of cut implementation: the `soft cut' noted with
``\texttt{A *-> B ; C}'' runs C if A has no solution otherwise the result is the
same as running the conjunction of A and B. The `hard cut' noted with the
``\texttt{!}'', "discards all choice points created since entering the predicate
in which the cut appears".

In mercury \cite{1996Somogy}\dots Dire anche che molti usano la super-homogenous
forms invece che regole distinte ma per noi non va bene visto che il programma
può essere esteso a piacimento

It is also interesting to see that further representations of cut may exists,
such as the firm cut explained in \cite{2003Andrews} where they provide a
restricted version of the hard cut which has some concistency properties.

Elpi implementation uses the hard cut definition. Even though the hard cut have
no equivalent representation from pure logic, we are convinced that from a more
more programming point of view it is quite practical to cut away not only the
alternative implementation of a predicate but also to hinder backtracking of all
choice points born from the begin of the clause-body to the current cut
position.

\subsection{Functionality in the literature}

\cite{1989Warren,1987vanroy}

\cite{1991Sahlin,1996mogensen}

\cite{king2005,king2006,2011king}

\cite{1996henderson} Mercury

\subsection{Modes in the literature}
\label{sec:modes}

\cite{2002overton} -> Mercury

\subsection{Summary}

\cref{tab:comparison}

\begin{table}
  \centering
  \begin{tabular}{c|c|c|c|c|c}
    Paper   & Mode check & Determincy check & Cut        & Harrop     & HO unif   \\
    \hline
    Mercury & Compile    & \checkmark       & \xmark     & \xmark     & \xmark    \\
    Mixtus  & Compile    & \checkmark       & \checkmark & \xmark     & \xmark    \\
    Elpi    & Runtime    & \checkmark       & \checkmark & \checkmark & Partially \\
  \end{tabular}

  \caption{Comparison}
  \label{tab:comparison}
\end{table}

