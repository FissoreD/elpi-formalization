
% \section{Classic mode and functional analysis}

% two steps, overlap + func body, with refinement for cut

% mode analysis to deduce groundness and hence sustain overlapping check hypothesis

% \subsection{Peculiarity of Elpi's input mode}

% dynamic effect of input make the ground

% \section{Higher Order programming}

% give the meaning of a type

% \begin{verbatim}
% ty_ := k ty | m:_ -> _ | prop f
% ty := all (ty_ -> ty) | mono ty_
% f  := F | R
% m  := i | o
% tm := tm tm | c
% rule := tm :- tm*
% \end{verbatim}

% \begin{verbatim}

% \end{verbatim}

% \begin{verbatim}
% spec (c : tm) (e : logic) ty : logic := match ty with
%   | prop R => True /\ e
%   | prop F => functional c
%   | data => True
%   | i:l -> r => forall x, spec x True l -> spec (c x) e r
%   | o:l -> r => forall x, -> spec x (spec x True l /\ e) r

% spec map True ... =
% forall c0, (forall c1, forall c2, func (c0 c1 c2)) → forall c1, forall c2, func (map c0 c1 c2) 
% \end{verbatim}

% in other words $func ~ F \to func~ (map~ F)$.

% TODO: how do we do partial application?

% \section{Higher order Abstract syntax}

% \begin{verbatim}
%   ty_ := k ty | m:_ -> _ | prop f
%   ty := all (ty_ -> ty) | mono ty_
%   f  := F | R
%   m  := i | o
%   tm := c | tm tm | x\ tm | tm => tm | pi x\ tm
%   rule := tm :- tm*
%   \end{verbatim}

% \section{Homoiconicity}

% not sure we need to change the syntax.
% what was the example where the skema
% parameters had to carry a mode?