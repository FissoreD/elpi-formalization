\section{Determinacy analysis of Horn Clauses with input matching}
\label{sec:basic-elpi}

As anticipated in the end of \cref{sec:modes}, the hypothesis that input
argument are ground, such as in \cref{th:det} is a too strong condition for our
use case. In automatic proof search, the input arguments may contain flexible
terms. They will actually assigned by the search engine. Similarly, proof search
may produce non-ground output terms. This means that \cref{def:well-moded}
should be relaxed.

In our first-order prolog system we work with the same data structures described
in the previous section, however, we change the unification algorithm wrt the
mode of the terms that are being unified.

\subsection{Modes and \match}
\label{sec:elpi-modes}

% Mode checking in this setting is performed using the classical groundness
% verification where term groundness is derived from their flow in the body of a
% clause. We start with the hypothesis that a term in input position in the head
% of a clause is ground. The premises in the body representing calls to predicates
% must be called with ground input terms, while output are supposed to become
% ground. At the end of the body analyse, we check that the output terms in the
% head of the clause has become ground. If this is not the case, or if a call in
% the body is done with a non-ground term in input position a mode-checking error
% is raised.

The modes we authorize in this setting are two, even though we call them
the input and the output
mode, ther interpretation we attach to them is quite different. 
% As explained at the very end of \cref{sec:modes}, we want predicates
% to be called with non-ground input terms, therefore
% the meaning we attach to our modes is quite particuar wrt what
% can be found in the literature.

We start from the concept of terms read and write as
sketched in \cite{1991ait-wam}. Essentially, in a call, a term in input position is an
object received in read mode. This means that, if the entry
term is a variable and the corresponding term in the head of the clause is
rigid, unification fails: we cannot write in input in the head of a clause.

On the other hand, it is always possible to unify a non-flexible input term with
a variable if that variable appears in the head of the clause: we are consuming
the pieces of information of the input to select the clause in the program. 

% Finally, no
% constraints are imposed on output terms: they are used to produce information
% and therefore they can be written. In a call an output is not forced to be a
% variable to be instantiated, but also a compound term.

We refer to this special type of unification as \match. Similar to the \unify
procedure, we use the notation \matchCmd{t_1}{t_2}{\subst}{\subst[']} to denote
the \match\!ing between two terms $t_1$ and $t_2$, transforming, if
possibile, an initial substitution \subst into a final substitution \subst['].

\def\vars{\texttt{vars}}

\begin{definition}[The \match procedure]
  Let \coqIn{vars(t)} be a function that returns the variables in a term
  \coqIn{t}. Then \match is defined as follows:
  \vspace{-1em}
  $$\forall\ t_1\ t_2\ \subst\ \subst['], \unifyCmd{t_1}{t_2}{\subst}{\subst[']} \land \vars\ (\sigma t_1) = \vars\ (\subst['] t_1)$$
  \vspace{-2em}
  % \begin{coqcode}
  %   Definition ~\matchCmd{t_1}{t_2}{\subst}{\subst[']}~ :=
  %     ~\unifyCmd{t_1}{t_2}{\subst}{\subst[']}~ /\ 
  %       ~$\forall$~ v, v \in vars(~$t_1$~) -> is_var ~(\subst['] $t_1$)~
  % \end{coqcode}
\end{definition}

As a mean of example, consider following program.

\begin{elpicode}
  pred p i:int, o:int.
  p 1 1.          % p1
  p X 2 :- X = 7. % pX
\end{elpicode}


The predicate \elpiIn{p} is decorated with a type\&mode signature. It tells not
only that the first two arguments should be of type \elpiIn{int}, but also that
the first argument is an input (\modeAlone{i}) and the second an output
(\modeAlone{o}). This means that the query \elpiIn{p Y Z} succeed exactly once:
the rule \elpiIn{p1} could not be applied, since the variable \elpiIn{Y} is in
input (i.e. read mode) and cannot be instantiated to $1$ in the head of the
clause. Note, that the application of \elpiIn{pX} on the query succeed and
producing the substitution $\subst = \{Y \mapsto 7, Z \mapsto 2\}$: \elpiIn{Y}
can be instantiated in the body of a clause. Moreover, the query
\elpiIn{p 1 Z} has one solution with $\subst = \{Z \mapsto 1\}$ due to the
application of \elpiIn{p1}. Note that the rule \elpiIn{pX} is a valid choice
point for the query: the head unifies with the query assigning the local
$\exists$variable \elpiIn{X} to $1$, but a failure will occur in the body
while unifying the value of \elpiIn{X} and \elpiIn{7}.

\begin{corollary}[\match is not reflexive]
  The \match operator is not reflexive:
\end{corollary}
\begin{proof}
  A counter-example: \matchCmd{1}{X}{\EmptySubst}{\{X \mapsto 1\}},
  but \matchCmd{X}{1}{\EmptySubst}{\text{\xmark}}.
\end{proof}

\begin{lemma}[\match: unification of right terms]
  Given three terms $t$, $t_1$, and $t_2$:
  $$\forall\ \subst_1\ \subst_2, 
    \matchCmd{t}{t_1}{\EmptySubst}{\subst_1} \to 
    \matchCmd{t}{t_2}{\EmptySubst}{\subst_2} \to 
    \subst_1\ t = \subst_2\ t $$
  \vspace{-2em}
  \label{th:match-right}
\end{lemma}

\begin{proof}
  By hypothesis, $\subst_1\ t = \subst_1\ h_1$ and $\subst_2\ t = \subst_2\
  h_2$. Due to the definition of \match and since $t$ is the term passed in
  input in both \match, no variables in $t$ is assigned in neither
  $\subst_1$ nor $\subst_2$, therefore $\subst_1 t = t$ and $\subst_2 t = t$.
  Now, after some rewriting, we have $\subst_1\ h_1 = \subst_2\ h_2$, which 
  concludes the proof.
\end{proof}

% Moreover, \elpi provides a way to further control the unification on input
% arguments. The user can put the \uvar keyword in front of a variable name in a
% input argument. This tells the unification engine to unify this head term with
% only flexible term. For example, we can add the rule ``\elpiIn{pUX} := \elpiIn{p
% (uvar X) 2 :- X = 7}'' to the previous database. The \uvar keyword does not
% allow the call \elpiIn{p 1 Z} to be unified with \elpiIn{pUX} since \elpiIn{1}
% is not a variable.

% We think that the combination of the input mode and the \uvar keyword give
% the user a powerful tool to  control how unification is dynamically performed
% at runtime.

We add the new term to the language:
% \begin{coqcode}
%   | UnifyM (t1:tm) (t2:tm) : atom
% \end{coqcode}
$$tm ::= ... \mid tm =_m tm$$

The introduction of dynamic modes requires modifying the interpreter so that the
\match procedure is called when unification is performed on input terms. To
achieve this, we need to slightly adjust the $\mathcal{F}$ function used by the
\ref{rule:call}, as follows:\todo{riscrivere con indici diversi}%
% $$
% \mathcal{F}(\prog, \pred, i, o, \subst, \alt) := 
%   [(\subst, (
%       \Cons{(\prog, i =_m i', \alt)}
%         \Cons{(\prog, o = o', \alt)}
%           {[(\prog, g, \alt) \mid g \in bs]})) \mid \clauseCmd{\pred}{i'\ o'}{bs} \in \prog\ \pred]
% $$
%
\begin{align*}
  \mathcal{F}(\prog, \pred, \vec{t}, \subst, \alt) :=
  \Bigg[\bigg(\subst,
    \Big(
    &\big[(\prog, (\vec{t}_i)_x =_m (\vec{u}_i)_x, \alt) \mid 0 \leq x < \text{len }\vec{t}_i\big] @ \\
    &\quad\big[(\prog, (\vec{t}_o)_x = (\vec{u}_o)_x, \alt) \mid 0 \leq x < \text{len }\vec{t}_o\big] @\\
    &\qquad\big[(\prog, g, \alt) \mid g \in bs\big]
    \Big)\bigg) \mid \clauseCmd{\pred}{\vec{u}}{bs} \in \prog\ \pred\Bigg]
\end{align*}

The main difference is that the usage of the $=_m$ to indicate that the
unification between input arguments in the goal and in the head of the clause
should be performed using the \match procedure.

Finally the derivation system in \cref{fig:basic-interp} is extended with the
rule in \cref{fig:interp-match}.

\begin{figure}
  
  \ruleMatchM{1}
  
  \caption{Dynamic semantics: rule for \match}
  \label{fig:interp-match}
\end{figure}


% \textcolor{red}{\textbf{TODO}}: add the derivation rule for the $=_m$ atom
% Recall:

% The \textit{unify rule} (\ruleUnif) is applied when a unification $t_1 =_m t_2$
% occurs in the head of the current goal list. Depending on $m$, which can be
% either $i$ or $o$, it calls respectively the \unify or the \match (see
% \cref{sec:elpi-modes}) algorithm between the two terms $t_1$ and $t_2$ from the
% substitution \subst and returns the updated substitution \subst[']. Finally the
% \run\ procedure is called the list of remaining goals \g and the new
% substitution \subst['].

\subsection{Determinacy checker for prolog with \match}

Mode checking is an essential ingredient for ensuring determinacy.

% Before explaining the interaction between these two, we need to take
% some time to talk about mutual clauses exclusiveness.
% %
% % \paragraph{Mutual exclusive clauses}
% From \cite{1989Warren}, we know that at most one clause can be applied
% successfully for any deterministic predicate.

% In order to satisfy this (necessary by not sufficient) condition, we need 
% that all the clauses of a deterministic predicate $p$ are mutually exclusive.
% Mutual exclusiveness can be stated as follows:

% \begin{definition}[Non-overlapping clauses]
%   For any pair of clauses of the same predicate ``\clauseCmd{p}{\vec{x}}{b1}''
%   and ``\clauseCmd{p}{\vec{y}}{b2}'', we say that they are non-overlapping if
%   there exists an input position $i$ such that $\forall \sigma, \sigma\ x_i \neq
%   \sigma\ y_i$
%   \label{def:mut-excl}
% \end{definition}

\Cref{th:det} ensures that at most one clause can be applied to any given
predicate, provided that the input terms are ground. However, since we no longer
perform mode-checking and our inputs are not necessarily ground, the theorem is
no longer valid.
%
% Thanks to mode checking, any call is validated only if its terms in input
% positions are ground. Groundness % (modulo \cut, see \cref{def:mut-excl+cut})
% guarantees that at most one clause can be executed successfully for a given
% call.
%
% We also emphasize that mode checking ensures that output terms become ground.
% This is a fundamental property; otherwise, outputs would be meaningless—if an
% output does not become ground, it cannot serve as the input for another
% predicate call.
%
% In our setting, we slightly extend this definition so that the \elpi's \uvar
% keyword is taken into account.
%
% \begin{definition}[Mutual exclusiveness with \uvar]
%   A clause with an input term marked with the \elpiIn{uvar} keyword,
%   does not overlap with any other rigid-head term.
%   \label{def:mut-excl-uvar}
% \end{definition}
%
% This means that a term marked with the \uvar keyword in the head of a clause
% overlaps only with unification variables or with another term marked with
% \uvar.
%
% In our first-order \elpi, we do not perform any static mode analysis. Instead,
% we rely on the \match operation, which is dynamically applied to input arguments at runtime.

We start therefore to state a new version of
\cref{th:mut-excl-head,th:mut-excl}
%BBB

\begin{lemma}[Mutually-exclusive heads with \match property]
  Given a context \ctx and two clauses $c_1 = \clauseCmd{\predVar}{\vec{t}}{b_1}$ and $c_2 = \clauseCmd{\predVar}{\vec{u}}{b_2}$ then
  let $\vec{t}_i$ and $\vec{u}_i$ be the inputs arguments of $\vec{t}$ and $\vec u$ respectively.
  % forall vector of \textit{ground} terms $\vec{v}$,
  $$\mutExclHeads\ \ctx\ c_1\ c_2 \to \forall \subst_1\ \subst_2, \lnot (\exists \vec{v},
    \matchCmd[1]{v}{t_i}{\EmptySubst}{\subst_1} \land
    \matchCmd[1]{v}{u_i}{\EmptySubst}{\subst_2})$$
  \vspace{-2em}
  \label{th:mut-excl-head1}
\end{lemma}

\begin{proof}
  By contradiction, let $t$ be a term matching with both $h_1$ and $h_2$. By
  \cref{th:match-right}, we have that $\subst_1 h_1 = \subst_2 h_2$, this
  implies that $(\subst_1 \cup \subst_2) h_1 = (\subst_1 \cup \subst_2) h_2$\todo{prove this}. By
  the hypothesis $\mutExclHeads\ \ctx\ c_1\ c_2$, and by \cref{def:mut-excl}, it does
  not exists a substitution allowing to unify the head of the two clause. This
  is a contradiction.
\end{proof}

\begin{lemma}[Mutually-exclusion with \match property]
  Given a context \ctx\ and a program \prog, if $\mutExcl\ \ctx\ \prog$ then for
  any call $\pred\ \vec{t}$, it exists at most one clause that can be
  successfully applied on $\pred\ \vec{t}$. 
  \label{th:mut-excl1}
\end{lemma}

\begin{proof}
  By \cref{th:mut-excl-head1,th:mut-excl+cut}
\end{proof}

\begin{theorem}[Determinacy checking with \match property]
  Given a context \ctx\ and a program \prog, forall deterministic predicate \pred in \ctx,
  we have
  $$\hyperref[def:det-check]{\detCheck}\ \ctx\ \prog \to \forall \vec{t}, \hyperref[def:is-det]{\isdet}\ \prog\ (\pred\ \vec{t})$$
  \vspace{-2em}
  \label{th:det1}
\end{theorem}

\begin{proof}
  We proceed by induction on the derivations in
  \cref{fig:basic-interp,fig:interp-match}. The structure of the proof closely
  follows the one in \cref{th:det}, with a few notable differences. In
  particular, this proof does not rely on the \wellModed\ hypothesis for the
  program, nor does it assume the groundness of input arguments in the call to
  \pred. The crucial insight lies in the use of the \match\ procedure, combined
  with the result from \cref{th:mut-excl-head1}, which strengthens the proof.
  %The
  % most significant case to consider is \ref{rule:call}. Let $\vec{c}$ denote the
  % result of $\prog\ \pred$. The function $\mathcal{F}$ produces a list of
  % alternatives $\alt$, where each $a_i \in \alt$ consists of the unification
  % between each term in $\vec{t}$ and the corresponding terms in the head of $c_i$,
  % followed by the body of $c_i$.
  % 
  % Let $\alt = a_1, \dots, a_s, \dots, a_n$, where $a_s$ is the first alternative that
  % can be successfully applied to the initial goal. We distinguish between two
  % cases:  
  % 1) $a_s$ contains a \cut.
  % 2) $a_s$ does not contain a \cut.
  % 
  % In the first case, by hypothesis $\detPrem\ \ctx\ \prog$, the alternative
  % $a_s$ has the form $b_1, \dots, b_x, \dots, b_m$, where $b_x = \cut$, and for
  % all $b_j \in b_{x+1},\dots,b_m$, $b_j$ is a call to a deterministic predicate.
  % The presence of the \cut\ discards the alternatives $a_{s+1},\dots,a_n$ as
  % well as all choice points generated by the execution of the goals
  % $b_1,\dots,b_{x-1}$. By the induction hypothesis, the execution of the goals
  % $b_{x+1},\dots,b_m$ produces alternatives with no solution. Consequently, the
  % goal is proved in this case.
  % 
  % In the second case, where $a_s$ does not contain a \cut, the hypothesis
  % guarantees that all goals in $c_i$ are calls to deterministic predicates. By
  % the induction hypothesis, the execution of these goals produces alternatives
  % $\alt[']$ with no solution. The final list of alternatives returned by the
  % call to \run is the concatenation of $\alt[']$ and $a_{s+1},\dots,a_n$. We need
  % to prove that:
  % $$\forall a_i \in \alt['] @ (a_{s+1},\dots,a_n), \lnot (\exists \alt\
  % \subst['], \runCmd{a_i}{\EmptyList}{\EmptySubst}{\alt}{\subst[']})$$
  % 
  % 
  % This holds for the alternatives in $\alt[']$. For any alternative $a_k \in
  % (a_{s+1},\dots,a_n)$, the mutual exclusion hypothesis \mutExcl\ \ctx\ \prog\
  % ensures that no alternative derived from clauses of \pred\ declared
  % chronologically after $c_s$ can unify with the goal $\pred\ \vec{t}$, since
  % every term in input position in $\vec{t}$ is ground. This completes the proof.
\end{proof}

% The main difference between this last definition and \cref{def:mut-excl} is the
% absence of the \coqIn{HG} hypothesis and the usage of the \match procedure
% instead of \unify in the conclusion.

% \begin{theorem}
%   The \elpi input/ouput modes guarantee that for any
%   predicate $p$ whose clauses respect
%   \cref{def:emut-excl}, there exists at most
%   one succeeding clause for any call to $p$.
% \end{theorem}

% \begin{proof}
%   Without loss of generality, we take a program \prog with only binary
%   predicates representing respectively an input and an output. Let $p$ be a
%   predicate in \prog such that all clauses respect
%   \cref{def:mut-excl}. Let
%   ``$p\ t_1\ t_2$'' be a valid call for $p$. Let ``$c_1 :=
%   \clauseCmd{p}{t_1'\ t_2'}{b_1}$'' and ``$c_2 := \clauseCmd{p}{t_1''\
%   t_2''}{b_2}$'' be two clauses implementing $p$. 
%   Note that the absence of
%   groundness check avoid us from saying that $t_1$, which is the input of the
%   call, is a ground term. We reason by induction on the shape of $t_1$ and show
%   that it cannot \match simultaneously with $t_1'$ and $t_1''$, i.e. at most one
%   between $c_1$ and $c_2$ can be applied on the goal.
%   \begin{itemize}
%     \item Case 1: $t_1$ is a constant. A constant, in input position, matches with
%           the same constant or a variable. By the definition of \match
%           $t_1$ only matches with
%           the same constant or a variable.
%           Due to \cref{def:mut-excl}, $t_1'$ and $t_1''$ cannot be neither the
%           constant $t_1$ nor a unification variable nor a combination of the
%           two. Therefore $c_1$ and $c_2$ cannot be applied both of the call to
%           $p$.
%     \item Case 2: $t_1$ is a variable. A variable, in input position, matches
%           only with another variable. 
%           By \cref{def:mut-excl}, $t_1'$ and $t_1''$ cannot be both
%           unification variables. This means that at most one of the two clauses 
%           can be applied on the call to $p$.
%     \item Case 3: $t_1$ is a compond term: a term starting with rigid head with
%           potentially flexible subterms. If the heads of $t_1'$ and $t_1''$ have
%           the same head as $t_1$ then the unification of $t_1$ proceed on the
%           subterms, but, by induction hypothesis, only one between $t_1'$ and
%           $t_1''$ can unify with $t_1$. If the heads of $t_1'$ and $t_1''$ are
%           different then only we are sure that at most one of the two clause
%           can be applied on the call.
%   \end{itemize}
% \end{proof}


% As explained in \cite{1989Warren}, thanks to the (hard-)cut operator, mutual
% exclusiveness can be relaxed.

% \begin{definition}[Mutual exclusiveness with \cut]
%   Two clauses for the same predicate are mutually exclusive if the
%   chronological antecedent has a cut in its body.
%   \label{def:mut-excl-cut}
% \end{definition}

% This ensures that if we reach the cut
% operator in the first clause, the second clause is not considered as a choice
% point. Conversely, if one of the premises before the cut fails, then the second
% clause will be tried. In both situations, the two clauses cannot be applied
% simultaneously to the same predicate call.

% This definition allows overlapping clauses to
% exist in a database under the condition that the antecedent has a cut
% guaranteeing that at most one clause can be applied on a predicate call. 

% \begin{definition}[Mutual-exclusion + \cut]
%   Same as \cref{def:det-prem-cut}
% \end{definition}

% % \paragraph{deterministic clauses after last \cut}
% The second, but no less important, condition for a predicate to be deterministic
% is the following:

% \begin{definition}[deterministic premises after last \cut]
%   In each clause of a deterministic predicate, the premises after the last \cut
%   operator are only calls to deterministic predicates. 
%   \label{def:det-prem-cut}
% \end{definition}

% This guarantees that any output produced is uniquely determined, i.e. no two
% solutions can be produced on the same call.

% \begin{definition}[Deterministic predicate in \elpi]
%   \begin{coqcode}
%     Definition ~\customlabel{edetpred}{\texttt{edet\_pred}}~(~\prog~: prog) (p: pn) :=
%       forall ~$i$~ ~$o$~ ~$a$~
%         (H : ~\runCmd{[\goalCmd{\prog}{\callCmd{p}{i}{o}}{\EmptyList}]}{\EmptyList}{\EmptySubst}{a}{\subst}~), ~$a$~ = ~\EmptyList~.
%   \end{coqcode}
%   \label{def:edt-pred}  
% \end{definition}

% Our definition of deterministic predicate in \elpi (called \ref{edetpred} with a
% leading \coqIn{e} for \elpi) changes from \ref{detpred} in \cref{sec:det}: we do
% not need the \coqIn{HG} hypothesis: the usage of \elpi modes allows to pass any
% (even not ground) term in input position. The derivation rule \ruleCall allows
% makes the difference between terms that should be unified with the \unify or
% the \match procedure at runtime.

% \begin{definition}[Determinacy checking in \elpi]
%   Determinacy checking (noted \coqIn{edet_check}) on a program
%   \prog is equivalent by the combination of 
%   \cref{def:emut-excl,def:mut-excl+cut,def:det-prem}
%   \label{def:det-check}
% \end{definition}

% The following lemma says that in a determinacy-checked program, if \pred is
% a deterministic-annotated program and the \run\ of a call to \pred gives a 
% solution, then the same solution is returned by a run of the same goal
% in a program where all of the clauses of \pred are rewritten such
% that thier last atom is a cut.

% \begin{lemma}
%   Let \tailcut be a function taking a program \prog and predicate \pred
%   returning a new program \prog['] such that the bodies of all clauses of \pred
%   in \prog have been added a \cut\ as last atom.

%   Let \pred be a deterministic-annotated predicate,
%   \begin{coqcode}
%     Lemma det_tail_cut ~\prog \alt~:
%       forall i o a ~\subst \subst[']~ (H: det_check ~\prog\!\!~)
%         (HR: ~\runCmd{[\goalCmd{\prog}{\callCmd{\pred}{i}{o}}{\alt}]}{[]}{\subst}{a}{\subst'}~),
%           ~\runCmd{[\goalCmd{(\tailcutCmd{\prog}{\pred})}{\callCmd{\pred}{i}{o}}{\alt}}{[]}{\subst}{a}{\subst'}~.
%   \end{coqcode}
%   \label{lemma:prog-all-cut}
% \end{lemma}

% \def\clauseL{\ensuremath{\mathcal{L}}\xspace}
% \begin{proof}
%   We reason by induction on \coqIn{HR}: $5$ cases should be taken
%   one per derication rule in \cref{fig:basic-interp}.
%   \begin{itemize}
%     \item Case \ruleStop: cannot be applied since the list of goals is not empty.
%     \item Case \ruleFail: the \texttt{fail} hypothesis tells that no
%           there is no implementation for the predicate \pred, therefore
%           adding a tail-cut to the rules of \pred does not change the behaviour
%           of the program.
%     \item Case \ruleUnif: cannot be applied since the first goal is not a
%           unification or a match.
%     \item Case \ruleBang: same problem as before with the \cut operator.
%     \item Case \ruleCall: there exists at least one rule implementating \pred.  
%           Due to the \tailcut function the list of new goal, together with the
%           alternatives have a \cut\ has their last atom. Moreover, the list
%           of cut-alternatives in each of these atoms is the empty list.
%           By the hypothesis \coqIn{H}, we know that ...
%   \end{itemize}
% \end{proof}

% % Thanks to \cref{th:all-cut}, we can give a common structure to all the clauses
% % of a deterministic predicate: we are free to assume that $c$ always has at
% % least one cut.

% The following lemma says that if all clauses of a predicate \pred have a cut as
% last atom in their body, if the \run\ of a call to this predicate as a goal with
% an empty list of alternatives gives an output, then the alternatives of the
% ouput are the empty list.

% \begin{lemma}[Tail-cut and cut-alternatives]
%   $$
%   \begin{array}{l}
%   \forall \subst\ \subst[']\ i\ o\ p,\\
%   \runCmd{[\goalCmd{(\tailcutCmd{\prog}{\pred})}{\callCmd{p}{i}{o}}{a}]}{[]}{\subst}{x'}{\subst'} \rightarrow [] = x'
%   \end{array}
%   $$
%   \label{lemma:cut-cat-alt}
% \end{lemma}

% \begin{proof}
%   INTUITION: The cut-alternative of the last cut is the empty list
% \end{proof}

% % \begin{corollary}
% %   If all clauses of a predicate have a cut, then at most one of
% %   these clauses can be applied successfully on the goal, the other being
% %   cut away.
% %   \label{cor:only-one-clause}
% % \end{corollary}

% % \begin{proof}
% %   By a slightly modified version of \cite{1989Warren}
% % \end{proof}

% %
% \begin{theorem}[determinacy check $\to$ deterministic pred]
%   For any deterministic-annotated predicate \pred,
%   \begin{coqcode}
%     Theorem det_check_det_pred ~\prog~ (H: edet_check ~\prog\!\!~):
%       ~\ref{edetpred} \prog \pred~.
%   \end{coqcode}
% \end{theorem}

% % Before giving the proof, we just want to point out that using input/output modes
% % of \elpi, no hypothesis on the groundness of terms can be done. 

% \begin{proof}
%   By \cref{lemma:prog-all-cut} and the hypothesis \coqIn{H}, the conclusion can
%   be rewritten such that all clauses of \pred have a cut as last atom in their
%   bodies. Finally, thanks to \cref{lemma:cut-cat-alt} we can conclude the proof.
% \end{proof}

TODO: add the uvar to the terms



