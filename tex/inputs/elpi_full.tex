\section{Full elpi}
\label{sec:full-elpi}

The full potential of \elpi comes with the fact that the language accepts
hareditary-harrop formulas and higher-order pattern unification
(\cite{1991miller-pf}), noted \pf. This extension allows program to be dynamically modified
by the insertion of local clauses, moreover, local universally quantified
terms can be quantified within a clause.

% In order to pass to the full \elpi language, we need to change the way terms and clauses are 
% represented. We give, therefore, their implementation below.

% \begin{coqcode}
%   Inductive tpos := 
%     | a (i:base)            
%     | impl (c:tneg) (b:tpos)   
%     | pi   (n:string) (b: tpos)
%     | conj (l:list tpos)       
%     | cut                      
%     | unif  (x:any) (y:any)    
%     | match (x:any) (y:any)    
%   with tneg :=
%     | hb  (h: base) (b: list tpos)
%     | piN (x: string) (b : tneg)  
%   with base :=
%     | c (c : pn) (l:list any)
%     | v (v : vn) (l:list any)
%   with any := 
%     | p (i:tpos) 
%     | n (i:tneg) 
%     | b (i:base) 
%     | s (i:string).
% \end{coqcode}

% NOTA: mi sembra sia più facile avere predicati di arità qualsiasi
In order to represent these two new constructs of the language,\todo{pred of arity N}
we add two new constructors to the inductive \coqIn{atom}: \todo{aggiungere lam-abstr?}
\begin{coqcode}
  | Impl (c:clause) (a:atom)
  | Pi (x:vn) (a:atom)
\end{coqcode}

The former constructor is noted with the syntax \implCmd{c}{a} and introduces
the clause \texttt{c}\todo{$c$ is mutual-excl check} locally in the scope of an atom \texttt{a}, by convention
this new inserted clause has the highest priority wrt the clauses existing in
the scope of \coqIn{a}. The \coqIn{pi} constractor is noted \piCmd{x}{a}
introduces the fresh variable nominal variable \coqIn{x} in \coqIn{a}.

A further modification to our language is the \coqIn{tm} type. Now, a term
consists not only of primitive data but also of atoms and clauses. Unlike
functional languages, logic programs do not mix commands and expressions; an
expression is not executable code, whereas a command, called a proposition, is.

In basic \elpi, predicates had expressions as parameters. In this new setting,
with the extension of \coqIn{tm}, a predicate can receive (or return) a command
as an argument -- a command that can be executed as a new piece of code.\todo{mettere esempio?}

% Moreover, from now on we consider that applied predicates are terms, therefore a
% predicate can be passed as argument of another predicate call. The type of a
% fully applied predicate is the type of propositions and is noted \elpiIn{prop}.
% This means that a clause can have the shape ``\elpiIn{apply X Y :- X Y}'': the\todo{The arity of X is 1, it is not a binary predicate}
% output of the predicate \elpiIn{apply} is deduced by calling \elpiIn{X} to
% \elpiIn{Y}. Moreover, \elpiIn{X} is a higher-order variable of type \elpiIn{A ->
% prop}.



\subsection{Interpreter}

To keep into account these two new constructs of the language,
we extend the interpreter. The new rules are depicted in \cref{fig:full-interp}

\begin{figure}
  \ruleImplM{.85}
  
  \rulePiM{.85}
  \caption{Higher-order derivation rules}
  \label{fig:full-interp}
\end{figure}

The notation $\clause + \prog$ stands for the insertion of the clause \clause in the program \prog so that \clause has
the highest priority in \prog. The rule \ruleImpl deals with the atom
\implCmd{H}{B} and its derivation rule tries to solve \elpiIn{B} under a
program \prog which has been expanded with the $H$ clause. The rule \rulePi
inserts the name of the fresh variable in the program. The \texttt{local}
predicate is a special unary predicate having as output the new inserted
universally quantified. The \unify and \match procedures use \texttt{local} to
make the \pf\ unification since this unification on higher-order variables works
on variables applied to a distinct list of local variables.

\subsection{Determinacy in higher-order context}

Since the parameters of predicates can be passed both as input and output to
predicate calls, we need to extend the syntax of signatures so that the user can instruct the
\elpi's compiler\todo{compiler vs interpreter in \elpi} about if an input (resp.
an output) should be considered as a deterministic proposition.

% \begin{elpicode}
%   pred divisor i:int, o:int.
%   pred help i:int, i:int, o:int.

%   divisor N R :- help N 1 R.
%   help N X X :- 0 is N mod X.
%   help N X R :- X < N, X' is X + 1, help N X' R. 

%   [(pi N R\ divisor N R :- help N 1 R),
%   (pi N X\ help N X X :- 0 is N mod X),
%   (pi N X R X'\ help N X R :- X < N, X' is X + 1, help N X' R)] => divisor 10 R. 
% \end{elpicode}

From now on let \elpiIn{divisor} be the predicate from \elpiIn{int} to
\elpiIn{int} taking an integer and returning its divisors, one by one in
increasing order.

Let \elpiIn{once} be a predicate with only one clause ``\elpiIn{once P (P, !)}''.
From this implementation, we see that the \elpiIn{once} predicate has
the following property: it takes a proposition and returns its deterministic
version, i.e. the output is the call to \elpiIn{P} where all of its choice
points have been cut away (note the presence of the \cut\ in the second argument). To be more concrete, the query
\elpiIn{once (divisor 5 X) P, P} will instantiate \elpiIn{X} only once
to \elpiIn{1} (note that the query \elpiIn{divisor 5 X}
has two solutions for \elpiIn{X}, namely \elpiIn{1} and \elpiIn{5}). Another
information about \elpiIn{once} is that it is a deterministic predicate, that
is, not only its output is a function but \elpiIn{once} itself is a function.

To provide the compiler all of these pieces of information about the predicate
\elpiIn{once}, the user should add a signature to the predicate. 
\Elpi's signatures are as follows:
GRAMMAR

TODO: dire che func sta per funzionale e rel sta per relazionale

So the signature of \elpiIn{once} will be: ``\elpiIn{func once i:pred, o:func}''
where \elpiIn{once} is a function, it takes a
relation and returns a function.

% Since predicate parameters can now be passed to other predicate calls, and since
% these calls may affect the determinacy of a predicate, we need to specialize
% our determinacy checker.
\begin{elpicode}
  func apply5 i:(func int -> int) o:int.
  apply5 F R :- F 5 R.
\end{elpicode}

Consider now, the \elpiIn{apply5} predicate depicted above. It is a function
that takes a function from \elpiIn{int} to \elpiIn{int} and returns the result of
applying that function to \elpiIn{5}. What happens if we
execute the query \elpiIn{apply
divisors X}? We see that its output is not deterministic: the assignment for
\elpiIn{X} is not deterministically determined -- there are, in fact, two
possible assignments: one where \elpiIn{X} is assigned to \elpiIn{1} and another
where it is assigned \elpiIn{5}.

Another problem is showed by the followig code:
\begin{elpicode}
  func idF i:(func A -> B), o:(func A -> B).
  idF X X.
\end{elpicode}
  % func p i:(func int -> int), o:(pred i:int, o:int).
  % p X X :- X = divisors.

In the snippet, we define the identity function over functions.

With \elpi's dynamic mode system, the call ``\elpiIn{idF Y Y, Y = divisor, Y 5
X}'' succeeds because we can pass flexible terms in input positions. The problem
here is that, based on the signature of \elpiIn{idF}, its output is a function.
This allows us to assume that the variable \elpiIn{Y} is a function. However,
when \elpiIn{Y} is unified with the \elpiIn{divisor} predicate, we lose that
property: the goal \elpiIn{Y 5 X} produces several results.

To address this issue, we could statically detect and prevent unification
between terms with mismatched determinacy annotations by raising fatal errors.
In the example above, \elpiIn{Y = divisor} would be rejected, as each variable
would carry a tag indicating its determinacy. This approach would work similarly
to static mode checking, tracking the flow of input-output terms while
preserving the labeling procedure.

\newcommand{\detT}[1]{\text{\mlIn{~\PYG{n+nc}{#1}~}}}

The labels we put on variables are essentially $3$: 1) \detT{Det} representing
relations, 2) \detT{Fun} representing functions and 3) \detT{Exp} representing
expressions, i.e. non exectuable terms. \todo{det\&fun have args.}

Although this solution is promising, it is not particularly appealing to us.
% Instead, our approach is to interpret a predicate's signature as a property conditional
% property of the predicate. That is, if a predicate is called with arguments
% corresponding to its signature, then its output will repsect the expected
% output signature. On the other hand, since no guarantee can be given on the
% determinacy of the output, the label of the output will be considered to have
% the lowest determinacy constraints.
%
% Moreover, when unification between two terms is performed, the 
%
Our idea is that the family of deterministic predicates is a subset of the
family of relational predicates. %That is, \elpiIn{divisor} is relational,
% \elpiIn{once} and \elpiIn{apply5} are functional, and therefore relational. 
% This to say, that we should be allowed to pass a function in place of a
% relation, invalidating our first determinacy-checking algorithm. Note also that
% we can go further since we can have higher level of determinacy, namely, we may
% pass terms with type \elpiIn{func x (func int -> int) -> (pred i:int, o:int)} to
% arguments with expected determinacy \elpiIn{pred y (func int -> int) -> (func
% int -> int)}
This definition of ordering can be improved to worked with arguments with nested
arrow types.

Below we define the partial ordering relation $\leq$ between two given
determinacy properties.\todo{if preds are binary remove forall2, otherwise what is forall2?}
\begin{myminted}[ocaml]
let rec a ~$\leq$~ b :=
  match a, b with
  | Rel, Func -> false
  | Func, Rel | Rel, Rel | Func, Func -> true
  | Exp, Exp -> true
  | a ~\texttt{->}~ b, a' ~\texttt{->}~ b' -> List.forall2 ~($\leq$)~ a' a && List.forall2 ~($\leq$)~ b b'
\end{myminted}

\todo{add the no prop case}
The base cases are quite intuitive: a relation is not less than a function. Note
that $\leq$ is not complete, as we expect type-checking to validate the
program before performing determinacy checking. Therefore, it is not possible to
have \detT{Exp} compared with any other determinacy type except \detT{Exp}. 
For the same reason an arrow can never appear against a \detT{Rel}, \detT{Func}
or \detT{Exp}.

The interesting case is the arrow type. Here, we are comparing two functions,
Concerning inputs, the relation $\leq$ is satisfied if the inputs of the left
are at least as restrictive as the right one and if the outputs of the left are
at most as restrictive as the right one.

As an example consider two terms $t_1 \coloneq \texttt{idF}$ and $t_2 \coloneq
\texttt{once}$. The determinacy of $t_1$ is $\detT{Func} \to \detT{Func}$ and
the determinacy of $t_2$ is $\detT{Rel} \to \detT{Func}$. We intuitively remark
that $\forall p_1\ p_2, \texttt{once}\ p_1\ p_2 \leq \texttt{idF}\ p_1\ p_2$.\todo{più o meno}

Thanks to the $\leq$ relation we can define a min (resp. max) relation between
two determinacy relations returning the minimal (resp. max) determinacy relation
between them.

\begin{myminted}[ocaml]
  let rec min a b = match a, b with
    | a ~\texttt{->}~ b, a' ~\texttt{->}~ b' -> List.map2 max a' a && List.map2 min b b'
    | a, b -> if a ~$\leq$~ b then a else b
  and max a b = match a, b with
    | a ~\texttt{->}~ b, a' ~\texttt{->}~ b' -> List.map2 min a' a && List.map2 max b b'
    | a, b -> if a ~$\leq$~ b then b else a
\end{myminted}

$$-----$$

For instance, we would like that the call \elpiIn{idF divisors X} is valid for
determinacy, but, since the input has not the expected determinacy, then the
output will not be a deterministic predicate. Indeed, we propose the following
ordering relation between deterministic and relational predicates.

TODO: presentare l'algoritmo di static check of determinacy with varaible flow.

Dare un esempio in cui l'input ground non è garantito e in cui il checkin viene
invalidato

\subsection{Mode checking: input-ground mode}
\label{sec:mc}
Dire come funziona il mode checking e spiegare che è una condizione sufficiente
per avere la determinacy.

\subsection{Determinacy checker}
Polarity check: inner clauses when using the impl construct

Da spostare prima della sezione \cref{sec:mc}

\subsection{Input-ground polymorphism}

Il problema è ad esmepio la funzione identità

\begin{elpicode}
  pred id i:A, o:A.
\end{elpicode}

Se id viene chiamato con una deterministic-predicate, allora l'output sarà
anch'esso deterministic. Il problema è che non è sempre vero che la determinacy
è passata tra due oggetti che hanno lo stesso tipo.

Per esempio:

\begin{elpicode}
  pred wrong_id i:A, o:A.
  wrong_id fun rel.
\end{elpicode}

Per l'identià si potrebbe definire versioni specializzate:

\begin{elpicode}
  pred id1 i:(func A -> A) (func A -> A).
  id1 X X.
\end{elpicode}

Ma ovviamente questo è tedioso, ci vorrebbe la possibilità di avere un
polimorphismo di determinacy.
Qualcosa tipo:

\begin{elpicode}
  pred id2 i:X~$^\texttt{Y}$~, o:X~$^\texttt{Y}$~.
\end{elpicode}

Dove $^\texttt{Y}$ trasmette l'informazione che i due argomenti di id2 hanno
la stessa funzionalità.