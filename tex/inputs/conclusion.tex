\section{Conclusion and related works}

SE FACCIAMO PI IMPL:
Unlike other prolog system, such as \mercury \cite{1996Somogy}, \elpi is not a
compiled language and we do not transform the list $\mathcal{L}$ of clauses of a
predicate $p$ in super-homogenous form, i.e. a sole clause containing the
disjunction of $\mathcal{L}$. This is mainly due to the fact that \elpi is an
homoiconic language and its program definition can change during the evaluation
of the code.

In \cite{1991Sahlin}, and more formally in \cite{1996mogensen}, determinacy is
used to work with \mixtus, a partial evaluator of \prolog. In that case,
determinacy is inferred so that it is possible to derive a new specialiezed,
and therefore more efficient, version of the original program under the
guarantee that the two program share the same semantics.

Finally, in \cite{1996henderson}, a determinacy checker for \mercury
captures different behaviours of a predicate. A predicate can return
exactly zero and/or one solution, zero and/or multiple solution. In \mercury the
user is allowed to annotate predicates with determinacy information. A
non-annotated predicate will be inferred with the lowest derived tag.
Determinacy, in \mercury, besides giving a guarantee on the program, allows to
specialized it so that a faster routine can be used in the compiled program.

Determinacy, as previously mentioned, is the property of a predicate that
returns at most one solution per call. Such a predicate behaves like a function,
which is why we refer to it as a deterministic predicate or simply a function.
Determinism checking statically ensures that the clauses implementing a
deterministic predicate adhere to this condition. The literature contains
numerous discussions on this topic, offering various descriptions and
applications of determinism.

In \cite{1989Warren}, the authors describe a property subsuming
determinism: they describe functionality. A predicate is
functional if it produces at most one \textit{distinct} solution per predicate
call. The keyword here is \textit{distinct}, since, in the determinacy setting,
a predicate call producing the same solution twice is not considered as
deterministic, while, it is functional. %In the paper they explain that mutual
% exclusivness of clauses can be improved not only by looking at the head and at
% the presence of the cut but also by instructing the checker that premises can
% put clauses in mutual exclusivness.



\cref{tab:comparison}

\begin{table}[h]
  \centering
  \begin{tabular}{c|c|c|c|c|c}
    Paper       & Mode check        & Determincy check & Hard Cut & Harrop & HO unif   \\
    \hline
    Mercury     & Compile           & \cmark           & \xmark   & \xmark & \xmark    \\
    Mixtus      & Compile           & \cmark           & \cmark   & \xmark & \xmark    \\
    Simple Elpi & Runtime           & \cmark           & \cmark   & \xmark & \xmark    \\
    Full Elpi   & Runtime + Compile & \cmark           & \cmark   & \cmark & Partially \\
  \end{tabular}

  \caption{Comparison}
  \label{tab:comparison}
\end{table}


\subsection{Future work}

\paragraph{Problem of charging local clauses with weaker conditions}
Overlapping check che rompe:

\begin{elpicode}
  pi x\ (pi Y\ f x Y) => (pi y => f x y) => Bo
\end{elpicode}

Dove i modi per f sono input, input.

In questo esempio se Bo = ``f x y'' allora ci sono due soluzioni,
in quanto entrambe le regole caricate colla freccia si applicherebbero 