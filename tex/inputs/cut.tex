\section{The role of the cut}

Here we give a definition of our cut: \cite{2003Andrews}

Logic programs are known for their non-deterministic behavior. The could be
several distinct ways to derive a query from a base of knowledge and logic
programs aims to find these all of this solution. 
% We have put quotations around
% the words non-deterministic since of course the interpreter of the prolog system
% has to make some choice about the order in which the rules of the program are
% treated. In our case, we consider that the intepreter takes the rules in a
% chronological way wrt the order in which rules are inserted in the database.

If non-determinism is a key feature, sometimes it is important to let the user
control if and when alternatives have to be rejected. The cut operator is meant
to address this problem. It is often used, for example, to provide an
implementation for the \textit{if-then-else} construct. 

Since we use the elpi dialect of lambda-prolog, it is important to clarify how
the cut behaves in our development. It is in case that each prolog-ish language
has its own cut implementation, for example in the official page of
swi-prolog\footnote{\href{www.swi-prolog.org}{www.swi-prolog.org}} we see that
there are two different kind of cut implementation: the `soft cut' noted with
``\texttt{A *-> B ; C}'' runs C if A has no solution otherwise the result is the
same as running the conjunction of A and B. The `hard cut' noted with the
``\texttt{!}'', "discards all choice points created since entering the predicate
in which the cut appears".

In mercury \cite{1996Somogy}\dots Dire anche che molti usano la super-homogenous
forms invece che regole distinte ma per noi non va bene visto che il programma
può essere esteso a piacimento

It is also interesting to see that further representations of cut may exists,
such as the firm cut explained in \cite{2003Andrews} where they provide a
restricted version of the hard cut which has some concistency properties.

Elpi implementation uses the hard cut definition. Even though the hard cut have
no equivalent representation from pure logic, we are convinced that from a more
more programming point of view it is quite practical to cut away not only the
alternative implementation of a predicate but also to hinder backtracking of all
choice points born from the begin of the clause-body to the current cut
position.