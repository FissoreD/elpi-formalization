
\section{Determinacy analysis}\label{sec:thm}

% \begin{definition}[Functionality (\isfunc)]\label{def:is-func}
%   Given a program \prog and a term $t$,
%   \isdet\ is defined as follows:
%   $$\forall a\ \subst, \isFuncCmd{t}$$
% \end{definition}

We are interested in identifying predicates that leave no choice points
called \emph{semidet} in \cite{1996henderson}, or in the wording of \cite{nakamura1986}
predicates that are \emph{operationally deterministic}. 
% Due to the lack of space, we leave the
% proofs of the following theorems in the appendix.

\begin{definition}[Operational determinacy (\isdet)]\label{def:is-det}
  Given a program \prog and a term $\predt$,
  \isdet\ is defined as follows:
  $$\forall a\ \subst, \isDetCmd{\predt}$$
\end{definition}

% \noindent Said in other words, a predicate call \isdet if its execution does not
% create choice points. 
% Without loss of generality, we can rephrase the lemma in
% the following way:

% \begin{definition}[Operational determinacy empty prog (\isdetE)]\label{def:is-det-empty}
%   The execution of an atom $t$ is deterministic iff
%   $$\forall a\ \subst, \isDetCmd[\EmptyList]{t}$$
% \end{definition}

% \noindent The main difference from the first definition is that we begin with an
% empty program and handle a generic term \( t \). Notably, due to the implication
% operator, this second definition remains equivalent to the first, as local
% clauses can be dynamically loaded during the evaluation of \( t \), thereby
% reconstructing the same program referenced in \cref{def:is-det}.  

\def\atomtoprog{\ensuremath{\mathcal{T}}}

% \begin{definition}[Split atom (\atomtoprog)]
%   Given $\Ainst = \piImplCmd[ty_1]{x_1}{c_1}{\dots \impl\
%   \piImplCmd[ty_n]{x_n}{c_n}{s}}$, then $\atomtoprog(\prog) = \langle \{c_n,\cdots,c_1\}, s \rangle$ 
% \end{definition}

% \begin{lemma}
%   Given an atom $\Ainst$ and let $(\prog, \Ainst') = \atomtoprog(\Ainst)$, then
%   $$\isdetE\ \Ainst \Rightarrow \isdet\ \prog\ \Ainst'$$
%   \label{th:run-empty}
% \end{lemma}
We assign to signatures a logic formula, a formal meaning, by recursion.

% Thanks to the definition of \isdet, we can provide the semantics of determinacy
% signatures.
% The intuition is given by the following formula.

\begin{definition}[Meaning of a signature $\sem{\func}\ (\predt)$ given a context \ctx and a program \prog]\label{def:sem}
\vspace{-1em}
\newcommand{\funcF}[1][]{\ensuremath{\mathcal{F}#1}}
% \begingroup\makeatletter\def\f@size{8}\check@mathfonts
$$
\arraycolsep=1.4pt
\begin{array}{rll}
  \sem{\detI}\ (\predt) = & %
      \bigwedge \vecL{\semi{\func_i}\ t_i} \Rightarrow \isdet\ \prog\ {(\predt)} \bigwedge \vecL{\semi{\func_o}\ t_o} %
      & \quad\mathrm{s.t.\ } \ctx\ \pred{} = \dtype{\detI}{\func_i}{\func_o} \\
\end{array}
$$
$$
\arraycolsep=1.4pt
\begin{array}{ccc}
  \semi{\detI}\ (\predt) = \isdet\ \prog\ (\predt) &
  \qquad \semi{\relI}\ (\predt) = \top &
  \qquad \semi{\expI}\ (\predt) = \top \vspace{0.5em}
\end{array}
$$
In order to handle partial application like \elpiIn{once likes} in the introduction
we consider the following extra rules:
$$
\arraycolsep=1.4pt
\begin{array}{rl}

  \semi{\dtype{\detI}{\func_i}{\func_o}}\ (\predt) = & %
    \forall \vec{u}\ \vec{v}, 
      %|\func[_i]| = |\vec{u}| \land |\func[_o]| = |\vec{v}| %
      %\land 
      \bigwedge \vecL{\semi{\func_i}\ u} \Rightarrow \isdet\ \prog\ {(\pred\ (\vec{t}\ \vec{u}\ \vec{v}))} \bigwedge \vecL{\semi{\funcF}\ t_i}  \\
      & \mathrm{where\ } |\func[_i]| = |\vec{u}| \mathrm{\ and\ } |\func[_o]| = |\vec{v}| %
        \mathrm{\ and\ } \ctx\ \pred{} = \dtype{\detI}{(\funcF \  \func_i)}{\func_o} \vspace{0.5em}\\
  \semi{\dtype{\detI}{\EmptyList}{\func_o}}\ (\predt) = & %
    \forall \vec{v}, 
     %|\func[_o]| = |\vec{v}| %
      %  \Rightarrow  %
        \isdet\ \prog\ {(\pred\ (\vec{t}\ \vec{v}))} \bigwedge \vecL{\semi{\funcF}\ t_i} \\
      & \mathrm{where\ } |\func[_o]| = |\vec{v}| \mathrm{\ and\ } \ctx\ \pred{} = \dtype{\detI}{\funcF}{(\funcF_o \  \func_o)} \vspace{0.5em}\\
  \semi{\dtype{\relI}{\_}{\_}}\ (\predt) = & \top \\
\end{array}
$$
% \endgroup
\end{definition}
The first equation states that if the preconditions of a signature are satisfied
then the call to a predicate of that signature is
functional and the post conditions on the outputs hold.
The contravariance of the \mySub test is reflected by the
fact that for any $t$, if $f_1 \mySub f_2$ then
$\sem{f_1} t \Rightarrow \sem{f_2} t$:
the post-conditions of $f_1$ can be stronger than the ones of $f_2$ while
the pre-conditions of $f_1$ must be weaker than the ones of $f_2$.

We use the meaning of signatures to describe the key invariant of program:
\newcommand{\iprog}[1][\prog]{\ensuremath{\texttt{checked}\ \Gamma\ #1}\xspace}
\begin{definition}[Program invariant \iprog]\label{def:i}
  We say that \iprog{} holds iff for all predicates $\pred \in \prog$ s.t. $\ctx\ \pred = \dtype{\detI}{\_}{\_}$
  then  $\forall \vecL{t}, \sem{\detI}\ (\predt)$.
\end{definition}

Since the program can be dynamically augmented by asserting clauses we need
\checkk\ to preserve the invariant.

\begin{lemma}[Invariant preservation]
  $$\forall\, \Gamma\ \prog\ r, \iprog \land \checkClause{\ctx}{r} \Rightarrow \iprog[(\prog + r)]$$
  \label{th:0}
\end{lemma}

Since we execute atoms we need need this

\begin{theorem}[Deterministic execution]
  Given a context \ctx and an atom $t$, then
  $$\forall\, \Gamma\ \prog\ t, \iprog \land \checkCmd{\ctx}{\detI}{t}{\_}{\detI}
  \Rightarrow \isdet\ \prog\ t$$
  \label{th:1}
\end{theorem}

Note that programs can be encoded into atoms by asserting all clauses dynamically
atop the empty program, that validates the invariant trivially.

% \noindent
% This theorem is crucial for establishing the connection between our determinacy
% checker and the a program execution. Specifically, if our \checkk\ procedure
% verifies that a given atom is \detI, we can guarantee that its interpretation
% will produce no choice points.

% The invariant we preserve in the proof is that \checkk\ ensures $\sem{\detI}\ t$
% while guaranteeing that all the input terms appearing in $t$ respect
% $\semi{\cdot}$.

% \Cref{th:1} can be generalized to work in a context where a concrete program
% \prog is provided (instead of having the initial empty program). This program is
% possible thanks to \cref{th:run-empty}. The program \prog is extracted thanks to
% the \atomtoprog\ function.

% The hypothesis in \cref{th:1} about the fact that local clauses for
% deterministic predicate should have a \cut can be relaxed in some situations.
% The interesting property allowing to remove \cut while respecting the semantics
% of the program is the so-called mutual-clause exclusion.

% \begin{lemma}[\mutExcl]
%   Given a program \(\prog\), it exists an equivalent program \(\prog'\) that
%   contains the minimal number of \cut\ operators. 
% \end{lemma}

% \def\remCut{\ensuremath{\mathcal{R}}}

% \begin{definition}[Remove redundant \cut (\remCut)]
%   Given a program \prog, we can build a program $\prog'$, which is a copy of
%   \prog where for each clause \( c = \clauseCmd{\pred}{\vec{t}}{\vecL{bs}, \cut,
%   \vecL{ds}} \) in \(\prog\), if all premises in \(\vecL{bs}\) are valid calls to
%   deterministic predicates and $\mutExclAll\ \prog\ c$, then
%   the \cut\ can be safely removed.
% \end{definition}

% Finally, we can generalize \cref{th:1} as follows.

% \begin{theorem}[Deterministic execution] \label{th:2}
%   Given a predicate context \ctx, a program \prog\ and a predicate \pred, s.t.
%   $\ctx\ \pred = \dtype{\detI}{\_}{\_}$, then
%   $$
%   \mutExclAll\ \ctx\ \prog \land
%   (\forall c \in \prog, \checkc\ \ctx\ c) \land
%   \checkk\ \ctx\ (\pred\ \vec{t})
%   \Rightarrow
%   \isdet\ \prog\ (\pred\ \vec{t})
%   $$
% \end{theorem}

% \begin{proof}
%   We can prove this final lemma thank to \cref{th:run-empty}, combining the

% \end{proof}

\paragraph{\textbf{Digression on \isdet}.}

In the logic programming literature, the notion of a ``function'' does not
necessarily align with the definition we adopt in \isdet. In particular, a
program that repeatedly produces the same result infinitely many times
undeniably represents a mathematical function. However, from a programmer's
perspective, such behavior is fundamentally different from that of a program
that terminates after returning a single result. Our definition intentionally
captures the latter, as it better reflects the computational complexity of
programs, particularly in cases of failure (as also observed by \cite{1989Vink}).

\begin{elpicode}
func silly list int -> int.
silly L V :- mem 1 L, len L V.
\end{elpicode}

Given the signature of \elpiIn{silly} the program is accepted only
if both both \elpiIn{mem} and \elpiIn{len} are observationally
deterministic hence the code is operationally
equivalent to the following one whose complexity is clealy
the sum of its parts.

\begin{elpicode}
func do! (pred).
do! P :- P, !.                 % encoding of if-then cut using Elpi's hard cut
silly L V :- do! (mem 1 L), do! (len L V).
\end{elpicode}

The complexity could have been, on failure, the product of its parts if we had chosen a more
relaxed definition of function, since the programmer could have called a
(cut free) implementation of \elpiIn{mem} that is allowed to give multiple
times the same result.
% \todo{anche in \href{https://repub.eur.nl/pub/1507/eur-few-cs-89-03.pdf}{1989Vink}, dicono simile}
% \todo{una delle robe di mercury può fare lo stesso?}

\paragraph{\textbf{Digression on  the eagherness of $\mathcal{H}$}}
\label{sec:digression-h}

The function $\mathcal{H}$ in the operational semantics (Figure~\ref{fig:basic-interp})
eagerly unifies all clauses with the goal.
A more natural and possibly more efficient semantics would be to just create the
alternatives and prepend to the list of goals the same matching/unification
problems, but that would not satisfy \isdet.

From a practical stanpoint an efficient implementation of that semantics
can make $\mathcal{H}$ lazy without compromising the \isdet property
if either it indexes clauses deep enough
or performs a program transformation consisting in inserting the
\elpiIn{do!} defined above around each call to a deterministic predicate
(as we manually did in the example above).

